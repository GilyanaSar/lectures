% Размер страницы и шрифта
\documentclass[12pt,a4paper]{article}

%% Работа с русским языком
\usepackage{cmap}                   % поиск в PDF
\usepackage{mathtext}               % русские буквы в формулах
\usepackage[T2A]{fontenc}           % кодировка
\usepackage[utf8]{inputenc}         % кодировка исходного текста
\usepackage[english,russian]{babel} % локализация и переносы

%% Изменяем размер полей
\usepackage[top=0.5in, bottom=0.75in, left=0.625in, right=0.625in]{geometry}

%% Различные пакеты для работы с математикой
\usepackage{mathtools}  % Тот же amsmath, только с некоторыми поправками
\usepackage{amssymb}    % Математические символы
\usepackage{amsthm}     % Пакет для написания теорем
\usepackage{amstext}
\usepackage{array}
\usepackage{amsfonts}
\usepackage{icomma}     % "Умная" запятая: $0,2$ --- число, $0, 2$ --- перечисление

%% Графика
\usepackage[pdftex]{graphicx}
\graphicspath{{images/}}

%% Прочие пакеты
\usepackage{listings}               % Пакет для написания кода на каком-то языке программирования
\usepackage{algorithm}              % Пакет для написания алгоритмов
\usepackage[noend]{algpseudocode}   % Подключает псевдокод, отключает end if и иже с ними
\usepackage{indentfirst}            % Начало текста с красной строки
\usepackage[colorlinks=true, urlcolor=blue]{hyperref}   % Ссылки
\usepackage{pgfplots}               % Графики
\pgfplotsset{compat=1.12}
\usepackage{forest}                 % Деревья
\usepackage{titlesec}               % Изменение формата заголовков
\usepackage[normalem]{ulem}         % Для зачёркиваний
\usepackage[autocite=footnote]{biblatex}    % Кавычки для цитат и прочее
\usepackage[makeroom]{cancel}       % И снова зачёркивание (на этот раз косое)

% Изменим формат \section и \subsection:
\titleformat{\section}
	{\vspace{1cm}\centering\LARGE\bfseries} % Стиль заголовка
	{}                                      % префикс
	{0pt}                                   % Расстояние между префиксом и заголовком
	{}                                      % Как отображается префикс
\titleformat{\subsection}                   	% Аналогично для \subsection
	{\Large\bfseries}
	{}
	{0pt}
	{}

% Поправленный вид lstlisting
\lstset { %
    backgroundcolor=\color{black!5}, % set backgroundcolor
    basicstyle=\footnotesize,% basic font setting
}

% Теоремы и утверждения. В комменте указываем номер лекции, в которой это используется.
\newtheorem*{hanoi_recurrent}{Свойство} % Лекция 1
\let\epsilent\varepsilon                % Лекция 8
\DeclareMathOperator{\rk}{rank}         % Лекция 20

% Информация об авторах
\author{Группа лектория ФКН ПМИ 2015-2016 \\
	Никита Попов \\
	Тамерлан Таболов \\
	Лёша Хачиянц}
\title{Лекции по предмету \\
	\textbf{Алгоритмы и структуры данных}}
\date{2016 год}


\begin{document}

\section{Лекция 7 от 02.02.2016}

\subsection{Умножение чисел. Алгоритм Карацубы}

Пусть \(x = \overline{x_1 x_2 \ldots x_n}\) и \(y = \overline{y_1 y_2 \ldots y_n}\). Распишем их умножение в столбик:
\begin{center}
	\[
	\renewcommand{\arraystretch}{0.8}
	\arraycolsep=1pt
	\begin{array}{r}
	\times\begin{array}{rrrr}
	x_1 & x_2 & \ldots & x_n \\
	y_1 & y_2 & \ldots & y_n \\
	\hline
	\end{array}
	\\
	+\begin{array}{rrrrrrr}
	& & & z_{11} & z_{12} & \ldots & z_{1n} \\
	& & z_{21} & z_{22} & \ldots & z_{2n} & \\
	& \hdotsfor{4} & & \\
	z_{n1} & z_{n2} & \ldots & z_{nn} & & & \\
	\hline
	\end{array}
	\\
	\begin{array}{rrrrrrrr}
	z_{11} & z_{12} & \dots & \dots & \dots & \dots & z_{2n} & z_{2n+1} \\
	\end{array}
	\end{array}\]
\end{center}


Какова сложность такого умножения? Всего \(n\) строк. На получение каждой строки тратится \(O(n)\) операций. Тогда сложность этого алгоритма --- \(nO(n) = O(n^2)\). Теперь вопрос: \emph{а можно ли быстрее?} Один из величайших математиков XX века, А.Н. Колмогоров, считал, что это невозможно.

Попробуем воспользоваться стратегией <<Разделяй и властвуй>>. Разобьём числа в разрядной записи пополам. Тогда
\[\begin{array}{c}
\times \begin{cases}
x = 10^{n/2}a + b\\
y = 10^{n/2}c + d\\
\end{cases} \\
\Downarrow\\
xy = 10^{n}ac + 10^{n/2}(ad+bc)+bd
\end{array}\]

Как видно, получается 4 умножения чисел размера \(\frac{n}{2}\). Так как сложение имеет сложность \(\Omega(n)\), то

\[T(n) = 4T\left( \frac{n}{2} \right) + \Theta(n)\]

Чему равно \(T(n)\)? Воспользуемся основной теоремой. Напомним: в общем виде неравенство
имеет вид:

\[T(n) \leqslant aT\left( \frac{n}{b} \right) + cn^d\]

В нашем случае \(a = 4, b = 2, d = 1\). Заметим, что \(4 > 2^1 \implies a > b^d\). Тогда \(T(n) = O(n^{\log_2 4}) = O(n^2)\).

Как видно, it’s not very effective. Хотелось бы свести число умножений на каждом этапе к
трём, так как это понизит сложность до \(O(n^{\log_2 3}) \approx O(n^{1.58})\)Но как?

Вернёмся к началу. Разложим \((a + b)(c + d)\)

\[(a+b)(c+d) = ac+(ad+bc) + bd \implies ad + bc = (a + b)(c + d) - ac - bd\]

Подставим это в начальное выражение для \(xy\):

\[xy = 10^{n}ac + 10^{n/2}((a + b)(c + d) - ac - bd)+bd\]

Отсюда видно, что достаточно посчитать три числа размера \(\frac{n}{2}\): \((a + b)(c + d), ac\) и \(bd\). Тогда:

\[T(n) = 3T\left( \frac{n}{2} \right) + \Theta(n) \implies T(n) = O(n^{\log_2 3})\]

Полученный алгоритм называется \emph{алгоритмом Карацубы}.
На данный момент доказано, что для любого $\varepsilon > 0$ существует алгоритм, который совершает умножение двух чисел с сложностью \(O(n^{1 + \varepsilon})\). Также стоит упомянуть \emph{алгоритм Шёнхаге-Штрассена}, работающий за \(O(n \log n \log \log n)\)

\subsection{Перемножение матриц. Алгоритм Штрассена}

Пусть у нас есть квадратные матрицы
\[A = \begin{pmatrix}
a_{11} & a_{12} & \ldots & a_{1n} \\
a_{21} & a_{22} & \ldots & a_{2n} \\
\vdots & \vdots & \ddots & \vdots \\
a_{n1} & a_{n2} & \ldots & a_{nn} \\
\end{pmatrix}
\text{и } 
B = \begin{pmatrix}
b_{11} & b_{12} & \ldots & b_{1n} \\
b_{21} & b_{22} & \ldots & b_{2n} \\
\vdots & \vdots & \ddots & \vdots \\
b_{n1} & b_{n2} & \ldots & b_{nn} \\
\end{pmatrix}\]

Сколько операций нужно для умножения матриц? Умножим их по определению. Матрицу
\(C = AB\) заполним следующим образом:
\[c_{ij} = \sum\limits_{k = 1}^{n} a_{ik}b_{kj}\]

Всего в матрице \(n^2\) элементов. На получение каждого элемента уходит \(O(n)\) операций (умножение за константное время и сложение \(n\) элементов). Тогда умножение требует \(n^2O(n) = O(n^3)\)
операций.

А можно ли быстрее? Попробуем применить стратегию «Разделяй и властвуй». Представим
матрицы \(A\) и \(B\) в виде:

\[A = \begin{pmatrix}
A_{11} & A_{12}\\
A_{21} & A_{22}
\end{pmatrix}
\text{и } 
B = \begin{pmatrix}
B_{11} & B_{12}\\
B_{21} & B_{22}
\end{pmatrix}\]
где каждая матрица имеет размер \(\frac{n}{2}\). Тогда матрица \(C\) будет иметь вид:
\[C = \begin{pmatrix}
A_{11}B_{11}+A_{12}B_{21} & A_{11}B_{12}+A_{12}B_{22}\\
A_{21}B_{11}+A_{22}B_{21} & A_{21}B_{12}+A_{22}B_{22}
\end{pmatrix}\]
 Как видно, получаем 8 перемножений матриц порядка \(\frac{n}{2}\). Тогда

\[T(n) = 8T\left( \frac{n}{2} \right) + O(n^2)\]

По основной теореме получаем, что $T(n) = O\left(n^{\log_{2} 8}\right) = O(n^{3})$.

Можно ли уменьшить число умножений до 7? \emph{Алгоритм Штрассена} утверждает, что можно. Он предлагает ввести следующие матрицы (даже не спрашивайте, как до них дошли):

\[\begin{cases}
    M_1 = (A_{11} + A_{22})(B_{11} + B_{22}); \\
    M_2 = (A_{21} + A_{22})B_{11}; \\
    M_3 = A_{11}(B_{12} - B_{22}); \\
    M_4 = A_{22}(B_{21} + B_{11}); \\
    M_5 = (A_{11} + A_{12})B_{22}; \\
    M_6 = (A_{21} - A_{11})(B_{11} + B_{12}); \\
    M_7 = (A_{12} - A_{22})(B_{21} + B_{22}); \\
\end{cases}\]
Тогда
\[\begin{cases}
    C_1 &= M_1+M_4-M_5+M_7; \\
    C_2 &= M_3+M_5; \\
    C_3 &= M_2+M_4; \\
    C_4 &= M_1-M_2+M_5+M_6; \\
\end{cases}\]

Можно проверить что всё верно (оставим это как \sout{наказание} упражнение читателю). Сложность алгоритма:

\[T(n) = 7T\left( \frac{n}{2} \right) + O(n^2) \implies T(n) = O\left(n^{\log_{2} 7} \right)\]

На данный момент один из самых быстрых алгоритмов имеет сложность \(\approx O(n^{2.3})\) (\emph{ал-
горитм Виноградова}). Но этот алгоритм быстрее только в теории — из-за астрономически
огромной константы.
\end{document}
