% Размер страницы и шрифта
\documentclass[12pt,a4paper]{article}

%% Работа с русским языком
\usepackage{cmap}                   % поиск в PDF
\usepackage{mathtext}               % русские буквы в формулах
\usepackage[T2A]{fontenc}           % кодировка
\usepackage[utf8]{inputenc}         % кодировка исходного текста
\usepackage[english,russian]{babel} % локализация и переносы

%% Изменяем размер полей
\usepackage[top=0.5in, bottom=0.75in, left=0.625in, right=0.625in]{geometry}

%% Различные пакеты для работы с математикой
\usepackage{mathtools}  % Тот же amsmath, только с некоторыми поправками
\usepackage{amssymb}    % Математические символы
\usepackage{amsthm}     % Пакет для написания теорем
\usepackage{amstext}
\usepackage{array}
\usepackage{amsfonts}
\usepackage{icomma}     % "Умная" запятая: $0,2$ --- число, $0, 2$ --- перечисление

%% Графика
\usepackage[pdftex]{graphicx}
\graphicspath{{images/}}

%% Прочие пакеты
\usepackage{listings}               % Пакет для написания кода на каком-то языке программирования
\usepackage{algorithm}              % Пакет для написания алгоритмов
\usepackage[noend]{algpseudocode}   % Подключает псевдокод, отключает end if и иже с ними
\usepackage{indentfirst}            % Начало текста с красной строки
\usepackage[colorlinks=true, urlcolor=blue]{hyperref}   % Ссылки
\usepackage{pgfplots}               % Графики
\pgfplotsset{compat=1.12}
\usepackage{forest}                 % Деревья
\usepackage{titlesec}               % Изменение формата заголовков
\usepackage[normalem]{ulem}         % Для зачёркиваний
\usepackage[autocite=footnote]{biblatex}    % Кавычки для цитат и прочее
\usepackage[makeroom]{cancel}       % И снова зачёркивание (на этот раз косое)

% Изменим формат \section и \subsection:
\titleformat{\section}
	{\vspace{1cm}\centering\LARGE\bfseries} % Стиль заголовка
	{}                                      % префикс
	{0pt}                                   % Расстояние между префиксом и заголовком
	{}                                      % Как отображается префикс
\titleformat{\subsection}                   	% Аналогично для \subsection
	{\Large\bfseries}
	{}
	{0pt}
	{}

% Поправленный вид lstlisting
\lstset { %
    backgroundcolor=\color{black!5}, % set backgroundcolor
    basicstyle=\footnotesize,% basic font setting
}

% Теоремы и утверждения. В комменте указываем номер лекции, в которой это используется.
\newtheorem*{hanoi_recurrent}{Свойство} % Лекция 1
\let\epsilent\varepsilon                % Лекция 8
\DeclareMathOperator{\rk}{rank}         % Лекция 20

% Информация об авторах
\author{Группа лектория ФКН ПМИ 2015-2016 \\
	Никита Попов \\
	Тамерлан Таболов \\
	Лёша Хачиянц}
\title{Лекции по предмету \\
	\textbf{Алгоритмы и структуры данных}}
\date{2016 год}


\begin{document}

\section*{Лекция ?? от 15.03.2016}

\subsection{Ассоциативный массив}

Продолжаем говорить про структуры данных. Ассоциативный массив (он же map в C++, он же словарь в Python). Это структура данных такая, что с каждой записью ассоциирован уникальный ключ и реализованы следующие операции:

Insert(S, x)

Delete(S, x)

Find(S, k)

Возможные способы:

Таблица с прямой адресацией:

Ключи берутся из $U = \{0,\ldots, m-1\}$; данные будем хранить в массиве размера m; все операции быстры, ожнако если мы хотим, чтобы ключи были, например, long'ами, то у нас такая таблица в память не поместится вообще никак. А что делать, если $U$ большое, бесконечное, например?

Хеш-функция:

Пусть есть функция $h: U\to \{0, \ldots, m-1\}$. При этом, записывать элемент будем в ячейку $h(k)$. Возможно, что двум ключам соответствует один хеш; тогда возникает \emph{коллизия}, и их можно решать разными способами. Например, можно хранить не элементы, а списки элементов с соответствующим хешем; необходимо только модифицировать наши функции.

\begin{lstlisting}
Find(T, k)
    list := T[h(k)]
    return k in list
\end{lstlisting}

Проблема этого подхода --- случай, когда хеши всех ключей равны. Тогда таблица вырождается в список, а он очень неэффективен. Худший случай для Find --- $O(n)$, в отличие от $O(1)$ в лучшем. А давайте рассмотрим средний случай?

Будем считать, что $h$ хорошая и распределяет $n$ ключей по $m$ хешам примерно равномерно. Пусть $\alpha = \frac{n}{m}$ --- коэффициент заполнения.

Ожидаемое время поиска отсутствующего элемента --- $O(1+\alpha)$. При этом, если $n = O(m)$, то поиск занимает $O()$

Но нам нужна хорошая функция. Где её взять?

Метод деления:

$h(x) = x \pmod{m}$.

Однако, если у нас, например, $m$ чётное, а так сложилось, что мы работаем с чётными числами, то половина ячеек нашей хеш-таблицы будет пустой. (Вообще, обычно $m$ берут простым и всё хорошо)

Метод умножения:

$h(k) = (ak\pmod{2^w}) >> (w-r)$, где $w$ --- длина слова.


А теперь снова к коллизиям. Можно использовать не списки, а таблицу с открытой адресацией, где используются по очереди несколько ($m$) хеш-функций, пока очередной хеш от ключа не окажется незанят. При этом мы хотим хоть какой-то эффективности, поэтому иметь больше чем $m$ функций --- излишне; значит, мы хотим, чтобы для всех $k,\ i\neq j$ выполнялось $h(k, i) \neq h(k, j)$.

Самый простой способ --- имея одну функцию $h'(k)$ определить $h(k, i) = (h'(k) + i) \pmod{m}$. Этот вариант не очень хорош --- будут образовываться большие подряд занятые блоки и они будут замедлять работу.

Лучше сделать так: $h(k, i) = (h'(k) + ih''(k)) \pmod{m}$.

А теперь поговорим об эффективности поиска:

Для каждого ключа все $m!$ перестановок равновероятны.

$\alpha = \frac{n}{m} < 1$

\[
    1+\frac{n}{m}\left(1+\frac{n-1}{m-1}\left(1+\frac{n-2}{m-2}\cdots\right)\right) \leqslant 1+\alpha(1+\alpha(1+\alpha\cdots)) \leqslant \frac{1}{1-\alpha}
\]<++>
\end{document}
