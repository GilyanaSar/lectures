\documentclass[a4paper, 12pt]{article}
\usepackage{header}

\begin{document}
\pagestyle{fancy}
\section{Лекция 01 от 05.09.2016 \\ Основные определения и свойства рядов. Критерий Коши. Необходимое условие сходимости}
\begin{Def}
	Пусть \(\{a_n\}^{\infty}_{n=1}\) --- последовательность действительных чисел. \emph{Числовым рядом} называется выражение вида \(\sum\limits_{n=1}^{\infty}a_n\), записываемое также как \(a_1 + a_2 + \ldots + a_n + \ldots \).
	
\end{Def}
\begin{Def}
	\(N\)-й частичной суммой называется сумма первых \(N\) членов. 
	$$S_n = a_1~+~\ldots~+~a_N$$
\end{Def}

\begin{Def}
	Последовательность \(\{S_n\}^{\infty}_{n=1}\)
	называется последовательностью частичных сумм ряда $\series{1}{\infty}a_n$.
\end{Def}

Говорят, что ряд \textit{сходится} (к числу $A$), если (к числу $A$) сходится последовательность его частичных сумм. Аналогично, ряд \textit{расходится к $+\infty$ (к $-\infty$)}, если к $+\infty$ (к $-\infty$) расходится последовательность его частичных сумм. Если последовательность частичных сумм расходится, ряд называют \textit{расходящимся}.

\begin{Def}
Суммой ряда называется предел $\lim\limits_{n \to \infty} S_n$.
\end{Def}
Вспоминая, что $a_n = S_{n} - S_{n-1}$, можно заключить, что особой разницы между самим рядом и последовательностью его частичных сумм нет --- из одного можно получить другое и наоборот. Следовательно, вместо ряда можно рассматривать его частичные суммы.

\begin{Examples}[Предел Коши для последовательностей]
	Последовательность $\{S_n\}_{n=1}^\infty$ сходится тогда и только тогда, когда она удовлетворяет условию Коши, т.е.
	$$
	\forall \varepsilon>0\; \exists N\in \N\colon \forall m,k > N \Rightarrow |S_m - S_k|<\varepsilon.
	$$
\end{Examples}
Таким образом, мы нахаляву получили первую теорему.
\begin{Theorem}[Критерий Коши сходимости ряда]
	Для сходимости ряда $\sum\limits_{n=1}^{\infty} a_n$ необходимо и достаточно, чтобы 
	$$
	\forall \varepsilon>0\; \exists N\in \N\colon \forall k>N,\; \forall p\in \N \Rightarrow  |a_{k+1} + a_{k+2} \ldots + a_{k+p}| < \varepsilon.
	$$
	
\end{Theorem}
Отсюда сразу же очевидно следует утверждение.
\begin{Statement}[Необходимое условие сходимости ряда]
	Если ряд $\sum\limits_{n=1}^{\infty} a_n$ сходится, то $\lim\limits_{n\to \infty} a_n~=~0$.
\end{Statement}
\begin{proof}
	Ряд сходится, значит,
	$$
	\forall \varepsilon>0\; \exists N \in \N\colon \forall k>N, p = 1 \Rightarrow |a_{k+1}| < \varepsilon.
	$$
	А это и есть определение предела, равного нулю.
	\par Другой способ доказательства: вспомним, что $a_n = S_n - S_{n-1}$ и что $S_n$, как и $S_{n-1}$, стремятся к одному пределу при стремлении $n$ к бесконечности. Итого, получаем, что 
	$$
	\lim\limits_{n \to \infty} a_n = \lim\limits_{n\to \infty} S_n - \lim\limits_{n\to \infty} S_n = 0.
	$$
\end{proof}

Теперь сформулируем и докажем несколько тривиальных свойств.
\begin{Properties}
	Пусть $\sum\limits_{n = 1}^{\infty}a_n = A$, $\sum\limits_{n = 1}^{\infty}b_n = B$. Тогда $\sum\limits_{n = 1}^{\infty}\left(a_n + b_n\right) = A + B$.
\end{Properties}
\begin{proof}
	Это напрямую следует из свойств предела последовательности и того, что $S^{a+b}_n = S_n^a + S_n^b$.
\end{proof}
\begin{Properties}
	Пусть $\sum\limits_{n = 1}^{\infty} a_n~=~A$. Тогда $\sum\limits_{n = 1}^{\infty} \alpha a_n = \alpha A$ для любого действительного $\alpha$.
\end{Properties}
\begin{proof}
	Аналогично вытекает из свойств предела последовательности.
\end{proof}

Введём еще одно определение.

\begin{Def}
	Пусть дан ряд $\sum\limits_{n=1}^{\infty}a_n$. Обозначим некоторые его подсуммы,
	$$
	\underbrace{a_1 + \ldots + a_{n_1}}_{b_1} + \underbrace{a_{n_1+1} + \ldots + a_{n_2}}_{b_2} + \underbrace{a_{n_2 + 1} \ldots + a_{n_3}}_{b_3} + a_{n_3 + 1} + \ldots,
	 $$
	 где $\{n_j\}_{j=1}^{\infty}$ --- возрастающая последовательность натуральных чисел. В таком случае говорят, что ряд $\sum\limits_{k =1}^{\infty} b_k$ получен из исходного \emph{расстановкой скобок}.
\end{Def}
\begin{Statement}
	Если ряд сходится или расходится к $\pm \infty$, то после любой расстановки скобок он сходится, неформально говоря, туда же.
\end{Statement}

\begin{proof}
	Достаточно заметить, что частичные суммы ряда, полученного расстановкой скобок, образуют подпоследовательость в последовательности частичных сумм исходного ряда:
	$$
	S^b_1 = S^a_{n_1}, \quad S^b_{2} = S^a_{n_2}, \quad S^b_3 = S^a_{n_3}, \quad \ldots
	$$
	Осталось только вспомнить, что любая подпоследовательность сходящейся последовательности сходится туда же, куда и сама последовательность.
\end{proof}
\emph{Обратное неверно!!!} Пример такого ряда:
$$
1 - 1 + 1 - \ldots = \sum\limits_{n = 0}^{\infty} \left(-1\right)^n.
$$
При расстановке скобок $(1 - 1) + (1 - 1) + \ldots = 0$ получается сходящийся ряд, в то время как исходный ряд расходится, хотя бы потому что не выполняется необходимое условие о стремлении членов ряда к нулю.

Однако сходимость элементов к нулю не единственное препятствие. Например, можно <<распилить>> единицы из предыдущего примера и получить следующий ряд:
$$
1 - 1 + \frac{1}{2} + \frac{1}{2} - \frac{1}{2} - \frac{1}{2} + \frac{1}{3} + \frac{1}{3} + \frac{1}{3} - \frac{1}{3} - \frac{1}{3} - \frac{1}{3} + \frac{1}{4} + \ldots
$$
Его элементы стремятся к нулю, но он все еще расходится. Однако расставив скобки, можно получить сходящийся ряд:
$$
(1 - 1) + \left(\frac{1}{2} + \frac{1}{2} - \frac{1}{2} - \frac{1}{2}\right) +\left( \frac{1}{3} + \frac{1}{3} + \frac{1}{3} - \frac{1}{3} - \frac{1}{3} - \frac{1}{3}\right) + \ldots = 0.
$$
\begin{Statement}
	Если $a_n \to 0$ и длины скобок ограничены (т.е. существует такое $C~\in~\R$, что $n_{k+1} - n_{k} < C$ при всех $k$), то из сходимости ряда, полученного расстановкой таких скобок, следует сходимость исходного ряда.
\end{Statement}
\begin{proof}
Доказать предлагается самостоятельно. Указание: ограничить через $\frac{\eps}{C}$.
\end{proof}
\begin{Statement}
	Изменение, удаление или добавление конечного числа членов ряда не влияет на его сходимость.
\end{Statement}

Поговорим теперь об абсолютной сходимости.
\begin{Def}
	Если сходится ряд $\sum\limits_{n = 1}^{\infty}|a_n|$, то говорят, что ряд $\sum\limits_{n = 1}^{\infty}a_n$ сходится абсолютно.
\end{Def}
\begin{Def}
	Если ряд сходится, но не сходится абсолютно, то говорят, что ряд сходится условно.
\end{Def}
\begin{Statement}
	Если ряд $\sum\limits_{n=1}^{\infty}a_n$ сходится абслютно, то он сходится.
\end{Statement}
\begin{proof}
	Сразу следует из критерия Коши. Возьмём произвольное $\varepsilon>0$. Так как ряд из модулей сходится, то $$\exists N\in \N\colon \forall k>N,\; \forall p\in \N \Rightarrow \sum\limits_{k+1}^{k+p}|a_k| < \varepsilon$$
	Тогда $$\left| \sum\limits_{n=k+1}^{k+p}a_n\right| \leqslant \sum\limits_{n=k+1}^{k+p}|a_n| < \varepsilon$$
\end{proof}

\begin{Def}
	Для ряда $\sum \limits_{n=1}^{\infty}a_n$ $N$-м хвостом называется сумма $r_N = \sum \limits_{n=N+1}^{\infty}a_n$.
\end{Def}
Для сходящегося ряда очевидно, что каждый его хвост сходится.
\end{document}