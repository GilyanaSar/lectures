\documentclass[a4paper, 12pt]{article}
\usepackage{header}

\begin{document}
\pagestyle{fancy}
\section{Лекция 03 от 19.09.2016 \\Признаки сходимости знакопостоянных и знакопеременных рядов}

\subsection{Граница между сходящимися и расходящимися рядами}

На прошлой лекции был сформулирован и доказан следующий признак:
\begin{Test}[Интегральный признак Коши--Маклорена]

Пусть $f(x) \geqslant 0$ --- невозрастающая на $[1, \infty]$ функция. Тогда $\sum\limits_{n=1}^{\infty}f(n)$ и $\int\limits_1^{\infty}f(x)dx$ сходятся или расходятся одновременно.
\end{Test}

С помощью него мы можем исследовать на сходимость семейство рядов
$$
\series{1}{\infty} \frac{1}{n^\alpha}.
$$
Как и для соответствующего интеграла, ряд сходится тогда и только тогда, когда $\alpha > 1$.

Может сложиться впечатление, что ряд $\series{1}{\infty} \frac{1}{n}$ является своего рода граничным между сходящимися и расходящимися рядами. Но исследуем теперь другой ряд (он нам также понадобится в дальнейшем):
$$
\sum\limits_{n=2}^{\infty} \frac{1}{n\ln n}.
$$
Он сходится тогда и только тогда, когда сходится соответствующий интеграл.
$$
\int\limits_{2}^{\infty} \frac{1}{x\ln x}dx = \int\limits_{2}^{\infty} \frac{1}{\ln x} d\ln x = \ln\ln x \Big|_2^{\infty} = \infty
$$
Данный ряд меньше, чем гармоничный ряд, однако расходится. Причем, как несложно убедиться, семейство рядов $\sum\limits_{n=2}^{\infty}\frac{1}{n\ln^\beta n}$ при $\beta > 1$ уже сходится. Но при этом <<граница>> между сходящимися и расходящимися рядами не проходит по ряду $\sum\limits_{n=2}^{\infty} \frac{1}{n\ln n}$ --- взять, например, ряд $\sum\limits_{n=2}^{\infty}\frac{1}{n\ln n \ln\ln n}$, который тоже расходится. И так далее, <<границу>> можно <<уточнять>> бесконечно. Так что точной <<границы>> не существует.

\subsection{Скорость роста частичных сумм расходящихся рядов}

В прошлой лекции мы с помощью интегрального признака Коши--Маклорена научились оценивать остаток сходящихся сумм. Теперь научимся оценивать скорость роста частичных сумм расходящихся рядов.

Возьмем, например, гармонический ряд. Утверждается, что его частичные суммы оцениваются следующим образом:
$$
\series{1}{N} \frac{1}{n} = \ln N + C + o(1),
$$
где $C$ эта некая константа. Но как доказать, что это действительно корректная оценка? 

Фактически мы утверждаем сходимость последовательности $\{S_n\}$, где
$$
S_n = \frac{1}{1} + \frac{1}{2} + \ldots + \frac{1}{n} - \ln n.
$$

Это можно воспринимать как последовательность частичных сумм и, соответственно, перейти к соответствующему ряду:
$$
a_n = S_n - S_{n-1} = \frac{1}{n} - \ln n + \ln(n-1) = \frac{1}{n} - \ln\frac{n}{n-1} = \frac{1}{n} + \ln\left(1 - \frac{1}{n}\right) = \frac{1}{n} + \left( - \frac{1}{n} + O\left(\frac{1}{n^2}\right) \right).
$$
На последнем шаге мы воспользовались разложением в ряд Тейлора.

Мы получили, что $a_n = O\left(\frac{1}{n^2}\right)$, следовательно, данный ряд сходится. И так как мы построили сходящийся ряд, у которого последовательность $\{ S_n\}$ будет последовательностью частичных сумм, данная последовательность также сходится. Что и доказывает нашу оценку.

Точно также можно доказать оценки расходимости частичных сумм следующих рядов:
\begin{gather*}
\series{2}{N} \frac{1}{n\ln n} = \ln \ln N + C + o(1) \\
\series{1}{N} \frac{1}{\sqrt[3]{n}} = \frac{2N^{2/3}}{2} + C + o(1) 
\end{gather*}

\subsection{Снова признаки сходимости знакопостоянных рядов}
Вернемся теперь к признакам сходимости. 
\begin{Test}[Признак Кумера]
Пусть $a_n,\ b_n > 0$ и $v_n := \dfrac{a_n}{a_{n+1}}b_n - b_{n+1}$. Тогда:
\begin{enumerate}
\item если существует такое $l > 0$, что начиная с некоторого места $v_n \geq l$, то ряд $\series{1}{\infty}a_n$ сходится;
\item если начиная с некоторого места $v_n \leq 0$ и $\series{1}{\infty}\dfrac{1}{b_n}$ расходится, то и ряд $\series{1}{\infty}a_n$ расходится.
\end{enumerate}
\end{Test}

\begin{proof}
Достаточно рассмотреть случай, когда наше неравенство выполняется для всех $n$.
\begin{enumerate}
\item Итого, мы имеем, что $\dfrac{a_n}{a_{n+1}}b_n - b_{n+1} \geq l$. Домножим неравенство на $a_{n+1}$, благо оно положительно:
$$
a_n\cdot b_n - a_{n+1}\cdot b_{n+1} \geq la_{n+1} > 0
$$

Воспользуемся этим, оценив частичную сумму следующего ряда, при $N \in \N$:
$$
\series{1}{N} la_n \leq la_1 + (a_1b_1 - a_2b_2) + (a_2b_2 - a_3b_3) + \ldots + (a_{N-1}b_{N-1} - a_Nb_N)
= la_1 + a_1b_1 - a_nb_N \leq la_1 + a_1b_1
$$
Итого, мы получили, что частичные суммы ряда $\series{1}{\infty} la_n$ ограничены сверху. Значит, этот ряд сходится и, следовательно, сходится ряд $\series{1}{\infty}a_n$.

\item Имеем, что $\dfrac{a_n}{a_{n+1}}b_n - b_{n+1} \leq 0$. Перенесем $b_{n+1}$ в правую часть и разделим все на $b_n$:
$$
\frac{a_n}{a_{n+1}} \leq \frac{b_{n+1}}{b_n}
$$
Теперь перевернем дробь:
$$
\frac{a_{n+1}}{a_n} \geq \frac{b_n}{b_{n+1}} = \frac{1/ b_{n+1}}{1 / b_{n}}.
$$
По условию ряд $\series{1}{\infty}\frac{1}{b_n}$ расходится, а значит признак сравнения дает расходимость ряда $\series{1}{\infty}a_n$.
\end{enumerate}
\end{proof}

Но признак Куммера особо не используется, он скорее нужен, чтобы вывести другие признаки.
\begin{Test}[Признак Раабе]
Пусть $a_n > 0$ и существует предел 
$$
\lim\limits_{n \rarr \infty} n \left( \dfrac{a_n}{a_{n+1}} - 1 \right) = A \in [-\infty, +\infty].
$$
Тогда:
\begin{enumerate}
\item если $A > 1$, то ряд $\series{1}{\infty}a_n$ сходится;
\item если $A < 1$, то ряд $\series{1}{\infty}a_n$ расходится.
\end{enumerate}
\end{Test}

\begin{proof}
Признак Куммера при $b_n = n$.
\end{proof}

Покажем, зачем нужен признак Раабе. Пусть $a_n = \frac{1}{n^\alpha}$. Тогда:
$$
n\left( \frac{(n+1)^\alpha}{n^{\alpha}} - 1 \right) = n\left( \left( 1 + \frac{1}{n} \right)^\alpha  - 1\right) = n\left( 1 + \frac{\alpha}{n} + o\left(\frac{1}{n}\right) - 1\right) \longrightarrow \alpha.
$$
Как мы видим, признак Раабе позволяет <<ловить>> ряды с полиномиальной скоростью роста. И это хорошо, так как раньше мы этого не умели.

Но у этого признака все еще есть <<мертвая зона>>, когда $A = 1$. Поэтому рассмотрим еще один признак, который не имеет <<мертвой зоны>>, но, к сожалению, не всегда применим.

\begin{Test}[Признак Гаусса]
Пусть для некоторого $\eps > 0$ и $\alpha,\ \beta \in \R$ верно, что 
$$
\dfrac{a_n}{a_{n+1}} = \alpha + \frac{\beta}{n} + O\left( \frac{1}{n^{1+\eps}} \right).
$$
Тогда:
\begin{enumerate}
\item если $\alpha > 1$, то ряд $\series{1}{\infty}a_n$ сходится;
\item если $\alpha < 1$, то ряд $\series{1}{\infty}a_n$ расходится;
\item если $\alpha=1$ и $\beta>1$, то ряд $\series{1}{\infty}a_n$ сходится;
\item если $\alpha=1$ и $\beta\leq1$, то ряд $\series{1}{\infty}a_n$ расходится.
\end{enumerate}
\end{Test}

\begin{proof}
Все эти утверждения на самом деле следуют из уже рассмотренных нами признаков. Так что просто назовем их.
\begin{enumerate}
\item Признак д'Аламбера.
\item Признак д'Аламбера.
\item Признак Раабе.
\item Если $\beta<1$ --- признак Раабе. Если $\beta = 1$ --- признак Куммера при $b_n = n\ln n$.
\end{enumerate}
Рассмотрим подробнее последний случай, когда $\alpha=\beta=1$. Воспользуемся признаком Куммера при $b_n = n\ln n$ и равенством из условия:
\begin{gather*}
v_n = \frac{a_n}{a_{n+1}}b_n - b_{n+1} = \left( 1 + \frac{1}{n} + O\left( \frac{1}{n^{1+\eps}}\right) \right)n\ln n - (n+1)\ln(n+1) = \\
= (n+1)\left( \ln n - \ln(n+1) \right) + O\left( \frac{\ln n}{n^\eps} \right) = \\
= -(n+1)\ln\left( 1 + \frac{1}{n} \right) + O\left( \frac{\ln n}{n^\eps} \right) = \\
= -(n+1)\left( \frac{1}{n} + o\left(\frac{1}{n}\right) \right) + O\left( \frac{\ln n}{n^\eps} \right) \longrightarrow -1
\end{gather*}
Итого, по признаку Куммера ряд действительно расходится.
\end{proof}

\begin{Comment}
Вместо $O\left(\dfrac{1}{n^{1+\eps}} \right)$ можно писать более сильное $O\left(
\dfrac{1}{n\ln n} \right)$. Но первое чаще появляется в интересных примерах, поэтому исторически сложилось использовать его.
\end{Comment}

\subsection{Признаки сходимости знакопеременных рядов}
\begin{Test}[Признак Лейбница]
Пусть последовательность $\{b_n\}$ строго монотонно убывает у нулю. Тогда ряд $\series{0}{\infty}(-1)^nb_n$ сходится, причем его остаток $r_N$ имеет знак $(-1)^{N+1}$ и по модулю меньше $b_{N+1}$.
\end{Test}
\begin{proof}
Докажем с помощью критерия Коши. Зафиксируем произвольное $\eps > 0$ и найдем такое $N \in \N$, что для всех $n > N$ верно, что $b_n < \eps$. Теперь для любого $m > N$ и $p \in \N$ рассмотрим следующую величину:
$$
\left| \series{m+1}{m+p} (-1)^n b_n \right|.
$$  
Можно вынести $(-1)^{m+1}$ из суммы --- на модуль это не повлияет, но зато нам будет удобнее считать, что первое слагаемое идет с положительным знаком.

Сгруппируем слагаемые следующим образом:
$$
\left| \series{m+1}{m+p} (-1)^n b_n \right| = \big|b_{m+1} + (- b_{m+2} + b_{m+3}) + (- b_{m+4} + b_{m+5}) + \ldots\big|.
$$

В силу строго монотонного убывания последовательности получаем, что каждая скобка меньше нуля. Последнее слагаемое, $b_{m+p}$ могло остаться без пары, но тогда оно идет с отрицательным знаком. Итого, получаем, что мы с $b_{m+1}$ складываем только отрицательные величины, следовательно:
$$
\left| \series{m+1}{m+p} (-1)^n b_n \right| \leq |b_{m+1}| < \eps.
$$

Итого, по критерию Коши ряд сходится. Отсюда же следует оценка на остаток: 
$$
|r_N| = \left| \series{N+1}{\infty}(-1)^n b_n  \right| \leq b_{N+1}.
$$
Аналогичным образом оценим остаток знака.

Снова вынесем за скобки знак $(-1)^{N+1}$ (но на этот раз его не убъет модуль), и сгруппируем слагаемые:
$$
r_N = \series{N+1}{\infty} (-1)^n b_n  = (-1)^{N+1}\big((b_{N+1} - b_{N+2}) + (b_{N+3} - b_{N+4}) + \ldots\big).$$
Каждая группа слагаемых больше нуля в силу строго монотонного убывания последовательности. Следовательно, вся скобка имеет положительный знак, а значит, $r_N$ имеет знак $(-1)^{N+1}$. Что нам и требовалось.
\end{proof}

\end{document}