\documentclass[a4paper, 12pt]{article}
\usepackage{header}

\begin{document}
\pagestyle{fancy}
\section{Лекция 06 от 10.10.2016 \\ Бесконечные произведения}
\subsection{Основные понятия и определения}
Итак, бесконечные произведения. Попытаемся применить к ним тот же подход, что и к рядам.

\begin{Def}
Пусть $\{a_k\}_{k=1}^\infty$ --- последовательность ненулевых действительных чисел. Бесконечным произведением называется выражение $a_1 a_2 \ldots a_n \ldots$, записываемое также как $\prod\limits_{n=1}^{\infty}a_n$. Частичным произведением называется величина $P_N = a_1 \ldots a_N$.
\end{Def}

\begin{Def}
Бесконечное произведение $\infprod a_n$ сходится к числу $A \neq 0$, если последовательность частичных произведений $P_N$ сходится к $A$ при $N \to \infty$.
\end{Def}

\begin{Def}
Бесконечное произведение сходится, если существует такое $A \neq 0$, к которому это произведение сходится.
\end{Def}

\begin{Statement}
Добавление/удаление/изменение конечного числа множителей не влияет на сходимость/расходимость бесконечного произведения.
\end{Statement}

Здесь важно понимать, что это возможно только потому, что мы запретили последовательности содержать нулевые элементы.

\begin{Statement}[Необходимое условие сходимости]
Если $\infprod a_n$ сходится, то $a_n \to 1$.
\end{Statement}

\begin{proof}
Пусть $\infprod a_n = A$. Тогда $a_n = \dfrac{P_n}{P_{n-1}} \to \dfrac{A}{A} = 1$.
\end{proof}

Здесь становится видно, почему $A=0$ --- это плохо. Именно поэтому мы запретили сходимость к нулю, когда давали соответствующие определения.

Раз сходиться к нулю нельзя, то определим это несколько иначе.
\begin{Def}
Бесконечное произведение $\infprod a_n$ расходится к нулю, если $P_N \to 0$ при $N \to~\infty$.

Бесконечное произведение $\infprod a_n$ расходится к $+\infty$, если $P_N \to +\infty$ при $N \to \infty$.
\end{Def}

\subsection{Связь с числовыми рядами, исследование сходимости}

Конечно, можно было бы потратить несколько лекций на то, чтобы заново доказать все те признаки, которые мы уже разобрали для рядов. Но гораздо легче просто свести задачу к предыдущей.

При изучении сходимости бесконечных произведений достаточно ограничиться случаем, когда $a_n \to 1$. Тогда можно считать, что начиная с некоторого места все члены последовательности строго положительны. А так как удаление конечного числа начальных членов на факт сходимости или расходимости не влияет, достаточно изучить бесконечные произведения только с положительными членами.

\begin{Statement}
Бесконечное произведение $\infprod a_n$ сходится тогда и только тогда, когда сходится ряд $\sum\limits_{n=1}^{\infty}\ln a_n$.
\end{Statement}
\begin{proof}
Заметим, что $S_N = \ln P_N$. Тогда, если существует предел $\lim\limits_{n \to \infty} P_n = A > 0$, то существует и предел $\lim\limits_{n\to \infty} S_n = \ln A$, то есть ряд $\series{1}{\infty}\ln a_n$ сходится.

И наоборот, если $\series{1}{\infty}\ln a_n = S$, то есть $S_N \to S$, то тогда $P_N = e^{S_n} \to e^S \neq 0$, то есть бесконечное произведение $\infprod a_n$ сходится. 
\end{proof}

Отсюда становится понятным, почему логично определять стремление бесконечного произведения $\infprod a_n$ к нулю как расходимость --- это соответствует случаю, когда ряд $\series{1}{\infty}\ln a_n$ расходится к $-\infty$.

Также получаем несколько халявных следствий.
\begin{Statement}
Пусть все $\alpha_n \geq 0$ или все $\alpha_n \in (-1, 0]$. Тогда бесконечное произведение $\infprod(1 + \alpha_n)$ сходится тогда и только тогда, когда сходится ряд $\series{1}{\infty}\alpha_n$.
\end{Statement}
\begin{proof}
Если $\alpha_n$ не стремится к нулю, то и $1 + \alpha_n$ не стремится к единице. Тогда и ряд, и бесконечное произведение расходятся, так как не выполняется необходимое условие сходимости.

Теперь пусть $\alpha_n \to 0$. Тогда сходимость бесконечного произведения $\infprod (1 + \alpha_n)$ равносильна сходимости ряда $\series{1}{\infty}\ln(1 + \alpha_n)$. И так как он знакопостоянный, то по соответствующему признаку сравнения этот ряд сходится тогда и только тогда, когда сходится ряд из эквивалентных членов $\series{1}{\infty}\alpha_n$.
\end{proof}

\begin{Statement}
Пусть $\alpha_n > -1$ и ряд $\series{1}{\infty} \alpha_n$ сходится. Тогда бесконечное произведение $\infprod (1 + \alpha_n)$ сходится тогда и только тогда, когда сходится ряд $\series{1}{\infty}\alpha_n^2$.
\end{Statement}
\begin{proof}
Так как ряд $\series{1}{\infty}\alpha_n$ сходится, то $\alpha_n \to 0$. Аналогично предыдущему доказательству, достаточно исследовать сходимость ряда $\series{1}{\infty}\ln(1 + \alpha_n)$.

Разложим его члены по формуле Тейлора, получив $\series{1}{\infty}\left( \alpha_n - \dfrac{\alpha_n^2}{2} + o(\alpha_n^2)  \right)$, а это уже равносильно сходимости ряда $\series{1}{\infty}\alpha_n^2(1 + o(1))$. Начиная с некоторого номера, ряд станет знакопостоянным, то есть можно применить все тот же признак сравнения. Что и приводит нас с исследованию сходимости ряда $\series{1}{\infty}\alpha_n^2$.
\end{proof}

Фактически мы доказали два необходимых и достаточных условия сходимости. Теперь рассмотрим просто достаточное.

\begin{Statement}
Пусть $\alpha_n > -1$. Тогда если ряд $\series{1}{\infty}|\alpha_n|$ сходится, то сходится и бесконечное произведение $\infprod (1 + \alpha_n)$.
\end{Statement}
\begin{proof}
$$
\series{1}{\infty}|\alpha_n| \text{ сходится} \Rightarrow \series{1}{\infty}|
\ln(1 + \alpha_n)| \text{ сходится} \Rightarrow \series{1}{\infty}\ln(1 + \alpha_n) \text{ сходится} \Rightarrow \infprod (1 + \alpha_n) \text{ сходится.}
$$
\end{proof}

Если ввести соответствующее определение, то на бесконечные произведения можно будет распространить теорему о перестановке множителей (слагаемых) и ее влиянии на сходимость.

\begin{Def}
Бесконечное произведение $\infprod a_n$ сходится абсолютно/условно, если абсолютно/условно сходится ряд $\series{1}{\infty}\ln a_n$.
\end{Def}

\subsection{Применение}
Теперь, окончательно убедившись, что изучение бесконечных произведений можно свести к изучению рядов, самое время задаться вопросом --- а зачем они нужны?

Оказывается, они могут быть удобным инструментом при доказательствах. Приведем несколько примеров.
\begin{Statement}
Пусть $a_n > 0$, $\series{1}{\infty}a_n = +\infty$ и $S_n = a_1 + \ldots a_n$. Тогда ряд $\series{1}{\infty}\dfrac{a_n}{S_n}$ расходится.
\end{Statement}
\begin{proof}
Достаточно доказать расходимость бесконечного произведения $\prod\limits_{n=2}^{\infty} \left( 1 - \dfrac{a_n}{S_n}\right)$:
\begin{gather*}
\prod\limits_{n=2}^{\infty}\left(1 - \dfrac{a_n}{S_n} \right) = \prod\limits_{n=2}^{\infty} \dfrac{S_n - a_n}{S_n} = \prod\limits_{n=2}^{\infty} \dfrac{S_{n-1}}{S_n}, \\
P_n = \dfrac{S_1}{S_2}\cdot \dfrac{S_2}{S_3} \ldots \dfrac{S_{n-1}}{S_n} = \frac{S_1}{S_n} \to 0.
\end{gather*}
\end{proof}

Отсюда в частности следует, что нет самого маленького расходящегося ряда.


Теперь докажем почти формулу Стирлинга.
\begin{Statement}
Пусть $a_n = \dfrac{n!e^n}{n^{n + 1/2}}$. Тогда существует предел $\lim\limits_{n \to \infty}a_n = A > 0$.
\end{Statement}
\begin{proof}
Представим элемент $a_n$ в следующем виде:
$$
a_n = a_1 \cdot \frac{a_2}{a_1}\cdot\frac{a_3}{a_2} \ldots \frac{a_n}{a_{n-1}} = a_1 / \prod\limits_{k=1}^{n-1} \frac{a_k}{a_{k+1}}.
$$

Тогда сходимость к положительной константе последовательности $\{a_n\}$ равносильна сходимости бесконечного произведения $\infprod \dfrac{a_n}{a_{n+1}}$.

Посчитаем, чему равен член этого произведения:
$$
\frac{a_n}{a_{n+1}} = \frac{n!e^n (n+1)^{n+1+1/2}}{(n+1)! e^{n+1}n^{n+1/2} } = \left(1 + \frac{1}{n} \right)^{n + 1/2} / e.
$$

Перейдем к рассмотрению ряда из логарифмов:
\begin{gather*}
\ln \frac{a_n}{a_{n+1}} = \left( n + \frac{1}{2}\right)\ln\left( 1 + \frac{1}{n}\right) - 1 = [\text{ф-ла Тейлора}] = \\ =\left(n + \frac{1}{2}\right)\left( \frac{1}{n} - \frac{1}{2n^2} + O\left(\frac{1}{n^3} \right) \right) - 1 = 1 - \frac{1}{2n} + \frac{1}{2n} - 1 + O\left(\frac{1}{n^2} \right) = O\left(\frac{1}{n^2} \right).
\end{gather*}


Получили, что такой ряд будет сходиться. Следовательно, существует положительный предел $\lim\limits_{n \to \infty}a_n$.

\end{proof}


\end{document}
