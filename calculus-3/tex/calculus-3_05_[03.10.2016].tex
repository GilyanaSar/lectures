\documentclass[a4paper, 12pt]{article}
\usepackage{header}
\begin{document}
\pagestyle{fancy}
\section{Лекция 05 от 03.10.2016 \\ Перестановки рядов и произведения рядов}
\subsection{Основные теоремы о перестановках рядов}
Напомним основное для этой лекции определение.
	\begin{Def}
		Пусть $\sigma$ --- биекция (перестановка) $\N \to \N$. Тогда говорят, что ряд $\sum\limits_{n=1}^{\infty}a_{\sigma(n)}$ является перестановкой ряда $\sum\limits_{n=1}^{\infty}a_n$.
	\end{Def}

	\begin{Theorem}[Коши]
		Пусть ряд $\sum\limits_{n=1}^{\infty}a_n$ абсолютно сходится, и его сумма равна $A$. Тогда любая его перестановка $\sum\limits_{n=1}^{\infty}a_{\sigma(n)}$ также сходится абсолютно, и её сумма равна $A$.
	\end{Theorem}
	\begin{proof} [Доказательство абсолютной сходимости:]  
		Докажем, что $\sum\limits_{n=1}^{\infty}a_{\sigma(n)}$ абсолютно сходится.
		Обозначим $A_+ := \sum\limits_{n=1}^{\infty}|a_n|$.
		Возьмём произвольное $N\in \mathbb{N}$ и покажем, что  $\sum\limits_{n=1}^{N}|a_{\sigma(n)}| \leq A_+$ (тогда возрастающая последовательность частичных сумм $\sum\limits_{n=1}^{\infty}|a_{\sigma(n)}|$ ограничена и ряд сходится).

		Определим $M := \max \left\{\sigma(1), \sigma(2), \dots, \sigma(N)\right\}$.  Тогда очевидно, что $\sum\limits_{n=1}^{N}|a_{\sigma(n)}| \leq \sum\limits_{n=1}^{M}|a_{n}|$, так как правая сумма содержит в себе и все слагаемые левой суммы. Но из этого неизбежно следует и  $\sum\limits_{n=1}^{N}|a_{\sigma(n)}| \leq A_+$, потому что любая частичная сумма $\sum\limits_{n=1}^{M}|a_{n}|$ ряда с неотрицательными слагаемыми не больше всей его суммы.
	\end{proof}
	\begin{proof} [Доказательство сходимости к тому же значению:]		
		Докажем, что $\sum\limits_{n=1}^{\infty}a_{\sigma(n)}$ сходится к $A$. Пусть есть некоторое $\eps>0$. Возьмём такое $N\in \mathbb{N}$, что $\sum\limits_{n=N}^{\infty}|a_{n}| < \frac{\eps}{2}$. (Тогда $\left| \sum\limits_{n=1}^{N}a_{n} - A\right| = \left| \sum\limits_{N+1}^{\infty}a_{n}\right| \leq \sum\limits_{N+1}^{\infty}|a_{n}| < \frac{\eps}{2}$.) 
		
		Обозначим $M := \max\{ \sigma^{-1}(1), \sigma^{-1}(2)\dots \sigma^{-1}(N)\}$.
		 
		Тогда для любого $\tilde{M}>M$:
		\begin{multline}
		 \left| \sum\limits_{m=1}^{\tilde{M}}a_{\sigma(m)} - A \right|\leq \left| \sum\limits_{m=1}^{\tilde{M}}a_{\sigma(m)} - \sum\limits_{n=1}^{N}a_{n} \right| + \left| \sum\limits_{n=1}^{N}a_{n} - A \right| < \\ < \left| \sum\limits_{\substack{m=1\dots \tilde{M}\\ \sigma(m)>N} }^{\tilde{M}}a_{\sigma(m)}\right| + \frac{\eps}{2} \leq   \sum\limits_{\substack {m=1\dots \tilde{M} \\ \sigma(m)>N} }^{\tilde{M}}|a_{\sigma(m)}| + \frac{\eps}{2} \leq \\ \leq \sum\limits_{n=N+1}^{\max\{ \sigma(1)\dots \sigma(N)\}}|a_n| + \frac{\eps}{2} \leq \sum\limits_{n=N+1}^{\infty}a_{|a_n|} +  \frac{\eps}{2} < \eps
		 \end{multline}	
	\end{proof}
	
	
	Теперь пусть $\sum\limits_{n=1}^{\infty}a_n$ сходится условно. В нём бесконечно много положительных слагаемых и бесконечно много отрицательных, так как иначе он сходился бы абсолютно. Через $\{p_n\}$ обозначим последовательность всех неотрицательных слагаемых, а через $\{q_n\}$, соответственно, отрицательных. 
	
	Раз $\sum\limits_{n=1}^{\infty}a_n$ сходится, то $\{a_n\}$ --- сходящаяся к нулю последовательность, а значит и $\{p_n\}$ и $\{q_n\}$ тоже сходятся.
	При этом несложно понять, что $\sum\limits_{n=1}^{\infty}p_n$ и $\sum\limits_{n=1}^{\infty}q_n$ --- расходятся. Так как если бы оба этих ряда сходились, то $\sum\limits_{n=1}^{\infty}a_n$ сходился бы абсолютно, а если бы один из них сходился, а другой --- расходился, то  $\sum\limits_{n=1}^{\infty}a_n$ бы расходился.
	
	\begin{Theorem}[Римана]
		Пусть ряд $\sum\limits_{n=1}^{\infty}a_n$ сходится условно. Тогда:
		\begin{enumerate}
			\item для любого $A \in \mathbb{R}$ найдётся такая перестановка $\sigma$, что $\sum\limits_{n=1}^{\infty}a_{\sigma(n)} = A $;
			\item существует такая перестановка $\sigma$, что ряд $\sum\limits_{n=1}^{\infty}a_{\sigma(n)}$ расходится к $+\infty$;
			\item существует такая перестановка $\sigma$, что ряд $\sum\limits_{n=1}^{\infty}a_{\sigma(n)}$ расходится к $-\infty$;
			\item существует такая перестановка $\sigma$, что для ряда $\sum\limits_{n=1}^{\infty}a_{\sigma(n)}$ последовательность частичных сумм не имеет ни конечного ни бесконечного предела.
		\end{enumerate}
	\end{Theorem}
	\begin{proof} \ 
		\begin{enumerate}
			\item Возьмём произвольное $A \in \mathbb{R}$.  
			
			Найдём наименьшее $k_1 \in \mathbb{N}$ такое что $p_1+p_2+\dots+p_{k_1} > A$. 
			
			Найдём наименьшее $\tilde{k}_1 \in \mathbb{N}$ такое что $p_1+p_2+\dots+p_{k_1} + q_1+q_2 +\dots +q_{\tilde{k}_1} < A$ 
			
			Найдём наименьшее $k_2 \in \mathbb{N}$ такое что $p_1+p_2+\dots+p_{k_1} + q_1+q_2 +\dots +q_{\tilde{k}_1}+ p_{k_1+1}+\dots+p_{k_2} > A$ 
			И так далее. В силу того, что $\{p_n\}$ и $\{q_n\}$ сходятся к нулю, построение выше и даст перестановку ряда, сумма которой равна $A$.
			
			В остальных пунктах всё вполне аналогично.
			\item 
			Найдём наименьшее $k_1 \in \mathbb{N}$ такое что $p_1+p_2+\dots+p_{k_1} > 1$. 
			
			Найдём наименьшее $k_2 \in \mathbb{N}$ такое что $p_1+p_2+\dots+p_{k_1} + q_1+ p_{k_1+1}+\dots+p_{k_2} > 2$ 

			Найдём наименьшее $k_3 \in \mathbb{N}$ такое что $p_1+p_2+\dots+p_{k_1} + q_1+ p_{k_1+1}+\dots+p_{k_2} + q_2
			+ p_{k_2+1}+\dots+p_{k_3} > 3$ 

			И так далее. Построение выше и даст перестановку ряда, расходящуюся к к $+\infty$.
			\item Аналогично предыдущему.
			\item Аналогично предыдущим, например, доводя сумму последовательно до 1, -1, 2, -2, 3, -3 и так далее.
		\end{enumerate}
	\end{proof}
\newpage
\subsection{Произведение числовых рядов}
	Произведение пары конечных сумм записывается вполне естественным и понятным образом:
	\[
	\sum\limits_{n=1}^{N}a_n \cdot \sum\limits_{m=1}^{M}b_m = (a_1 + \dots + a_n) (b_1 + \dots + b_m)=\sum\limits_{n=1, \ m=1}^{N,\  M}a_nb_m
	\]
	C бесконечными суммами всё менее понятно. Казалось бы,
	\[
	\sum\limits_{n=1}^{\infty}a_n \cdot \sum\limits_{m=1}^{\infty}b_m = \sum\limits_{n=1 \ m=1}^{\infty}a_nb_m,
	\] однако объект в правой части равенства мы не определяли.

	\medskip
	
	\begin{wrapfigure}{l}{0.3\linewidth}
		\begin{tabular}[t]{c|ccccc}
			$\vdots$ & $\vdots$ & $\vdots$ & $\vdots$ &$\vdots$& $\iddots$ \\
			$b_4$ & $16$ & $15$ & $14$ & $13$ &$\dots$\\
			$b_3$ & 9 & 8 & 7 &  12&$\dots$ \\
			$b_2$ & $4$ & $3$ & 6 & 11&$\dots$ \\
			$b_1$ & $1$ & $2$ & 5 & 10& $\dots $\\
			\hline
			& $a_1$ & $a_2$ & $a_3$ & $a_4$ &$\dots$ 
		\end{tabular}
		\caption{Нумерация по квадратам}
		\vspace{-50pt}
	\end{wrapfigure}
	Но по крайней мере множество пар индексов $(n,\ m)$ счетно, а значит и множество слагаемых в сумме счётно, то есть его можно занумеровать и таким образом превратить произведение рядов в обычный ряд. Вопрос лишь в том, как именно это сделать.
	
	\begin{Theorem}[Коши о произведении абсолютно сходящихся рядов]
		Пусть $\sum\limits_{n=1}^{\infty}a_n =A $ и$ \sum\limits_{m=1}^{\infty}b_m = B$, причём оба ряда абсолютно сходятся. Тогда ряд из произведений $a_nb_m$, занумерованных в любом порядке, сходится абсолютно и его сумма равна $A\cdot B$.
	\end{Theorem} 
	
\smallskip

	\begin{proof} По недавно доказанной теореме Коши о перестановках абсолютно сходящегося ряда нам достаточно доказать, что хотя бы при какой-то одной нумерации ряд из произведений абсолютно сходится к $A\cdot B$.	
		
		Будем использовать довольно очевидный способ нумерации, вполне достаточно описываемый картинкой слева, обычно называемый <<нумерация по квадратам>>. 	Обозначим $A_+ := \sum\limits_{n=1}^{\infty}|a_n|$,  $B_+ := \sum\limits_{m=1}^{\infty}|b_m|$, и $\sum\limits_{k=1}^{\infty}c_k$ --- ряд из произведений, занумерованный выбранным нами способом. Тогда последовательность частичных сумм ряда из модулей $c_k$ ограничена \\
		\[
		\sum\limits_{k=1}^{K}|c_k| \leq \sum\limits_{k=1}^{K^2}|c_k| = \left(\sum\limits_{n=1}^{K}|a_n|\right)\left(\sum\limits_{m=1}^{K}|b_m|\right) \leq A_+\cdot B_+,
		\] 
		то есть ряд $\sum\limits_{k=1}^{\infty}c_k$ сходится абсолютно. 
		Сумму этого ряда посчитать теперь совсем несложно:
		\[\sum\limits_{k=1}^{\infty}c_k = \lim\limits_{K\rarr \infty} \sum\limits_{k=1}^{K}c_k = \lim\limits_{K\rarr \infty} \sum\limits_{k=1}^{K^2}c_k = \lim\limits_{K\rarr \infty} \left(\sum\limits_{n=1}^{K}a_n\right)\left(\sum\limits_{m=1}^{K}b_m\right)=A\cdot B
		\]
		\end{proof}
		
		\newpage
	\begin{wrapfigure}{r}{0.3\linewidth}
		\begin{tabular}[t]{c|ccccc}
			$\vdots$ &15& $\iddots$& & & \\
			$b_4$ & 10 & 14&$\iddots$ &&  \\
			$b_3$ & 6 & 9 & 13&$\iddots$ &   \\
			$b_2$ & 3 & 5& 8 & 12&$\iddots$ \\
			$b_1$ & $1$ & $2$ &4&7&11\\
			\hline
			& $a_1$ & $a_2$ & $a_3$ & $a_4$ &$\dots$ \\
		\end{tabular}
		\caption{Нумерация по треугольникам}
		\vspace{-60pt}
	\end{wrapfigure}
	Если хоть один из рядов не сходится абсолютно, такое утверждение уже неверно. Так что для всех остальных случаев важно договориться о нумерации. Один из часто встречающихся удобных способов нумерации, который в дальнейшем будет подразумеваться по умолчанию --- это так называемая <<нумерация по треугольникам>> или <<произведение Коши>>.
	\begin{Def}
		Для рядов $\sum\limits_{n=1}^{\infty}a_n$ и$ \sum\limits_{m=1}^{\infty}b_m$ их произведением называется ряд  $\sum\limits_{k=1}^{\infty}c_k$, где $c_k =\sum\limits_{j=0}^{k}a_kb_{k-j}$
	\end{Def}
	
	\begin{Theorem}[Мертенса]
		Пусть ряд $\sum\limits_{n=1}^{\infty}a_n = A $ и$ \sum\limits_{m=1}^{\infty}b_m = B$, причём хотя бы один из рядов сходится абсолютно. Тогда $\sum\limits_{k=1}^{\infty}\sum\limits_{j=0}^{k}a_kb_{k-j} = AB.$
	\end{Theorem}
	
	Доказательство этой теоремы опустим.
\end{document}
