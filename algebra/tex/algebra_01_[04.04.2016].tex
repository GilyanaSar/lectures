\documentclass[a4paper,10pt]{amsart}

\usepackage[T2A]{fontenc}
\usepackage[utf8x]{inputenc}
\usepackage{amssymb}
\usepackage[russian]{babel}
\usepackage{geometry}

\geometry{a4paper,top=2cm,bottom=2cm,left=2cm,right=2cm}

\setlength{\parindent}{0pt}
\setlength{\parskip}{\medskipamount}

\def\Ker{{\rm Ker}}%
\def\Im{{\rm Im}}%
\def\Mat{{\rm Mat}}%
\def\cont{{\rm cont}}%
\def\Tor{{\rm Tor}}%
\def\Char{{\rm Char}}%
\def\signum{{\rm sig}}%
\def\Sym{{\rm Sym}}%
\def\St{{\rm St}}%
\def\Aut{{\rm Aut}}%
\def\Chi{{\mathbb X}}%
\def\Tau{{\rm T}}%
\def\Rho{{\rm R}}%
\def\rk{{\rm rk}}%
\def\ggT{{\rm ggT}}%
\def\kgV{{\rm kgV}}%
\def\Div{{\rm Div}}%
\def\div{{\rm div}}%
\def\quot{/\!\!/}%
\def\mal{\! \cdot \!}%
\def\Of{{\mathcal{O}}}
%
\def\subgrpneq{\le}%
\def\subgrp{\le}%
\def\ideal#1{\le_{#1}}%
\def\submod#1{\le_{#1}}%
%
\def\Bild{{\rm Bild}}%
\def\Kern{{\rm Kern}}%
\def\bangle#1{{\langle #1 \rangle}}%
\def\rq#1{\widehat{#1}}%
\def\t#1{\widetilde{#1}}%
\def\b#1{\overline{#1}}%
%
\def\abs#1{{\vert #1 \vert}}%
\def\norm#1#2{{\Vert #1 \Vert}_{#2}}%
\def\PS#1#2{{\sum_{\nu=0}^{\infty} #1_{\nu} #2^{\nu}}}%
%
\def\C{{\rm C}}%
\def\O{{\rm O}}%
\def\HH{{\mathbb H}}%
\def\LL{{\mathbb L}}%
\def\FF{{\mathbb F}}%
\def\CC{{\mathbb C}}%
\def\KK{{\mathbb K}}%
\def\TT{{\mathbb T}}%
\def\ZZ{{\mathbb Z}}%
\def\RR{{\mathbb R}}%
\def\SS{{\mathbb S}}%
\def\NN{{\mathbb N}}%
\def\QQ{{\mathbb Q}}%
\def\PP{{\mathbb P}}%
\def\AA{{\mathbb A}}%
%
\def\eins{{\mathbf 1}}%
%
\def\AG{{\rm AG}}%
\def\Aut{{\rm Aut}}%
\def\Hol{{\rm Hol}}%
\def\GL{{\rm GL}}%
\def\SL{{\rm SL}}%
\def\SO{{\rm SO}}%
\def\Sp{{\rm Sp}}%
\def\gl{\mathfrak{gl}}%
\def\rg{{\rm rg}}%
\def\sl{\mathfrak{sl}}%
\def\HDiv{{\rm HDiv}}%
\def\CDiv{{\rm CDiv}}%
\def\Res{{\rm Res}}%
\def\Pst{{\rm Pst}}%
\def\Nst{{\rm Nst}}%
\def\rad{{\rm rad}}%
\def\GL{{\rm GL}}%
\def\Tr{{\rm Tr}}%
\def\Pic{{\rm Pic}}%
\def\Hom{{\rm Hom}}%
\def\hom{{\rm hom}}%
\def\Mor{{\rm Mor}}%
\def\codim{{\rm codim}}%
\def\Supp{{\rm Supp}}%
\def\Spec{{\rm Spec}}%
\def\Proj{{\rm Proj}}%
\def\Maps{{\rm Maps}}%
\def\cone{{\rm cone}}%
\def\ord{{\rm ord}}%
\def\pr{{\rm pr}}%
\def\id{{\rm id}}%
\def\mult{{\rm mult}}%
\def\inv{{\rm inv}}%
\def\neut{{\rm neut}}%
%
\def\AAA{\mathcal{A}}
\def\BBB{\mathcal{B}}
\def\CCC{\mathcal{C}}
\def\EEE{\mathcal{E}}
\def\FFF{\mathcal{F}}

\def\CF{{\rm CF}}
\def\GCD{{\rm GCD}}
\def\Mat{{\rm Mat}}
\def\End{{\rm End}}
\def\cont{{\rm cont}}
\def\Kegel{{\rm Kegel}}
\def\Char{{\rm Char}}
\def\Der{{\rm Der}}
\def\signum{{\rm sg}}
\def\grad{{\rm grad}}
\def\Spur{{\rm Spur}}
\def\Sym{{\rm Sym}}
\def\Alt{{\rm Alt}}
\def\Abb{{\rm Abb}}
\def\Chi{{\mathbb X}}
\def\Tau{{\rm T}}
\def\Rho{{\rm R}}
\def\ad{{\rm ad}}
\def\Frob{{\rm Frob}}
\def\Rang{{\rm Rang}}
\def\SpRang{{\rm SpRang}}
\def\ZRang{{\rm ZRang}}
\def\ggT{{\rm ggT}}
\def\kgV{{\rm kgV}}
\def\Div{{\rm Div}}
\def\div{{\rm div}}
\def\quot{/\!\!/}
\def\mal{\! \cdot \!}
\def\add{{\rm add}}
\def\mult{{\rm mult}}
\def\smult{{\rm smult}}

\def\subgrpneq{\le}
\def\subgrp{\le}
\def\ideal#1{\unlhd_{#1}}
\def\submod#1{\le_{#1}}

\def\Bild{{\rm Bild}}
\def\Kern{{\rm Kern}}
\def\Kon{{\rm Kon}}
\def\bangle#1{{\langle #1 \rangle}}
\def\rq#1{\widehat{#1}}
\def\t#1{\widetilde{#1}}
\def\b#1{\overline{#1}}

\def\abs#1{{\vert #1 \vert}}
\def\norm#1#2{{\Vert #1 \Vert}_{#2}}
\def\PS#1#2{{\sum_{\nu=0}^{\infty} #1_{\nu} #2^{\nu}}}


\def\eins{{\mathbf 1}}

\def\ElM{{\rm ElM}}
\def\ZOp{{\rm ZOp}}
\def\SpOp{{\rm SpOp}}
\def\Gal{{\rm Gal}}
\def\Def{{\rm Def}}
\def\Fix{{\rm Fix}}
\def\ord{{\rm ord}}
\def\Aut{{\rm Aut}}
\def\Hol{{\rm Hol}}
\def\GL{{\rm GL}}
\def\SL{{\rm SL}}
\def\SO{{\rm SO}}
\def\Sp{{\rm Sp}}
\def\Spann{{\rm Spann}}
\def\Lin{{\rm Lin}}
\def\gl{\mathfrak{gl}}
\def\rg{{\rm rg}}
\def\sl{\mathfrak{sl}}
\def\so{\mathfrak{so}}
\def\sp{\mathfrak{sp}}
\def\gg{\mathfrak{g}}
\def\HDiv{{\rm HDiv}}
\def\CDiv{{\rm CDiv}}
\def\Res{{\rm Res}}
\def\Pst{{\rm Pst}}
\def\Nst{{\rm Nst}}
\def\WDiv{{\rm WDiv}}
\def\GL{{\rm GL}}
\def\Tr{{\rm Tr}}
\def\Pic{{\rm Pic}}
\def\Hom{{\rm Hom}}
\def\hom{{\rm hom}}
\def\Mor{{\rm Mor}}
\def\codim{{\rm codim}}
\def\Supp{{\rm Supp}}
\def\Spec{{\rm Spec}}
\def\Proj{{\rm Proj}}
\def\Maps{{\rm Maps}}
\def\cone{{\rm cone}}
\def\ord{{\rm ord}}
\def\pr{{\rm pr}}
\def\id{{\rm id}}
\def\mult{{\rm mult}}
\def\inv{{\rm inv}}
\def\neut{{\rm neut}}
\def\trdeg{{\rm trdeg}}
\def\sing{{\rm sing}}
\def\reg{{\rm reg}}


%%%%%%%%%%%%%%%%%%%%%%%%%%%

\newtheorem{theorem}{Теорема}
\newtheorem{proposition}{Предложение}
\newtheorem{lemma}{Лемма}
\newtheorem{corollary}{Следствие}
\theoremstyle{definition}
\newtheorem{definition}{Определение}
\newtheorem{problem}{Задача}
%
\theoremstyle{remark}
\newtheorem{exc}{Упражнение}
\newtheorem{remark}{Замечание}
\newtheorem{example}{Пример}

\renewcommand{\theenumi}{\textup{(\alph{enumi})}}
\renewcommand{\labelenumi}{\theenumi}
\newcounter{property}
\renewcommand{\theproperty}{\textup{(\arabic{property})}}
\newcommand{\property}{\refstepcounter{property}\item}
\newcounter{prooperty}
\renewcommand{\theprooperty}{\textup{(\arabic{prooperty})}}
\newcommand{\prooperty}{\refstepcounter{prooperty}\item}

\makeatletter
\def\keywords#1{{\def\@thefnmark{\relax}\@footnotetext{#1}}}
\let\subjclass\keywords
\makeatother
%
\begin{document}
%
\sloppy
%\thispagestyle{empty}
%
\centerline{\large \bf Лекции курса \guillemotleft
Алгебра\guillemotright{}, лектор Р.\,С.~Авдеев}

\smallskip

\centerline{\large ФКН НИУ ВШЭ, 1-й курс ОП ПМИ, 4-й модуль,
2015/2016 учебный год}


\bigskip

\section*{Лекция 1}

\medskip

{\it Полугруппы и группы: основные определения и примеры. Группы
подстановок и группы матриц. Подгруппы. Порядок элемента и
циклические подгруппы. Смежные классы и индекс подгруппы. Теорема
Лагранжа.}

\medskip

\begin{definition}
{\it Множество с бинарной операцией}~--- это множество $M$ с
заданным отображением
$$
M\times M \to M, \quad (a,b) \mapsto a\circ b.
$$
\end{definition}

Множество с бинарной операцией обычно обозначают $(M,\circ)$.

\begin{definition}
Множество с бинарной операцией $(M,\circ)$ называется {\it
полугруппой}, если данная бинарная операция {\it ассоциативна},
т.\,е.
$$
a\circ (b \circ c) = (a\circ b)\circ c \quad \text{для всех} \ a,b,c\in M.
$$
\end{definition}

Не все естественно возникающие операции ассоциативны. Например, если
$M=\NN$ и $a\circ b:=a^b$, то
$$
2^{\left(1^2\right)}=2\ne (2^1)^2=4.
$$

Другой пример неассоциативной бинарной операции: $M = \ZZ$ и $a
\circ b := a - b$ (проверьте!).

Полугруппу обычно обозначают $(S,\circ)$.

\begin{definition}
Полугруппа $(S,\circ)$ называется {\it моноидом}, если в ней есть
{\it нейтральный элемент}, т.\,е. такой элемент $e\in S$, что
$e\circ a=a\circ e=a$ для любого $a\in S$.
\end{definition}

Во Франции полугруппа $(\NN,+)$ является моноидом, а в России нет.

\begin{remark}
Если в полугруппе есть нейтральный элемент, то он один. В самом
деле, $e_1\circ e_2=e_1=e_2$.
\end{remark}

\begin{definition}
Моноид $(S,\circ)$ называется {\it группой}, если для каждого
элемента $a\in S$ найдется {\it обратный элемент}, т.\,е. такой
$b\in S$, что $a\circ b = b\circ a= e$.
\end{definition}

\begin{exc}
Докажите, что если обратный элемент существует, то он один.
\end{exc}

Обратный элемент обозначается $a^{-1}$. Группу принято обозначать
$(G,\circ)$ или просто $G$, когда понятно, о какой операции идёт
речь. Обычно символ $\circ$ для обозначения операции опускают и
пишут просто $ab$.

\begin{definition}
Группа $G$ называется {\it коммутативной} или {\it абелевой}, если
групповая операция {\it коммутативна}, т.\,е. $ab=ba$ для любых
$a,b\in G$.
\end{definition}

Если в случае произвольной группы $G$ принято использовать
мультипликативные обозначения для групповой операции~--- $gh$, $e$,
$g^{-1}$, то в теории абелевых групп чаще используют аддитивные
обозначения, т.\,е. $a+b$, $0$, $-a$.

\begin{definition}
{\it Порядок} группы $G$~--- это число элементов в~$G$. Группа
называется {\it конечной}, если её порядок конечен, и {\it
бесконечной} иначе.
\end{definition}

Порядок группы $G$ обозначается $|G|$.

\smallskip

Приведём несколько серий примеров групп.

\smallskip

1) Числовые аддитивные группы: \ $(\ZZ,+)$, $(\QQ,+)$, $(\RR,+)$,
$(\CC,+)$, $(\ZZ_n,+)$.

\smallskip

2) Числовые мультипликативные группы: \
$(\QQ\setminus\{0\},\times)$, $(\RR\setminus\{0\},\times)$,
$(\CC\setminus\{0\},\times)$,
$(\ZZ_p\setminus\{\overline{0}\},\times)$, $p$~--- простое.

3) Группы матриц: \ $\GL_n(\RR)=\{A\in\Mat(n\times n, \RR) \mid
\det(A)\ne 0\}$;  \ $\SL_n(\RR)=\{A\in\Mat(n\times n, \RR) \mid
\det(A)=1\}$.

4) Группы подстановок: \ симметрическая группа $S_n$~--- все
подстановки длины $n$, $|S_n|=n!$;

знакопеременная группа $A_n$~--- чётные подстановки длины $n$,
$|A_n|=n!/2$.

\begin{exc}
Докажите, что группа $S_n$ коммутативна $\Leftrightarrow$ $n
\leqslant 2$, а $A_n$ коммутативна $\Leftrightarrow$ $n \leqslant
3$.
\end{exc}

\begin{definition}
Подмножество $H$ группы $G$ называется {\it подгруппой}, если выполнены следующие три условия: (1) $e \in H$; \quad (2) $ab\in H$ для любых $a,b
\in H$; \quad (3) $a^{-1}\in H$ для любого
$a\in H$.
\end{definition}

\begin{exc}
Проверьте, что $H$ является подгруппой тогда и только тогда, когда
 $H$ непусто и $ab^{-1}\in H$ для любых $a,b\in H$.
\end{exc}

В каждой группе $G$ есть {\it несобственные} подгруппы $H=\{e\}$ и
$H=G$. Все прочие подгруппы называются {\it собственными}. Например,
чётные числа $2\ZZ$ образуют собственную подгруппу в $(\ZZ,+)$.

\begin{proposition} \label{sbgrz}
Всякая подгруппа в $(\ZZ,+)$ имеет вид $k\ZZ$ для некоторого целого
неотрицательного $k$.
\end{proposition}

\begin{proof}
Пусть $H$~--- подгруппа в $\ZZ$. Если $H=\{0\}$, положим $k=0$.
Иначе пусть $k$~--- наименьшее натуральное число, лежащее в~$H$
(почему такое есть?). Тогда $k\ZZ \subseteq H$. С другой стороны,
если $a\in H$ и $a=qk+r$~--- результат деления $a$ на $k$ с
остатком, то $0 \leqslant r \leqslant k-1$ и $r = a - qk \in H$.
Отсюда $r=0$ и $H=k\ZZ$.
\end{proof}

\begin{definition}
Пусть $G$~--- группа и $g\in G$. {\it Циклической подгруппой},
порождённой элементом~$g$, называется подмножество $\{g^n \mid
n\in\ZZ\}$ в $G$.
\end{definition}

Циклическая подгруппа, порождённая элементом $g$, обозначается
$\langle g\rangle$. Элемент $g$ называется {\it порождающим} или
{\it образующим} для подгруппы $\langle g\rangle$. Например,
подгруппа $2\ZZ$ в $(\ZZ,+)$ является циклической, и в качестве
порождающего элемента в ней можно взять $g=2$ или $g=-2$. Другими
словами, $2\ZZ=\langle 2\rangle=\langle -2\rangle$.

\begin{definition}
Пусть $G$~--- группа и $g\in G$. {\it Порядком} элемента $g$
называется такое наименьшее натуральное число~$m$, что $g^m=e$. Если
такого натурального числа $m$ не существует, говорят, что порядок
элемента $g$ равен бесконечности.
\end{definition}

Порядок элемента обозначается $\ord(g)$. Заметим, что $\ord(g)=1$ тогда и только тогда, когда $g=e$.

Следующее предложение объясняет, почему для порядка группы и порядка элемента используется одно и то же слово.

\begin{proposition} \label{p1}
Пусть $G$~--- группа и $g\in G$. Тогда $\ord(g)=|\langle g\rangle|$.
\end{proposition}

\begin{proof}
Заметим, что если $g^k=g^s$, то $g^{k-s}=e$. Поэтому если элемент
$g$ имеет бесконечный порядок, то все элементы $g^n$, $n\in\ZZ$,
попарно различны, и подгруппа $\langle g\rangle$ содержит бесконечно
много элементов. Если же порядок элемента $g$ равен $m$, то из
минимальности числа $m$ следует, что элементы $e=g^0, g=g^1,
g^2,\ldots,g^{m-1}$ попарно различны. Далее, для всякого $n\in\ZZ$
мы имеем $n=mq+r$, где $0 \leqslant r \leqslant m-1$, и
$$
g^n=g^{mq+r}=(g^m)^qg^r=e^qg^r=g^r.
$$
Следовательно, $\langle g\rangle=\{e,g,\ldots, g^{m-1}\}$ и
$|\langle g\rangle|=m$.
\end{proof}

\begin{definition}
Группа $G$ называется {\it циклической}, если найдётся такой элемент
$g\in G$, что $G=\langle g\rangle$.
\end{definition}

Ясно, что любая циклическая группа коммутативна и не более чем
счётна. Примерами циклических групп являются группы $(\ZZ,+)$ и
$(\ZZ_n,+)$, $n \ge 1$.

Перейдем ещё к одному сюжету, связанному с парой группа--подгруппа.

\begin{definition}
Пусть $G$~--- группа, $H\subseteq G$~--- подгруппа и $g\in G$. {\it
Левым смежным классом} элемента $g$ группы $G$ по подгруппе $H$
называется подмножество
$$
gH=\{gh \mid h\in H\}.
$$
\end{definition}

\begin{lemma} \label{l1}
Пусть $G$~--- группа, $H\subseteq G$~--- её подгруппа и $g_1,g_2\in
G$. Тогда либо $g_1H=g_2H$, либо $g_1H\cap g_2H=\varnothing$.
\end{lemma}

\begin{proof}
Предположим, что $g_1H\cap g_2H\ne\varnothing$, т.\,е.
$g_1h_1=g_2h_2$ для некоторых $h_1,h_2\in H$. Нужно доказать, что
$g_1H=g_2H$. Заметим, что $g_1H=g_2h_2h_1^{-1}H\subseteq g_2H$.
Обратное включение доказывается аналогично.
\end{proof}

\begin{lemma} \label{l2}
Пусть $G$~--- группа и $H\subseteq G$~--- конечная подгруппа. Тогда
$|gH|=|H|$ для любого $g\in G$.
\end{lemma}

\begin{proof}
Поскольку $gH=\{gh \mid h\in H\}$, в $|gH|$ элементов не больше, чем
в~$H$. Если $gh_1=gh_2$, то домножаем слева на $g^{-1}$ и получаем
$h_1=h_2$. Значит, все элементы вида $gh$, где $h\in H$, попарно
различны, откуда $|gH|=|H|$.
\end{proof}

\begin{definition}
Пусть $G$~--- группа и $H\subseteq G$~--- подгруппа. {\it Индексом}
подгруппы $H$ в группе $G$ называется число левых смежных классов
$G$ по~$H$.
\end{definition}

Индекс группы $G$ по подгруппе $H$ обозначается $[G:H]$.

\smallskip

{\bf Теорема Лагранжа}.\ Пусть $G$~--- конечная группа и $H\subseteq
G$~--- подгруппа. Тогда
$$
|G| = |H| \cdot [G:H].
$$

\begin{proof}
Каждый элемент группы $G$ лежит в (своём) левом смежном классе по
подгруппе $H$, разные смежные классы не пересекаются
(лемма~\ref{l1}) и каждый из них содержит по $|H|$ элементов
(лемма~\ref{l2}).
\end{proof}

На следующей лекции мы обсудим следствия из данной теоремы.

\bigskip

\begin{thebibliography}{99}
\bibitem{Vi}
Э.\,Б.~Винберг. Курс алгебры. М.: Факториал Пресс, 2002 (глава~4,
$\S$~1,3,5)
\bibitem{Ko1}
А.\,И.~Кострикин. Введение в алгебру. Основы алгебры. М.: Наука.
Физматлит, 1994 (глава~4, $\S$~1-2)
\bibitem{Ko3}
А.\,И.~Кострикин. Введение в алгебру. Основные структуры алгебры.
М.: Наука. Физматлит, 2000 (глава~1, $\S$~2)
\bibitem{SZ}
Сборник задач по алгебре под редакцией А.\,И.~Кострикина. Новое
издание. М.: МЦНМО, 2009 (глава~13, $\S$~54-56)
\end{thebibliography}

\end{document}
