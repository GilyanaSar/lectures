\documentclass[a4paper,10pt]{amsart}

\usepackage[T2A]{fontenc}
\usepackage[utf8x]{inputenc}
\usepackage{amssymb}
\usepackage[russian]{babel}
\usepackage{geometry}
\usepackage{hyperref}
\usepackage{enumitem}

\geometry{a4paper,top=2cm,bottom=2cm,left=2cm,right=2cm}

\setlength{\parindent}{0pt}
\setlength{\parskip}{\medskipamount}

\newcommand{\Ker}{\mathop{\mathrm{Ker}}}
\renewcommand{\Im}{\mathop{\mathrm{Im}}}
\DeclareMathOperator{\Tor}{\mathrm{Tor}}
\newcommand{\xar}{\mathop{\mathrm{char}}}

%\def\Ker{{\rm Ker}}%
%\def\Im{{\rm Im}}%
\def\Mat{{\rm Mat}}%
\def\cont{{\rm cont}}%
%\def\Tor{{\rm Tor}}%
\def\Char{{\rm Char}}%
\def\signum{{\rm sig}}%
\def\Sym{{\rm Sym}}%
\def\St{{\rm St}}%
\def\Aut{{\rm Aut}}%
\def\Chi{{\mathbb X}}%
\def\Tau{{\rm T}}%
\def\Rho{{\rm R}}%
\def\rk{{\rm rk}}%
\def\ggT{{\rm ggT}}%
\def\kgV{{\rm kgV}}%
\def\Div{{\rm Div}}%
\def\div{{\rm div}}%
\def\quot{/\!\!/}%
\def\mal{\! \cdot \!}%
\def\Of{{\mathcal{O}}}
%
\def\subgrpneq{\le}%
\def\subgrp{\le}%
\def\ideal#1{\le_{#1}}%
\def\submod#1{\le_{#1}}%
%
\def\Bild{{\rm Bild}}%
\def\Kern{{\rm Kern}}%
\def\bangle#1{{\langle #1 \rangle}}%
\def\rq#1{\widehat{#1}}%
\def\t#1{\widetilde{#1}}%
\def\b#1{\overline{#1}}%
%
\def\abs#1{{\vert #1 \vert}}%
\def\norm#1#2{{\Vert #1 \Vert}_{#2}}%
\def\PS#1#2{{\sum_{\nu=0}^{\infty} #1_{\nu} #2^{\nu}}}%
%
\def\C{{\rm C}}%
\def\O{{\rm O}}%
\def\HH{{\mathbb H}}%
\def\LL{{\mathbb L}}%
\def\FF{{\mathbb F}}%
\def\CC{{\mathbb C}}%
\def\KK{{\mathbb K}}%
\def\TT{{\mathbb T}}%
\def\ZZ{{\mathbb Z}}%
\def\RR{{\mathbb R}}%
\def\SS{{\mathbb S}}%
\def\NN{{\mathbb N}}%
\def\QQ{{\mathbb Q}}%
\def\PP{{\mathbb P}}%
\def\AA{{\mathbb A}}%
%
\def\eins{{\mathbf 1}}%
%
\def\AG{{\rm AG}}%
\def\Aut{{\rm Aut}}%
\def\Hol{{\rm Hol}}%
\def\GL{{\rm GL}}%
\def\SL{{\rm SL}}%
\def\SO{{\rm SO}}%
\def\Sp{{\rm Sp}}%
\def\gl{\mathfrak{gl}}%
\def\rg{{\rm rg}}%
\def\sl{\mathfrak{sl}}%
\def\HDiv{{\rm HDiv}}%
\def\CDiv{{\rm CDiv}}%
\def\Res{{\rm Res}}%
\def\Pst{{\rm Pst}}%
\def\Nst{{\rm Nst}}%
\def\rad{{\rm rad}}%
\def\GL{{\rm GL}}%
\def\Tr{{\rm Tr}}%
\def\Pic{{\rm Pic}}%
\def\Hom{{\rm Hom}}%
\def\hom{{\rm hom}}%
\def\Mor{{\rm Mor}}%
\def\codim{{\rm codim}}%
\def\Supp{{\rm Supp}}%
\def\Spec{{\rm Spec}}%
\def\Proj{{\rm Proj}}%
\def\Maps{{\rm Maps}}%
\def\cone{{\rm cone}}%
\def\ord{{\rm ord}}%
\def\pr{{\rm pr}}%
\def\id{{\rm id}}%
\def\mult{{\rm mult}}%
\def\inv{{\rm inv}}%
\def\neut{{\rm neut}}%
%
\def\AAA{\mathcal{A}}
\def\BBB{\mathcal{B}}
\def\CCC{\mathcal{C}}
\def\EEE{\mathcal{E}}
\def\FFF{\mathcal{F}}

\def\CF{{\rm CF}}
\def\GCD{{\rm GCD}}
\def\Mat{{\rm Mat}}
\def\End{{\rm End}}
\def\cont{{\rm cont}}
\def\Kegel{{\rm Kegel}}
\def\Char{{\rm Char}}
\def\Der{{\rm Der}}
\def\signum{{\rm sg}}
\def\grad{{\rm grad}}
\def\Spur{{\rm Spur}}
\def\Sym{{\rm Sym}}
\def\Alt{{\rm Alt}}
\def\Abb{{\rm Abb}}
\def\Chi{{\mathbb X}}
\def\Tau{{\rm T}}
\def\Rho{{\rm R}}
\def\ad{{\rm ad}}
\def\Frob{{\rm Frob}}
\def\Rang{{\rm Rang}}
\def\SpRang{{\rm SpRang}}
\def\ZRang{{\rm ZRang}}
\def\ggT{{\rm ggT}}
\def\kgV{{\rm kgV}}
\def\Div{{\rm Div}}
\def\div{{\rm div}}
\def\quot{/\!\!/}
\def\mal{\! \cdot \!}
\def\add{{\rm add}}
\def\mult{{\rm mult}}
\def\smult{{\rm smult}}

\def\subgrpneq{\le}
\def\subgrp{\le}
\def\ideal#1{\unlhd_{#1}}
\def\submod#1{\le_{#1}}

\def\Bild{{\rm Bild}}
\def\Kern{{\rm Kern}}
\def\Kon{{\rm Kon}}
\def\bangle#1{{\langle #1 \rangle}}
\def\rq#1{\widehat{#1}}
\def\t#1{\widetilde{#1}}
\def\b#1{\overline{#1}}

\def\abs#1{{\vert #1 \vert}}
\def\norm#1#2{{\Vert #1 \Vert}_{#2}}
\def\PS#1#2{{\sum_{\nu=0}^{\infty} #1_{\nu} #2^{\nu}}}


\def\eins{{\mathbf 1}}

\def\ElM{{\rm ElM}}
\def\ZOp{{\rm ZOp}}
\def\SpOp{{\rm SpOp}}
\def\Gal{{\rm Gal}}
\def\Def{{\rm Def}}
\def\Fix{{\rm Fix}}
\def\ord{{\rm ord}}
\def\Aut{{\rm Aut}}
\def\Hol{{\rm Hol}}
\def\GL{{\rm GL}}
\def\SL{{\rm SL}}
\def\SO{{\rm SO}}
\def\Sp{{\rm Sp}}
\def\Spann{{\rm Spann}}
\def\Lin{{\rm Lin}}
\def\gl{\mathfrak{gl}}
\def\rg{{\rm rg}}
\def\sl{\mathfrak{sl}}
\def\so{\mathfrak{so}}
\def\sp{\mathfrak{sp}}
\def\gg{\mathfrak{g}}
\def\HDiv{{\rm HDiv}}
\def\CDiv{{\rm CDiv}}
\def\Res{{\rm Res}}
\def\Pst{{\rm Pst}}
\def\Nst{{\rm Nst}}
\def\WDiv{{\rm WDiv}}
\def\GL{{\rm GL}}
\def\Tr{{\rm Tr}}
\def\Pic{{\rm Pic}}
\def\Hom{{\rm Hom}}
\def\hom{{\rm hom}}
\def\Mor{{\rm Mor}}
\def\codim{{\rm codim}}
\def\Supp{{\rm Supp}}
\def\Spec{{\rm Spec}}
\def\Proj{{\rm Proj}}
\def\Maps{{\rm Maps}}
\def\cone{{\rm cone}}
\def\ord{{\rm ord}}
\def\pr{{\rm pr}}
\def\id{{\rm id}}
\def\mult{{\rm mult}}
\def\inv{{\rm inv}}
\def\neut{{\rm neut}}
\def\trdeg{{\rm trdeg}}
\def\sing{{\rm sing}}
\def\reg{{\rm reg}}


%%%%%%%%%%%%%%%%%%%%%%%%%%%

\newtheorem{theorem}{Теорема}
\newtheorem{proposition}{Предложение}
\newtheorem{lemma}{Лемма}
\newtheorem{corollary}{Следствие}
\theoremstyle{definition}
\newtheorem{definition}{Определение}
\newtheorem{problem}{Задача}
%
\theoremstyle{remark}
\newtheorem{exercise}{Упражнение}
\newtheorem{remark}{Замечание}
\newtheorem{example}{Пример}

\renewcommand{\theenumi}{\textup{(\alph{enumi})}}
\renewcommand{\labelenumi}{\theenumi}
\newcounter{property}
\renewcommand{\theproperty}{\textup{(\arabic{property})}}
\newcommand{\property}{\refstepcounter{property}\item}
\newcounter{prooperty}
\renewcommand{\theprooperty}{\textup{(\arabic{prooperty})}}
\newcommand{\prooperty}{\refstepcounter{prooperty}\item}

\makeatletter
\def\keywords#1{{\def\@thefnmark{\relax}\@footnotetext{#1}}}
\let\subjclass\keywords
\makeatother
%
\begin{document}
%
\sloppy
%\thispagestyle{empty}
%
\centerline{\large \bf Лекции курса \guillemotleft
Алгебра\guillemotright{}, лекторы И.\,В.~Аржанцев и Р.\,С.~Авдеев}

\smallskip

\centerline{\large ФКН НИУ ВШЭ, 1-й курс ОП ПМИ, 4-й модуль,
2014/2015 учебный год}


\bigskip

\section*{Лекция~10}

\medskip

{\it Конечные поля. Простое подполе и порядок конечного поля.
Автоморфизм Фробениуса. Теорема существования и единственности для
конечных полей. Поле из четырех элементов. Цикличность
мультипликативной группы. Неприводимые многочлены над конечным
полем. Подполя конечного поля.}

\medskip

В этой лекции будем использовать следующее обозначение: $K^\times =
K \setminus \lbrace 0 \rbrace$~--- мультипликативная группа
поля~$K$.

Пусть $K$~--- конечное поле. Тогда его характеристика отлична от
нуля и потому равна некоторому простому числу~$p$. Значит, $K$
содержит поле $\ZZ_p$ в качестве простого подполя.

\begin{theorem} \label{thm1}
Число элементов конечного поля равно $p^n$ для некоторого простого
$p$ и натурального $n$.
\end{theorem}

\begin{proof}
Пусть $K$~--- конечное поле характеристики~$p$, и пусть размерность
$K$ над простым подполем $\ZZ_p$ равна~$n$. Выберем в $K$ базис
$e_1, \ldots, e_n$ над $\ZZ_p$. Тогда каждый элемент из $K$
однозначно представляется в виде $\alpha_1 e_1 + \ldots + \alpha_n
e_n$, где $\alpha_1, \ldots, \alpha_n$ пробегают~$\ZZ_p$.
Следовательно, в $K$ ровно $p^n$ элементов.
\end{proof}

Пусть $K$~--- произвольное поле характеристики $p > 0$. Рассмотрим
отображение
$$
\varphi \colon K \to K, \quad a \mapsto a^p.
$$
Покажем, что $\varphi$~--- гомоморфизм. Для любых $a,b \in K$ по
формуле бинома Ньютона имеем
$$
(a + b)^p = a^p + C_p^1 a^{p-1}b + C_p^2 a^{p-2}b^2 + \ldots +
C_p^{p-1} a b^{p-1} + b^p.
$$
Так как $p$~--- простое число, то все биномиальные коэффициенты
$C_p^i$ при $1 \leqslant i \leqslant p-1$ делятся на~$p$. Это
значит, что в нашем поле характеристики $p$ все эти коэффициенты
обнуляются, в результате чего получаем $(a + b)^p = a^p + b^p$.
Ясно, что $(ab)^p = a^p b^p$, так что $\varphi$~--- гомоморфизм.
Ядро любого гомоморфизма колец является идеалом, поэтому $\Ker
\varphi$~--- идеал в~$K$. Но в поле нет собственных идеалов, поэтому
$\Ker \varphi = \lbrace 0 \rbrace$, откуда $\varphi$ инъективен.

Если поле $K$ конечно, то инъективное отображение из $K$ в $K$
автоматически биективно. В этой ситуации $\varphi$ называется {\it
автоморфизмом Фробениуса} поля $K$.

\begin{remark}
Пусть $K$~--- произвольное поле и $\psi$~--- произвольный
автоморфизм (т.\,е. изоморфизм на себя) поля~$K$. Легко видеть, что
множество неподвижных точек $K^{\psi} = \{ a \in K \mid \psi(a) =
a\}$ является подполем в~$K$.
\end{remark}

Прежде чем перейти к следующей теореме, обсудим понятие формальной
производной многочлена. Пусть $K[x]$~--- кольцо многочленов над
произвольным полем~$K$. Формальной производной называется
отображение $K[x] \to K[x]$, которое каждому многочлену $f(x) =
a_nx^n + a_{n-1}x^{n-1} + \ldots + a_1 x + a_0$ сопоставляет
многочлен $f'(x) = na_n x^{n-1} + (n-1)a_{n-1}x^{n-2} + \ldots +
a_1$. Из определения следует, что это отображение линейно. Легко
проверить, что для любых $f,g \in K[x]$ справедливо привычное нам
равенство $(fg)' = f'g + fg'$ (в~силу дистрибутивности умножения
проверка этого равенства сводится к случаю, когда $f,g$~---
одночлены). В~частности, $(f(x)^m)' = mf(x)^{m-1}$ для любых $f(x)
\in K[x]$ и $m \in \NN$.

\begin{theorem} \label{thm2}
Для всякого простого числа $p$ и натурального числа $n$ существует
единственное \textup(с точностью до изоморфизма\textup) поле из
$p^n$ элементов.
\end{theorem}

\begin{proof}
Положим $q = p^n$ для краткости.

{\it Единственность.}\ Пусть поле $K$ содержит $q$ элементов. Тогда
мультипликативная группа $K^{\times}$ имеет порядок $q-1$. По
следствию~3 из теоремы Лагранжа мы имеем $a^{q-1}=1$ для всех $a \in
K \setminus \{0\}$, откуда $a^q - a = 0$ для всех $a\in K$. Это
значит, что все элементы поля $K$ являются корнями многочлена $x^q -
x \in \ZZ_p[x]$. Отсюда следует, что $K$ является полем разложения
многочлена $x^q - x$ над $\ZZ_p$. Из теоремы о полях разложения,
формулировавшейся на прошлой лекции, следует, что поле $K$
единственно с точностью до изоморфизма.

\smallskip

{\it Существование.} Пусть $K$~--- поле разложения многочлена $f(x)
= x^q - x \in \ZZ_p[x]$. Тогда имеем $f'(x)= qx^{q-1} - 1 =\nobreak
-1$ ($qx^{q-1}$ обнуляется, так как $q$ делится на~$p$, а $p$~---
характеристика поля~$\ZZ_p$). Покажем, что многочлен $f(x)$ не имеет
кратных корней в~$K$. Действительно, если $\alpha$~--- корень
кратности $m \geqslant 2$, то $f(x) = (x - \alpha)^m g(x)$ для
некоторого многочлена $g(x) \in \ZZ_p[x]$. Но тогда $f'(x) =
m(x-\alpha)^{m-1} g(x) + (x - \alpha)^m g'(x)$, откуда видно, что
$f'(x)$ делится на $(x - \alpha)$. Но последнее невозможно, ибо
$f'(x) = -1$~--- многочлен нулевой степени. Итак, многочлен $f(x)$
имеет ровно $q$ различных корней в поле~$K$. Заметим, что эти
корни~--- в точности неподвижные точки автоморфизма $\varphi^n =
\underbrace{\varphi \circ \ldots \circ \varphi}_n$, где
$\varphi$~--- автоморфизм Фробениуса. В~самом деле, для элемента $a
\in K$ равенство $a^q - a = 0$ выполнено тогда и только тогда, когда
$a^{p^n} = a$, т.\,е. $\varphi^n(a) = a$. Значит, корни многочлена
$x^q-x$ образуют подполе в~$K$, которое по определению поля
разложения совпадает с~$K$. Следовательно, в поле $K$ ровно $q$
элементов.
\end{proof}

Конечныe поля еще называют {\it полями Галуа}. Поле из $q$ элементов
обозначают $\FF_q$. Например, $\FF_p \cong \ZZ_p$.

\begin{example}
Построим явно поле из четырёх элементов. Многочлен $x^2+x+1$
неприводим над $\ZZ_2$. Значит, факторкольцо $\ZZ_2[x]/(x^2+x+1)$
является полем и его элементы~--- это классы $\overline{0},
\overline{1}, \overline{x}, \overline{x+1}$ (запись $\overline a$
означает класс элемента $a$ в факторкольце $\ZZ_2[x]/(x^2+x+1)$).
Например, произведение $\overline{x} \cdot \overline{x+1}$~--- это
класс элемента $x^2+x$, который равен $\overline{1}$.
\end{example}

\begin{proposition}
Мультипликативная группа конечного поля $\FF_q$ является
циклической.
\end{proposition}

\begin{proof}
Заметим, что $\FF_q^\times$~--- конечная абелева группа, и обозначим
через $m$ её экспоненту (см. конец лекции~4). Предположим, что
группа $\FF_q^{\times}$ не является циклической. Тогда $m < q-1$ по
следствию~2 лекции~4. По определению экспоненты это значит, что $a^m
= 1$ для всех $a \in \FF_q^{\times}$. Но тогда многочлен $x^m-1$
имеет в поле $\FF_q$ больше корней, чем его степень,~---
противоречие.
\end{proof}

\begin{theorem}
Конечное поле $\FF_q$, где $q=p^n$, можно реализовать в виде
$\ZZ_p[x]/(h(x))$, где $h(x)$~--- неприводимый многочлен степени $n$
над $\ZZ_p$. В~частности, для всякого $n \in \NN$ в кольце
$\ZZ_p[x]$ есть неприводимый многочлен степени~$n$.
\end{theorem}

\begin{proof}
Пусть $\alpha$~--- порождающий элемент группы $\FF_q^{\times}$.
Тогда минимальное подполе $\ZZ_p(\alpha)$ поля~$\FF_q$, содержащее
$\alpha$, совпадает с~$\FF_q$. Значит, поле $\FF_q$ изоморфно полю
$\ZZ_p[x]/(h(x))$, где $h(x)$~--- минимальный многочлен элемента
$\alpha$ над $\ZZ_p$. Из результатов прошлой лекции следует, что
многочлен $h(x)$ неприводим. Поскольку степень расширения $[\FF_q :
\ZZ_p]$ равна~$n$, этот многочлен имеет степень~$n$.
\end{proof}

\begin{theorem}
Всякое подполе поля $\FF_q$, где $q=p^n$, изоморфно $\FF_{p^m}$, где
$m$~--- делитель числа $n$. Обратно, для каждого делителя $m$ числа
$n$ в поле $\FF_q$ существует ровно одно подполе из $p^m$ элементов.
\end{theorem}

\begin{proof}
Пусть $F$~--- подполе поля $\FF_q$. По определению простого подполя
имеем $F \supset \ZZ_p$, откуда $\xar F = p$. Тогда
теорема~\ref{thm1} нам сообщает, что $|F| = p^m$ для некоторого $m
\in \NN$. По теореме~\ref{thm2} имеем $F \cong\nobreak \FF_{p^m}$.
Обозначим через $s$ степень (конечного) расширения $F \subset
\FF_q$. Рассуждая так же, как в доказательстве теоремы~\ref{thm1},
мы получим $p^n = (p^m)^s$, откуда $p^n = p^{ms}$ и $m$ делит~$n$.

Пусть теперь $m$~--- делитель числа~$n$, т.\,е. $n = ms$ для
некоторого $s \in \NN$. Рассмотрим многочлены $f(x) = x^{p^n} - x$ и
$g(x) = x^{p^m} - x$ над $\ZZ_p$. Заметим, что для элемента $a \in
\FF_q$ равенства $a^{p^m} = a$ следует
$$
a^{p^n} = a^{p^{ms}} = a^{(p^m)^s} =
(\ldots((a^{p^m})^{p^m})^{p^m}\ldots)^{p^m} \ \text{($s$ раз возвели
в степень $p^m$)} = a.
$$
Поэтому каждый корень многочлена $g(x)$ является и корнем многочлена
$f(x)$. Отсюда поле разложения многочлена $f(x)$ лежит в поле
разложения многочлена $g(x)$. Значит, $\FF_{p^m}$ содержится в
$\FF_{p^n}$.

Наконец, все элементы подполя из $p^m$ элементов неподвижны при
автоморфизме $\psi = \underbrace{\varphi \circ \ldots \circ
\varphi}_m \colon x \mapsto x^{p^m}$ ($\varphi$~--- автоморфизм
Фробениуса). Поскольку число корней многочлена $x^{p^m}-x$ в поле
$\FF_q$ не превосходит~$p^m$, множество элементов данного подполя
совпадает с множеством неподвижных точек автоморфизма~$\psi$.
Значит, такое подполе единственно.
\end{proof}

\bigskip

\begin{thebibliography}{99}
\bibitem{Vi}
Э.\,Б.~Винберг. Курс алгебры. М.: Факториал Пресс, 2002 (глава~9,
\S\,5)
\bibitem{Ko3}
А.\,И.~Кострикин. Введение в алгебру. Основные структуры алгебры.
М.: Наука. Физматлит, 2000 (глава~5, \S\,2)
\bibitem{SZ}
Сборник задач по алгебре под редакцией А.\,И.~Кострикина. Новое
издание. М.: МЦНМО, 2009 (глава~14, \S\,68)
\bibitem{LN}
Р.~Лидл и Г.~Нидеррайтер. Конечные поля (2 тома). М.: Мир, 1988
(главы~2--3)
\end{thebibliography}



\end{document}
