\documentclass[a4paper,10pt]{amsart}

\usepackage[T2A]{fontenc}
\usepackage[utf8x]{inputenc}
\usepackage{amssymb}
\usepackage[russian]{babel}
\usepackage{geometry}
\usepackage{hyperref}

\geometry{a4paper,top=2cm,bottom=2cm,left=2cm,right=2cm}

\setlength{\parindent}{0pt}
\setlength{\parskip}{\medskipamount}

\newcommand{\Ker}{\mathop{\mathrm{Ker}}}
\renewcommand{\Im}{\mathop{\mathrm{Im}}}
\DeclareMathOperator{\Tor}{\mathrm{Tor}}
%\newcommand{\Tor}{\mathop{\mathrm{Tor}}}

%\def\Ker{{\rm Ker}}%
%\def\Im{{\rm Im}}%
\def\Mat{{\rm Mat}}%
\def\cont{{\rm cont}}%
%\def\Tor{{\rm Tor}}%
\def\Char{{\rm Char}}%
\def\signum{{\rm sig}}%
\def\Sym{{\rm Sym}}%
\def\St{{\rm St}}%
\def\Aut{{\rm Aut}}%
\def\Chi{{\mathbb X}}%
\def\Tau{{\rm T}}%
\def\Rho{{\rm R}}%
\def\rk{{\rm rk}}%
\def\ggT{{\rm ggT}}%
\def\kgV{{\rm kgV}}%
\def\Div{{\rm Div}}%
\def\div{{\rm div}}%
\def\quot{/\!\!/}%
\def\mal{\! \cdot \!}%
\def\Of{{\mathcal{O}}}
%
\def\subgrpneq{\le}%
\def\subgrp{\le}%
\def\ideal#1{\le_{#1}}%
\def\submod#1{\le_{#1}}%
%
\def\Bild{{\rm Bild}}%
\def\Kern{{\rm Kern}}%
\def\bangle#1{{\langle #1 \rangle}}%
\def\rq#1{\widehat{#1}}%
\def\t#1{\widetilde{#1}}%
\def\b#1{\overline{#1}}%
%
\def\abs#1{{\vert #1 \vert}}%
\def\norm#1#2{{\Vert #1 \Vert}_{#2}}%
\def\PS#1#2{{\sum_{\nu=0}^{\infty} #1_{\nu} #2^{\nu}}}%
%
\def\C{{\rm C}}%
\def\O{{\rm O}}%
\def\HH{{\mathbb H}}%
\def\LL{{\mathbb L}}%
\def\FF{{\mathbb F}}%
\def\CC{{\mathbb C}}%
\def\KK{{\mathbb K}}%
\def\TT{{\mathbb T}}%
\def\ZZ{{\mathbb Z}}%
\def\RR{{\mathbb R}}%
\def\SS{{\mathbb S}}%
\def\NN{{\mathbb N}}%
\def\QQ{{\mathbb Q}}%
\def\PP{{\mathbb P}}%
\def\AA{{\mathbb A}}%
%
\def\eins{{\mathbf 1}}%
%
\def\AG{{\rm AG}}%
\def\Aut{{\rm Aut}}%
\def\Hol{{\rm Hol}}%
\def\GL{{\rm GL}}%
\def\SL{{\rm SL}}%
\def\SO{{\rm SO}}%
\def\Sp{{\rm Sp}}%
\def\gl{\mathfrak{gl}}%
\def\rg{{\rm rg}}%
\def\sl{\mathfrak{sl}}%
\def\HDiv{{\rm HDiv}}%
\def\CDiv{{\rm CDiv}}%
\def\Res{{\rm Res}}%
\def\Pst{{\rm Pst}}%
\def\Nst{{\rm Nst}}%
\def\rad{{\rm rad}}%
\def\GL{{\rm GL}}%
\def\Tr{{\rm Tr}}%
\def\Pic{{\rm Pic}}%
\def\Hom{{\rm Hom}}%
\def\hom{{\rm hom}}%
\def\Mor{{\rm Mor}}%
\def\codim{{\rm codim}}%
\def\Supp{{\rm Supp}}%
\def\Spec{{\rm Spec}}%
\def\Proj{{\rm Proj}}%
\def\Maps{{\rm Maps}}%
\def\cone{{\rm cone}}%
\def\ord{{\rm ord}}%
\def\pr{{\rm pr}}%
\def\id{{\rm id}}%
\def\mult{{\rm mult}}%
\def\inv{{\rm inv}}%
\def\neut{{\rm neut}}%
%
\def\AAA{\mathcal{A}}
\def\BBB{\mathcal{B}}
\def\CCC{\mathcal{C}}
\def\EEE{\mathcal{E}}
\def\FFF{\mathcal{F}}

\def\CF{{\rm CF}}
\def\GCD{{\rm GCD}}
\def\Mat{{\rm Mat}}
\def\End{{\rm End}}
\def\cont{{\rm cont}}
\def\Kegel{{\rm Kegel}}
\def\Char{{\rm Char}}
\def\Der{{\rm Der}}
\def\signum{{\rm sg}}
\def\grad{{\rm grad}}
\def\Spur{{\rm Spur}}
\def\Sym{{\rm Sym}}
\def\Alt{{\rm Alt}}
\def\Abb{{\rm Abb}}
\def\Chi{{\mathbb X}}
\def\Tau{{\rm T}}
\def\Rho{{\rm R}}
\def\ad{{\rm ad}}
\def\Frob{{\rm Frob}}
\def\Rang{{\rm Rang}}
\def\SpRang{{\rm SpRang}}
\def\ZRang{{\rm ZRang}}
\def\ggT{{\rm ggT}}
\def\kgV{{\rm kgV}}
\def\Div{{\rm Div}}
\def\div{{\rm div}}
\def\quot{/\!\!/}
\def\mal{\! \cdot \!}
\def\add{{\rm add}}
\def\mult{{\rm mult}}
\def\smult{{\rm smult}}

\def\subgrpneq{\le}
\def\subgrp{\le}
\def\ideal#1{\unlhd_{#1}}
\def\submod#1{\le_{#1}}

\def\Bild{{\rm Bild}}
\def\Kern{{\rm Kern}}
\def\Kon{{\rm Kon}}
\def\bangle#1{{\langle #1 \rangle}}
\def\rq#1{\widehat{#1}}
\def\t#1{\widetilde{#1}}
\def\b#1{\overline{#1}}

\def\abs#1{{\vert #1 \vert}}
\def\norm#1#2{{\Vert #1 \Vert}_{#2}}
\def\PS#1#2{{\sum_{\nu=0}^{\infty} #1_{\nu} #2^{\nu}}}


\def\eins{{\mathbf 1}}

\def\ElM{{\rm ElM}}
\def\ZOp{{\rm ZOp}}
\def\SpOp{{\rm SpOp}}
\def\Gal{{\rm Gal}}
\def\Def{{\rm Def}}
\def\Fix{{\rm Fix}}
\def\ord{{\rm ord}}
\def\Aut{{\rm Aut}}
\def\Hol{{\rm Hol}}
\def\GL{{\rm GL}}
\def\SL{{\rm SL}}
\def\SO{{\rm SO}}
\def\Sp{{\rm Sp}}
\def\Spann{{\rm Spann}}
\def\Lin{{\rm Lin}}
\def\gl{\mathfrak{gl}}
\def\rg{{\rm rg}}
\def\sl{\mathfrak{sl}}
\def\so{\mathfrak{so}}
\def\sp{\mathfrak{sp}}
\def\gg{\mathfrak{g}}
\def\HDiv{{\rm HDiv}}
\def\CDiv{{\rm CDiv}}
\def\Res{{\rm Res}}
\def\Pst{{\rm Pst}}
\def\Nst{{\rm Nst}}
\def\WDiv{{\rm WDiv}}
\def\GL{{\rm GL}}
\def\Tr{{\rm Tr}}
\def\Pic{{\rm Pic}}
\def\Hom{{\rm Hom}}
\def\hom{{\rm hom}}
\def\Mor{{\rm Mor}}
\def\codim{{\rm codim}}
\def\Supp{{\rm Supp}}
\def\Spec{{\rm Spec}}
\def\Proj{{\rm Proj}}
\def\Maps{{\rm Maps}}
\def\cone{{\rm cone}}
\def\ord{{\rm ord}}
\def\pr{{\rm pr}}
\def\id{{\rm id}}
\def\mult{{\rm mult}}
\def\inv{{\rm inv}}
\def\neut{{\rm neut}}
\def\trdeg{{\rm trdeg}}
\def\sing{{\rm sing}}
\def\reg{{\rm reg}}


%%%%%%%%%%%%%%%%%%%%%%%%%%%

\newtheorem{theorem}{Теорема}
\newtheorem{proposition}{Предложение}
\newtheorem{lemma}{Лемма}
\newtheorem{corollary}{Следствие}
\theoremstyle{definition}
\newtheorem{definition}{Определение}
\newtheorem{problem}{Задача}
%
\theoremstyle{remark}
\newtheorem{exercise}{Упражнение}
\newtheorem{remark}{Замечание}
\newtheorem{example}{Пример}

\renewcommand{\theenumi}{\textup{(\alph{enumi})}}
\renewcommand{\labelenumi}{\theenumi}
\newcounter{property}
\renewcommand{\theproperty}{\textup{(\arabic{property})}}
\newcommand{\property}{\refstepcounter{property}\item}
\newcounter{prooperty}
\renewcommand{\theprooperty}{\textup{(\arabic{prooperty})}}
\newcommand{\prooperty}{\refstepcounter{prooperty}\item}

\makeatletter
\def\keywords#1{{\def\@thefnmark{\relax}\@footnotetext{#1}}}
\let\subjclass\keywords
\makeatother
%
\begin{document}
%
\sloppy
%\thispagestyle{empty}
%
\centerline{\large \bf Лекции курса \guillemotleft
Алгебра\guillemotright{}, лектор Р.\,С.~Авдеев}

\smallskip

\centerline{\large ФКН НИУ ВШЭ, 1-й курс ОП ПМИ, 4-й модуль,
2015/2016 учебный год}


\bigskip

\section*{Лекция 5}

\medskip

{\it Строение конечно порождённых абелевых груп (продолжение). Экспонента
конечной абелевой группы. Действие группы на множестве. Орбиты и стабилизаторы.}
%Транзитивные и свободные действия. Три действия группы на себе.
%Теорема Кэли. Классы сопряжённости.}

Продолжим доказательство теоремы с прошлой лекции.

\begin{theorem} \label{traz}
Всякая конечно порождённая абелева группа $A$ разлагается в прямую
сумму примарных и бесконечных циклических подгрупп, т.\,е.
\begin{equation} \label{eqn}
A \cong \ZZ_{p_1^{k_1}} \oplus \ldots \oplus \ZZ_{p_s^{k_s}} \oplus
\ZZ \oplus \ldots \oplus \ZZ,
\end{equation}
где $p_1, \ldots, p_s$~--- простые числа \textup(не обязательно
попарно различные\textup) и $k_1, \ldots, k_s \in \NN$. Кроме того,
число бесконечных циклических слагаемых, а~также число и порядки
примарных циклических слагаемых определено однозначно.
\end{theorem}

\begin{proof}
На прошлой лекции мы доказали существование разложения и то, что количество
бесконечных циклических групп $\ZZ$ определено однозначно. Для этого мы вводили
понятие \textit{подгруппы кручения}:
\begin{equation} \label{eqn3}
\Tor A = \langle c_1 \rangle_{p_1^{k_1}} \oplus \ldots \oplus
\langle c_s \rangle_{p_s^{k_s}}.
\end{equation}
Далее, для каждого простого числа $p$ определим в $A$ {\it подгруппу
$p$-кручения}
\begin{equation} \label{eqn4}
\Tor_p A := \{ a\in A \mid p^ka=0 \ \text{для некоторого} \ k \in
\NN \}.
\end{equation}
Ясно, что $\Tor_p A \subset \Tor A$. Выделим подгруппу $\Tor_p A$ в
разложении~(\ref{eqn3}). Легко видеть, что $\langle c_i
\rangle_{p_i^{k_i}} \subseteq \Tor_p A$ для всех $i$ с условием $p_i
= p$. Если же $p_i \ne p$, то по следствию~2 из теоремы Лагранжа
(см. лекцию~2) порядок любого ненулевого элемента $x \in \langle c_i
\rangle_{p_i^{k_i}}$ является степенью числа~$p_i$, а~значит, $p^k x
\ne 0$ для всех $k \in \NN$. Отсюда следует, что $\Tor_p A$ является
суммой тех конечных слагаемых в разложении~(\ref{eqn3}), порядки
которых суть степени~$p$. Поэтому доказательство теперь сводится к
случаю, когда $A$~--- примарная группа.

Пусть $|A|=p^k$ и
$$
A = \langle c_1\rangle_{p^{k_1}}\oplus\ldots\oplus\langle
c_r\rangle_{p^{k_r}}, \quad k_1+\ldots+k_r=k.
$$
Докажем индукцией по~$k$, что набор чисел $k_1, \ldots, k_r$ не
зависит от разложения.

Если $k = 1$, то $|A| = p$, но тогда $A \cong \ZZ_p$ по следствию~5
из теоремы Лагранжа (см. лекцию~2). Пусть теперь $k > 1$. Рассмотрим
подгруппу $pA: = \{ pa \mid a \in A \}$. В~терминах
равенства~(\ref{eqn4}) имеем
$$
pA = \langle pc_1 \rangle_{p^{k_1-1}} \oplus \ldots \oplus \langle
pc_r\rangle_{p^{k_r-1}}.
$$
В частности, при $k_i = 1$ соответствующее слагаемое равно $\lbrace
0 \rbrace$ (и тем самым исчезает). Так как $|pA| = p^{k - r} < p^k$,
то по предположению индукции группа $pA$ разлагается в прямую сумму
примарных циклических подгрупп однозначно с точностью до порядка
слагаемых. Следовательно, ненулевые числа в наборе $k_1 - 1, \ldots,
k_r-1$ определены однозначно (с точностью до перестановки). Отсюда
мы находим значения $k_i$, отличные от~$1$. Количество тех~$k_i$,
которые равны~$1$, однозначно восстанавливается из условия $k_1 +
\ldots + k_r = k$.
\end{proof}

Заметим, что теорема о согласованных базисах даёт нам другое
разложение конечной абелевой группы~$A$:
\begin{equation} \label{eqn5}
A=\ZZ_{u_1}\oplus\ldots\oplus\ZZ_{u_m}, \quad \text{где} \
u_i|u_{i+1} \ \text{при} \ i = 1, \ldots, m-1.
\end{equation}
Числа $u_1, \ldots, u_m$ называют {\it инвариантными множителями}
конечной абелевой группы~$A$.

\begin{definition}
{\it Экспонентой} конечной абелевой группы $A$ называется число
$\exp A$, равное наименьшему общему кратному порядков элементов
из~$A$. Легко заметить, что это равносильно следующему условию:
$$
\exp A = \min \lbrace n \in \NN \mid na = 0 \
\text{для всех} \ a \in A \rbrace
$$
\end{definition}

\begin{proposition}
Экспонента конечной абелевой группы~$A$ равна её последнему
инвариантному множителю~$u_m$.
\end{proposition}

\begin{proof}
Обратимся к разложению~(\ref{eqn5}). Так как $u_i | u_m$ для всех $i
= 1, \ldots, m$, то $u_ma=0$ для всех $a \in A$. Это означает, что
$\exp A \leqslant u_m$ (и тем самым $\exp A \, | u_m$). С~другой
стороны, в $A$ имеется циклическая подгруппа порядка $u_m$. Значит,
$\exp A \geqslant u_m$.
\end{proof}

\begin{corollary}
Конечная абелева группа $A$ является циклической тогда и только
тогда, когда $\exp A =\nobreak |A|$.
\end{corollary}

\begin{proof}
Группа $A$ является циклической тогда и только тогда, когда в
разложении~(\ref{eqn5}) присутствует только одно слагаемое, т.\,е.
$A = \ZZ_{u_m}$ и $|A| = u_m$.
\end{proof}


Пусть $G$~--- произвольная группа и $X$~--- некоторое множество.

\begin{definition}
\textit{Действием} группы $G$ на множестве $X$ называется
отображение $G\times X\to X$, $(g,x)\mapsto gx$, удовлетворяющее
следующим условиям:

1) $ex=x$ для любого $x\in X$ ($e$~--- нейтральный элемент
группы~$G$);

2) $g(hx)=(gh)x$ для всех $g,h\in G$ и $x\in X$.

Обозначение: $G:X$.
\end{definition}

Если задано действие группы $G$ на множестве~$X$, то каждый элемент
$g \in G$ определяет биекцию $a_g \colon X \to\nobreak X$ по правилу
$a_g(x) = gx$ (обратным отображением для $a_g$ будет $a_{g^{-1}}$).
Обозначим через $S(X)$ группу всех биекций (перестановок) множества
$X$ с операцией композиции. Тогда отображение $a \colon G \to S(X)$,
$g \mapsto a_g$, является гомоморфизмом групп. Действительно, для
произвольных элементов $g,h \in G$ и $x \in X$ имеем
$$
a_{gh}(x) = (gh)x = g(hx) = g a_h(x) = a_g (a_h(x)) = (a_g a_h)(x).
$$
Можно показать, что задание действия группы $G$ на множестве $X$
равносильно заданию соответствующего гомоморфизма $a \colon G \to
S(X)$.

\begin{example}
Симметрическая группа $S_n$ естественно действует на множестве $X =
\lbrace 1, 2, \ldots, n \rbrace$ по формуле $\sigma x =\nobreak
\sigma (x)$ ($\sigma \in S_n$, $x \in X$). Условие~1) здесь
выполнено по определению тождественной подстановки, условие~2)
выполнено по определению композиции подстановок.
\end{example}

Пусть задано действие группы $G$ на множестве~$X$.

\begin{definition}
{\it Орбитой} точки $x\in X$ называется подмножество
$$
Gx = \lbrace x' \in X \mid x' = gx \ \text{для некоторого} \ g \in G
\rbrace = \{ gx \mid g\in G\}.
$$
\end{definition}

\begin{remark}
Для точек $x, x' \in X$ отношение \guillemotleft$x'$ лежит в орбите
$Gx$\guillemotright{} является отношением эквивалентности:

(1) (рефлексивность) $x \in Gx$ для всех $x \in X$: это верно, так
как $x = ex \in Gx$ для всех $x \in X$;

(2) (симметричность) если $x' \in Gx$, то $x \in Gx'$: это верно,
так как из условия $x' = gx$ следует $x = ex = (g^{-1}g)x =
g^{-1}(gx) = g^{-1}x' \in Gx'$;

(3) (транзитивность) если $x' \in Gx$ и $x'' \in Gx'$, то $x'' \in
Gx$: это верно, так как из условий $x' = gx$ и $x'' = hx'$ следует
$x'' = hx' = h(gx) = (hg)x \in Gx$.

Отсюда вытекает, что множество $X$ разбивается в объединение попарно
непересекающихся орбит действия группы~$G$.
\end{remark}

\begin{definition}
{\it Стабилизатором \textup(стационарной подгруппой\textup)} точки
$x \in X$ называется подгруппа $\St(x) := \{ g \in G \mid gx = x
\}$.
\end{definition}

\begin{exercise}
Проверьте, что множество $\St(x)$ действительно является подгруппой
в~$G$.
\end{exercise}

%\begin{example}
%Рассмотрим действие группы $\SL_n(\RR)$, $n \geqslant 2$ на
%множестве~$\RR^n$, заданное формулой $(A, v) \mapsto A \cdot v$, где
%в правой части вектор $v$ рассматривается как столбец своих
%координат. Оказывается, что для этого действия имеется всего две
%орбиты $\lbrace 0 \rbrace$ и $\RR^n \setminus \lbrace 0 \rbrace$.
%Чтобы показать, что $\RR^n \setminus\nobreak \lbrace 0 \rbrace$
%действительно является одной орбитой, достаточно проверить, что
%всякий ненулевой вектор можно получить, подействовав на элемент
%$e_1$ (первый базисный вектор) подходящей матрицей из
%группы~$\SL_n(\RR)$. Пусть $v \in \RR^n$~--- произвольный вектор с
%координатами $(x_1, \ldots, x_n)$. Покажем, что существует
%матрица~$A \in \SL_n(\RR)$, для которой $Ae_1 = v$ или,
%эквивалентно,
%\begin{equation} \label{eqn1}
%A\begin{pmatrix} 1 \\ 0 \\ \vdots \\ 0 \end{pmatrix} =
%\begin{pmatrix} x_1\\ x_2 \\ \vdots \\ x_n \end{pmatrix}.
%\end{equation}
%Из уравнения~(\ref{eqn1}) следует, что в первом столбце матрицы~$A$
%должны стоять в точности числа $x_1, \ldots, x_n$. Как мы знаем из
%линейной алгебры, вектор $v$ можно дополнить до базиса $v, v_2,
%\ldots, v_n$ пространства~$\RR^n$. Пусть $A'$~--- квадратная матрица
%порядка~$n$, в которой по столбцам записаны координаты векторов $v,
%v_2, \ldots, v_n$. Эта матрица невырожденна и удовлетворяет условию
%$A'e_1 = v$ (а~также $A'e_i = v_i$ для всех $i = 2, \ldots, n$).
%Однако её определитель может быть отличен от~$1$. Поделив все
%элементы последнего столбца матрицы $A'$ на $\det A'$, мы получим
%искомую матрицу~$A$ с определителем~$1$. Итак, мы показали, что
%$\RR^n \setminus \lbrace 0 \rbrace$~--- одна орбита для нашего
%действия. Легко видеть, что стабилизатор точки $e_1$ при этом будет
%состоять из всех матриц в $\SL_n(\RR)$, у которых первый столбец
%равен $\begin{pmatrix} 1 \\ 0 \\ \vdots \\ 0 \end{pmatrix}$. (У
%любой другой точки стабилизатор будет другим!)
%\end{example}

\begin{lemma}
Пусть конечная группа $G$ действует на множестве~$X$. Тогда для
всякого элемента $x\in X$ справедливо равенство
$$
|Gx| = |G| / |\St(x)|.
$$
В~частности, число элементов в \textup(любой\textup) орбите делит
порядок группы~$G$.
\end{lemma}

\begin{proof}
Рассмотрим множество\footnote{Это множество может не быть
факторгруппой, так как подгруппа $\St(x)$ не обязана быть нормальной
в~$G$.} $G / \St(x)$ левых смежных классов группы $G$ по подгруппе
$\St(x)$ и определим отображение $\psi \colon G / \St(x) \to Gx$ по
формуле $g\St(x) \mapsto gx$. Это определение корректно, поскольку
для любого другого представителя $g'$ левого смежного класса
$g\St(x)$ имеем $g' = g h$, где $h \in \St(x)$, и тогда $g'x = (gh)x
= g(hx) = gx$. Сюръективность отображения $\psi$ следует из
определения орбиты $Gx$. Проверим инъективность. Предположим, что
$g_1 \St(x) = g_2 \St(x)$ для некоторых $g_1, g_2 \in G$. Тогда
$g_1x = g_2x$. Подействовав на левую и правую части элементом
$g_2^{-1}$, получим $(g_2^{-1}g_1)x = x$, откуда $g_2^{-1}g_1 \in
\St(x)$. Последнее и означает, что $g_1 \St(x) = g_2 \St(x)$. Итак,
мы показали, что отобржание $\psi$ является биекцией. Значит, $|Gx|
= |G / \St(x)| = [G : \St(x)]$ и требуемое равенство вытекает из
теоремы Лагранжа (см. лекцию~1).
\end{proof}

\begin{example}
Рассмотрим действие группы $S^1 = \lbrace z \in \CC \mid |z| = 1
\rbrace$ на множестве~$\CC$, заданное
формулой $(z,w) \mapsto zw$, где $z \in S^1$, $w \in \CC$,
а~$zw$~--- обычное произведение комплексных чисел. Для этого
действия орбитами будут множества вида $|z| = c$, где $c \in
\RR_{\geqslant 0}$,~--- это всевозможные окружности с центром в
нуле, а также отдельная орбита, состоящая из нуля. Имеем
$$
\St(z) =
\begin{cases}
\lbrace 1 \rbrace, & \text{если} \ z \ne 0;\\
S^1, & \text{если} \ z = 0.
\end{cases}
$$
\end{example}

%
%Пусть снова группа $G$ действует на множестве~$X$.
%
%\begin{definition}
%Действие $G$ на $X$ называется {\it транзитивным}, если для любых
%$x, x' \in X$ найдётся такой элемент $g \in G$, что $x' = gx$.
%(Иными словами, все точки множества $X$ образуют одну орбиту.)
%\end{definition}
%
%\begin{definition}
%Действие $G$ на $X$ называется {\it свободным}, если для любой точки
%$x \in X$ условие $gx=x$ влечёт $g=e$. (Иными словами, $\St(x) =
%\lbrace e \rbrace$ для всех $x \in X$.)
%\end{definition}
%
%\begin{definition}
%Действие $G$ на $X$ называется {\it эффективным}, если условие
%$gx=x$ для всех $x\in X$ влечёт $g=e$. (Иными словами, $\bigcap
%\limits_{x \in X} \St(x) = \lbrace e \rbrace$.)
%\end{definition}
%
%\begin{remark}
%Из определений следует, что всякое свободное действие эффективно.
%Обратное утверждение неверно, как показывает пример~1 при $n
%\geqslant 3$, см. ниже.
%\end{remark}
%
%В~примерах 1--3 все действия эффективны. В~примере~1 действие
%транзитивно, свободно при $n \leqslant 2$ и не свободно при $n
%\geqslant 3$. В~примере~2 действие не транзитивно и не свободно; но
%если его ограничить на подмножество $\CC \setminus \lbrace 0
%\rbrace$ (то есть выбросить из $\CC$ точку~$0$), то оно станет
%свободным. В примере~3 действие не транзитивно и не свободно; но
%если его ограничить на подмножество $\RR^n \setminus \lbrace 0
%\rbrace$, то оно станет транзитивным.
%
%\begin{remark}
%Действие $G$ на $X$ эффективно тогда и только тогда, когда
%определяемый им гомоморфизм $a \colon G \to S(X)$ инъективен.
%\end{remark}
%
%\begin{definition}
%{\it Ядром неэффективности} действия группы $G$ на множестве~$X$
%называется подгруппа $K = \{ g\in G \mid gx = x \ \text{для всех} \
%x\in X\}$.
%\end{definition}
%
%Легко проверить, что $K = \Ker a$, где $a \colon G \to S(X)$~---
%определяемый действием гомоморфизм. Отсюда следует, что $K$~---
%нормальная подгруппа в~$G$. Рассмотрим факторгруппу $G/K$ и
%определим её действие на множестве $X$ по формуле $(gK)x = gx$.
%Поскольку $kx = x$ для всех $k \in K$ и $x \in X$, действие
%определено корректно.
%
%\begin{lemma}
%Определённое выше действие группы $G/K$ на множестве $X$ является
%эффективным.
%\end{lemma}
%
%\begin{proof}
%Пусть элемент $g \in G$ таков, что $(gK)x = x$ для всех $x \in X$.
%Тогда $gx = x$ для всех $x \in X$, откуда $g \in K$ и $gK = K$.
%\end{proof}
%
%Пусть $G$~--- произвольная группа. Рассмотрим три действия $G$ на
%самой себе, т.\,е. положим $X=G$:
%
%1) действие {\it умножениями слева}: $(g,h)\mapsto gh$;
%
%2) действие {\it умножениями справа}: $(g,h)\mapsto hg^{-1}$;
%
%3) действие {\it сопряжениями}: $(g,h)\mapsto ghg^{-1}$.
%
%Непосредственно проверяется, что первые два действия свободны и
%транзитивны. Орбиты третьего действия называются {\it классами
%сопряжённости} группы~$G$. Например, $\{e\}$~--- класс сопряжённости
%в любой группе. В~частности, для нетривиальных групп действие
%сопряжениями не является транзитивным.
%
%\begin{definition}
%Два действия группы $G$ на множествах $X$ и $Y$ называются {\it
%изоморфными}, если существует такая биекция $\varphi\colon X\to Y$,
%что
%\begin{equation} \label{eqn2}
%\varphi(gx)=g\varphi(x) \ \text{для любых} \ g\in G, x\in X.
%\end{equation}
%\end{definition}
%
%\begin{proposition}
%Всякое свободное транзитивное действие группы $G$ на множестве $X$
%изоморфно действию группы $G$ на себе левыми сдвигами.
%\end{proposition}
%
%\begin{proof}
%Зафиксируем произвольный элемент $x\in X$. Покажем, что отображение
%$\varphi \colon G\to X$, заданное формулой $\varphi(h) = hx$,
%является искомой биекцией. Сюръективность (соответственно
%инъективность) отображения $\varphi$ следует из транзитивности
%(соответственно свободности) действия $G$ на~$X$.
%Условие~(\ref{eqn2}) следует из цепочки равенств $\varphi(gh) =
%(gh)x = g(hx) = g(\varphi(h))$.
%\end{proof}
%
%\begin{corollary}
%Действия группы $G$ на себе правыми и левыми сдвигами изоморфны.
%\end{corollary}
%
%\smallskip
%
%{\bf Теорема Кэли.} Всякая конечная группа $G$ порядка $n$ изоморфна
%подгруппе симметрической группы~$S_n$.
%
%\begin{proof}
%Рассмотрим действие группы $G$ на себе левыми сдвигами. Как мы
%знаем, это действие свободно, поэтому соответствующий гомоморфизм $a
%\colon G \to S(G) \simeq\nobreak S_n$ инъективен, т.\,е. $\Ker a =
%\lbrace e \rbrace$. Учитывая, что $G / \lbrace e \rbrace \cong G$,
%по теореме о гомоморфизме получаем $G \cong \Im a$.
%\end{proof}


\bigskip

\begin{thebibliography}{99}
\bibitem{Vi}
Э.\,Б.~Винберг. Курс алгебры. М.: Факториал Пресс, 2002 (глава~10,
\S\,3)
\bibitem{Ko3}
А.\,И.~Кострикин. Введение в алгебру. Основные структуры алгебры.
М.: Наука. Физматлит, 2000 (глава~1, \S\,3)
\bibitem{SZ}
Сборник задач по алгебре под редакцией А.\,И.~Кострикина. Новое
издание. М.: МЦНМО, 2009 (глава~13, \S\,57)
\end{thebibliography}

\end{document}
