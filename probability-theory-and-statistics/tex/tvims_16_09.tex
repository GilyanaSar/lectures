% Размер страницы и шрифта
\documentclass[12pt,a4paper]{article}

%% Работа с русским языком
\usepackage{cmap}                   % поиск в PDF
\usepackage{mathtext}               % русские буквы в формулах
\usepackage[T2A]{fontenc}           % кодировка
\usepackage[utf8]{inputenc}         % кодировка исходного текста
\usepackage[english,russian]{babel} % локализация и переносы

%% Изменяем размер полей
\usepackage[top=0.5in, bottom=0.75in, left=0.625in, right=0.625in]{geometry}

%% Различные пакеты для работы с математикой
\usepackage{mathtools}  % Тот же amsmath, только с некоторыми поправками
\usepackage{amssymb}    % Математические символы
\usepackage{amsthm}     % Пакет для написания теорем
\usepackage{amstext}
\usepackage{array}
\usepackage{amsfonts}
\usepackage{icomma}     % "Умная" запятая: $0,2$ --- число, $0, 2$ --- перечисление

%% Графика
\usepackage[pdftex]{graphicx}
\graphicspath{{images/}}

%% Прочие пакеты
\usepackage{listings}               % Пакет для написания кода на каком-то языке программирования
\usepackage{algorithm}              % Пакет для написания алгоритмов
\usepackage[noend]{algpseudocode}   % Подключает псевдокод, отключает end if и иже с ними
\usepackage{indentfirst}            % Начало текста с красной строки
\usepackage[colorlinks=true, urlcolor=blue]{hyperref}   % Ссылки
\usepackage{pgfplots}               % Графики
\pgfplotsset{compat=1.12}
\usepackage{forest}                 % Деревья
\usepackage{titlesec}               % Изменение формата заголовков
\usepackage[normalem]{ulem}         % Для зачёркиваний
\usepackage[autocite=footnote]{biblatex}    % Кавычки для цитат и прочее
\usepackage[makeroom]{cancel}       % И снова зачёркивание (на этот раз косое)

% Изменим формат \section и \subsection:
\titleformat{\section}
	{\vspace{1cm}\centering\LARGE\bfseries} % Стиль заголовка
	{}                                      % префикс
	{0pt}                                   % Расстояние между префиксом и заголовком
	{}                                      % Как отображается префикс
\titleformat{\subsection}                   	% Аналогично для \subsection
	{\Large\bfseries}
	{}
	{0pt}
	{}

% Поправленный вид lstlisting
\lstset { %
    backgroundcolor=\color{black!5}, % set backgroundcolor
    basicstyle=\footnotesize,% basic font setting
}

% Теоремы и утверждения. В комменте указываем номер лекции, в которой это используется.
\newtheorem*{hanoi_recurrent}{Свойство} % Лекция 1
\let\epsilent\varepsilon                % Лекция 8
\DeclareMathOperator{\rk}{rank}         % Лекция 20

% Информация об авторах
\author{Группа лектория ФКН ПМИ 2015-2016 \\
	Никита Попов \\
	Тамерлан Таболов \\
	Лёша Хачиянц}
\title{Лекции по предмету \\
	\textbf{Алгоритмы и структуры данных}}
\date{2016 год}

\newcommand\independent{\protect\mathpalette{\protect\independenT}{\perp}}
\def\independenT#1#2{\mathrel{\rlap{$#1#2$}\mkern2mu{#1#2}}}

\begin{document}

\section*{Лекция 2 от 16.09.2016}

\subsection{Задача о сумасшедшей старушке}

\textbf{Условие:} Есть самолёт имеющий $n$ мест, в который садятся $n$ пассажиров. Первой в него заходит некоторая старушка, которая своего места не знает, и садится на случайное; каждый следующий пассажир действует правильно: садится на своё место, если оно свободно, и на случайное, если своё занято. 

\paragraph{Вопрос 1:} Какова вероятность того, что последний пассажир сядет на своё место?

~\

\noindent\textbf{Ответ:} Правильный ответ, как ни странно, угадывается. Это $\frac{1}{2}$ (прямо как в задаче про динозавра --- либо сядет, либо не сядет). Кажется неверным, но если задуматься, становится понятно, что есть только два варианта того, куда последний пассажир может сесть --- либо на своё место, либо на место старушки. 

\paragraph{Вопрос 2:} Какова вероятность, что предпоследний пассажир сядет на своё место?

~\

\noindent\textbf{Ответ:} Тут ответом является $\frac{2}{3}$. Рассуждение похоже на предыдущий пункт. Есть $3$ варианта места, куда может сесть старушка: к себе, на место предпоследнего или на место последнего пассажира. Нас устраивают $2$ из них.


\paragraph{Вопрос 3:} Какова вероятность того, что они оба сядут на свои места?

~\ 

\noindent\textbf{Ответ:} Как уже можно догадаться, $\frac{1}{3}$.

\paragraph{} Стоит строже объяснить, почему вышесказанное верно:

\paragraph{1)} 
$A = $ \{ последний сядет на свое место \}.
\begin{itemize}
    \item $n = 2$;\par $P(A) = \frac{1}{2}$;
    \item $n = 3$;\par $P(A) = \frac{1}{3} \text{(старушка села на своё место)} + \frac{1}{3}P(A\mid \text{старушка села на второе место}) = \frac{1}{3} +~\frac{1}{6} =~ \frac{1}{2}$.
\end{itemize}

    \textbf{Гипотеза:} $\forall n,\; P(A) = \frac{1}{2}$. \\ 
    Пусть $B_i = \{\text{старушка села на место пассажира с номером $i$, $1 \leqslant i \leqslant n$}\}$; считаем, что её номер равен 1. Воспользуемся методом индукции:

    \textbf{База:} Для $k=2$ верно.
    \textbf{Переход:} Пусть для всех $k < n$ гипотеза верна; докажем для $k = n$:

    \[
    P(A) = \{\text{формула полной вероятности}\} = \sum\limits_{i=1}^n P(A\mid B_i) \cdot P(B_i) 
    \]
	\[P(A\mid\ B_i) = \frac{1}{n},\; \forall i = 1..n\]
    \[P(A\mid B_1) = 1\]
    \[P(A\mid B_n) = 0\]
    \[P(A\mid B_i) = \frac{1}{2}, \; 2\leqslant i\leqslant n-1 \text{, т.к. теперь $i$-ый пассажир ``стал старушкой''.}\]

    \[P(A) = \frac{n-2}{n}\cdot \frac{1}{2} + \frac{1}{n}\cdot 1 + \frac{1}{n}\cdot 0 = \frac{1}{2} \qed\]

\paragraph{2)}
    $C = $ \{ последний сядет на свое место \}.
\begin{itemize}
    \item $n=3$: $P(C) = \frac{2}{3}$ --- у старушки есть $3$ варианта, при этом два из них (свое и последнее места) нас устраивают.
\end{itemize}

\textbf{Гипотеза:} $\forall n,\; P(C) = \frac{2}{3}$. \\ 

\textbf{База:} Для $k=3$ верно. Докажем для $k=n:$

    \[P(C\mid B_1) = 1\]
    \[P(C\mid B_n) = 1\]
    \[P(C\mid B_{n-1}) = 0\]
    \[P(C\mid B_i) = \frac{2}{3}, \; 2\leqslant i\leqslant n-2\]
	\[P(C) = \frac{n-3}{n} \cdot \frac{2}{3} + \frac{2}{n} \cdot 1 + \frac{1}{n} \cdot 0 = \frac{2}{3}\]

\paragraph{3)}
    $D$ = \{ последние $2$ пассажира сели на свои места \}.
	
    $D = A\cap C \implies P(D) = P(A\cap B).$ 
    
~\
    
 \textbf{Гипотеза:} $\forall n,\; P(D) = \frac{1}{3}$. \\
 
 Индукция с той же базой.
 
    \[P(D) = \{\text{формула полной вероятности}\} = \sum\limits_{i=1}^n P(D\mid B_i)P(B_i)\]


    \[P(D\mid B_1) = 1\]
    \[P(D\mid B_n) = 0\]
    \[P(D\mid B_{n-1}) = 0\]
    \[P(D\mid B_i) = \frac{1}{3}, \; 2\leqslant i\leqslant n-2\]

    \[P(D) = \frac{n-3}{n}\cdot \frac{1}{3} + \frac{1}{n}\cdot 1 + \frac{2}{n}\cdot 0 = \frac{1}{3} \qed\]

    Кажется, что эта вероятность равна произведению двух прошлых; \emph{это \sout{счастливое соврадение} неспроста.}

\subsection{Удачливый студент}

\textbf{Условие:}

Студент знает $k$ билетов из $n$. Каким ему нужно встать в очередь из $n$ студентов, чтобы вероятность вытянуть ``хороший'' билет была максимальнв?

\textbf{Решение:} 

Пусть $A_s = \{\text{студент вытянул хороший билет, стоя на $s$-ом месте в очереди}\}$;

$B_i = \{\text{до студента взяли ровно $i$ ``хороших билетов''}\}$.

\[P(A_s) = \sum\limits_{c = 0}^{\min(k,s-1)} P(A_s\mid B_i)P(B_i)\]

\[P(A_s\mid B_i) = \frac{k-i}{n-s+1}\]
\[P(B_i) = \frac{C_k^i \cdot C_{n-k}^{s-i-1} \cdot \text{\sout{$(s-1)!$}}}{C_n^{s-1} \cdot \text{\sout{$(s-1)!$}}}\]

\[P(A_s) = \sum\limits_{i=max(..)}^{min(..)} \frac{k-i}{n-s+1} \cdot \frac{C_k^i \cdot C_{n-k}^{s-i-1}}{C_n^{s-1}} = \sum\limits_{i=(..)}^{min(..)} \frac{k}{n} \cdot \frac{C_{k-1}^i \cdot C_{n-k}^{s-i-1}}{C_{n-1}^{s-1}} =
\]
\[ = \frac{k}{n} \cdot \sum\limits_{i = 0}^{min(k-1, s-1)} \frac{C_{k-1}^i \cdot C_{n-k}^{s-i-1}}{C_{n-1}^{s-1}} = \frac{k}{n} \qed
\]


\subsection{Формула Байеса}

Пусть $\{B_i, i = 1\ldots n\}$ --- разбиение $\Omega$, причём $P(B_i) > 0$. Тогда для события $A$ т.ч. $P(A) > 0$ выполняется

\[P(B_i\mid A) = \frac{P(A\mid B_i) \cdot P(B_i)}{\sum\limits_{j=1}^nP(A\mid B_j) \cdot P(B_j)}\]

Доказательство тривиально:
\[P(B_i\mid A) = \frac{P(A\cap B_i)}{P(A)} = \frac{P(A\mid B_i) \cdot P(B_i)}{P(A)} = \{ \text{ф-ла полной вероятности} \} = \frac
{P(A\mid B_i) \cdot P(B_i)}
{\sum\limits_{j=1}^{n}P(A\mid B_j) \cdot P(B_j)} \qed
\]

\subsection{Независимость}

\textbf{Определение:} события $A$ и $B$ называются \emph{независимыми} если, $P(A\cap B) = P(A)P(B)$; обозначение  --- $A \independent B$. 

~\

\textbf{Примеры:}
\begin{itemize}
    \item Задача про старушку; $A = \{\text{последний сел на своё место}\}$, \\ $B = \{\text{предпоследний сел на своё место}\}$ $A$ и $C$ --- независимы.
    \item Бросок игральной кости;  $A = \{\text{выпало чётное число}\}$, \\ $B = \{\text{выпало число, делящееся на три}\}$
	\[P(A) = \frac{1}{2},\; P(B) =\frac{1}{3}.\; P(A\cap B) = P(\text{``шестерка'')} = \frac{1}{6} = P(A) \cdot P(B).\]
	
\end{itemize}

\textbf{Определение:} cобытия $A_1, \ldots, A_n$ называются \emph{попарно независимыми}, если $\forall i \neq j$ $A_i$ независимо от $A_j$.

\textbf{Определение:} cобытия $A_1, \ldots, A_n$ называются \emph{независимыми в совокупности}, если $\forall k ~\leqslant~ n$, $1\leqslant i_1<\ldots< i_k \leqslant n$ выполнено  
$P(A_{i_1}\cap\ldots\cap A_{i_k}) = \prod\limits_{j=1}^{k} P(A_{i_j})$.

\underline{Замечание:} независимость в сосокупности $\neq$ попарной независимости.

~\

\textbf{Пример:} есть тетраэдр с раскрашенными гранями: К, С, З и КСЗ. Три события:
\begin{itemize}
    \item  $A_{red} = \{\text{на нижней грани есть красный цвет}\}$
    \item  $A_{blue} = \{\text{на нижней грани есть синий цвет}\}$
    \item  $A_{green} = \{\text{на нижней грани есть зелёный цвет}\}$
\end{itemize}

Очевидно, что вероятность любого события --- $\frac{1}{2}$; любой пары событий --- $\frac{1}{4}$; однако вероятность всех трёх разом не равна $\frac{1}{8}$, значит, эти события независимы попарно, но не в совокупности.

~\

\textbf{Упражнение:} привести пример событий, т.ч. любой набор из $n-1$ события независим, а все вместе $n$ событий вместе --- зависимы.

~\

\textbf{Утверждения:}
\begin{itemize}
    \item $A$ независимо с $A \Leftrightarrow P(A) = 0$ или $1$ $\Leftrightarrow$ $A$ независимо с любым другим событием.
    \item $A \independent B \implies \overline{A}\independent B$
    \item Если $A_1 \ldots A_n$ --- независимы в совокупности, то $\forall B_1 \dots B_n : \; B_i = A_i \text{ или } B_i = \overline{A_i}$ --- тоже независимы.
\end{itemize}

\section{Случайные величины в дискретных вероятностных пространствах}

\textbf{Определение:} пусть $(\Omega, P)$ --- вероятностное дискретное пространство; отображение \\ $\xi: \Omega \to \mathbb{R}$ называется \emph{случайной величиной (c.в.)}.

~\

\textbf{Примеры:} 
\begin{itemize}
    \item Индикаторы.

        Пусть $A\in \Omega$ --- событие. Тогда \emph{индикатором} события $A$ называют называется с.в.

        \[
            I_A(\omega) = \begin{cases}
                1, w \in A; \\
                1, w \not\in A;
            \end{cases}
        \]

    \item Бросок игральной кости;
    $\xi$ --- число очков на кубике, с.в.

    \item Схема Бернулли;\par
    $\Omega = \{\omega = (\omega_1 \ldots \omega_n),\; \omega_i \in \{0, 1\} \}$.
    $\xi(\omega) = \sum\limits_{i=1}^{n}\omega_i$ --- число ``успехов'' в схеме Бернулли.
\end{itemize}

\end{document}
