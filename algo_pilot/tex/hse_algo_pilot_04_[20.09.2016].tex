\documentclass[a4paper, 12pt]{article}
\usepackage{header}

\begin{document}
\pagestyle{fancy}

\section{Лекция 4 от 20.09.2016. Простейшие теоретико числовые алгоритмы}

Числовые алгоритмы играют огромную роль в криптографии, фактически вся криптография
держится на том, что не придуман до сих пор алгоритм, который умеет факторизовать числа за
полиномиальное время от размера числа.

\subsection{Алгоритм Евклида}

Начнём, пожалуй, с одного из самых известных алгоритмов нахождения наибольшего
общего делителя, а именно --- алгоритм Евклида и его расширенную версию.

\begin{algorithm}
  \caption{Алгоритм Евклида.}
  \begin{algorithmic}[1]
    \Function{$\gcd$}{int $a$, int $b$} 
      \If{$b = 0$}
        \State return $a$;
      \Else
        \State return $\gcd(b, a \Mod b)$;
      \EndIf
    \EndFunction
  \end{algorithmic}
\end{algorithm}

Практически очевидно, что данный алгоритм возвращает нужное нам число.
Вспомните курс дискретной математики или выпишите на бумаге то,
что делает данный алгоритм. 

Асимптотика такого алгоритма $O(\log n)$ (где $n$ --- максимальное значение числа)
--- легко проверить, что каждое число уменьшается хотя бы в 2 раза за 2 шага алгоритма.

\subsection{Расширенный алгоритм Евклида}

Пусть даны числа $a, b, c$, мы хотим найти хотя бы одну пару решений $x, y$ таких,
что $ax + by = c$. Понятно, что $\gcd(a, b) \ | \ c$, поэтому если это условие
не выполняется, то найти решение мы не сможем. Пусть $c = k \gcd(a, b)$. Сейчас
мы предъявим хотя бы одну пару чисел $x, y$, что $ax + by = \gcd(a, b)$ --- после этого
мы просто домножим $x, y$ на $k$ и получим, что сможем представить $c$ в таком
виде.

\begin{algorithm}
  \caption{Расширенный алгоритм Евклида.}
  \begin{algorithmic}[1]
    \Function{extended\_$\gcd$}{int $a$, int $b$} \Comment --- возвращаем тройку чисел $(x, y, \gcd(a, b))$.
      \If{$b = 0$}
        \State return $(1, 0, a)$;
      \EndIf
      \Let{$(x', y', d)$}{EXTENDED\_$\gcd(b, a \Mod b)$}
      \State return $(y', x' - \left\lfloor\frac{a}{b}\right\rfloor y', d)$
    \EndFunction
  \end{algorithmic}
\end{algorithm}

\begin{Lemma}
  Для произвольных неотрицательных чисел $a$ и $b$ $(a \geqslant b)$ расширенный
алгоритм Евклида возвращает целые числа $x, y, d$, для которых $\gcd(a, b) = d = ax + by$.
\end{Lemma}

\begin{proof}
  Если не рассматривать $x, y$ в алгоритме, то такой алгоритм полностью
  повторяет обычный алгоритм Евклида. Поэтому алгоритм 3-им параметром 
  действительно вычислит $\gcd(a, b)$.

  Про корректность $x, y$ будет вести индукцию по $b$. Если $b = 0$, тогда мы действительно вернём верное значение. Шаг индукции: заметим, что алгоритм находит $\gcd(a, b)$, произведя рекурсивный вызов для $(b, a \Mod b)$. Поскольку $(a \Mod b) < b$, мы можем
  воспользоваться предположением индукции и заключить, что для возвращаемых рекурсивным вызовом чисел $x', y'$ выполняется равенство:

  \[
    \gcd(b, a \Mod b) = bx' + (a \Mod b)y'
  \]

  Понятно, что $a \Mod b = a - \left\lfloor\frac{a}{b}\right\rfloor b$, поэтому
  \begin{multline*}
    d = \gcd(a, b) = \gcd(b, a \Mod b) = bx' + (a \Mod b)y' =\\
    = bx' + \left(a - \left\lfloor\frac{a}{b}\right\rfloor b\right)y' = 
    ay' + b\left(x' - \left\lfloor\frac{a}{b}\right\rfloor y'\right)
  \end{multline*}
\end{proof}

\begin{Examples}
  Мы умеем с помощью расширенного алгоритма Евклида вычислять обратные остатки по
  простому модулю (в поле $\mathbb{F}_p$). Действительно, если $(a, p) = 1$ то
  существуют $x, y$, что $ax + py = 1$, а значит в поле $\mathbb{F}_p$ ---
  $ax = 1$, откуда $x = a^{-1}$.
\end{Examples}

\subsection{Алгоритм быстрого возведения в степень по модулю}

Хотим вычислить $a^b \Mod p$. Основная идея в том, чтобы разложить $b$ в двоичную
систему и вычислять только $a^{2^i} \Mod p$. Здесь будем
предполагать, что операции с числами выполняются достаточно быстро.
Приведём ниже псевдокод такого алгоритма:

\begin{algorithm}
  \caption{Алгоритм быстрого возведения в степень.}
  \begin{algorithmic}[1]
    \Function{fast\_pow}{int $a$, int $b$, int $p$} \Comment --- возвращаем $a^b \Mod p$.
    \If{$b = 0$}
      \State return 1;
    \EndIf
    \If{$b \Mod 2 = 1$}
      \State return FAST\_POW($a, b - 1, p) \cdot a \Mod p$
    \Else
      \Let{$c$}{FAST\_POW($a, b/2, p$)}
      \State return $c^2 \Mod p$
    \EndIf
    \EndFunction
  \end{algorithmic}
\end{algorithm}  

Корректность этого алгоритма следует из того, что $a^b = a^{b - 1}\cdot a$ 
для нечетных $b$ и 

$a^b = a^{b/2} \cdot a^{b/2}$ для четных $b$. Также мы
здесь неявно пользуемся индукцией по $b$, в которой корректно возвращается база при $b = 0$.

От каждого числа $b$, если оно четно, мы запускаем наш алгоритм от $b/2$, а если
оно нечетно, то от $b - 1$, откуда получаем, что количество
действий, совершенным нашим алгоритмом будет не более, чем $2\log b = O(\log b)$.

\begin{Commentary}
  На самом деле быстрое возведение в степень работает на всех ассоциативных операциях.
  Например, если вы хотите вычислить $A^n$, где $A$ --- квадратная матрица, то
  это можно сделать тем же самым алгоритмом
  за $O(T(m)\log n)$, где $T(m)$ асимптотика перемножения
  матриц $m \times m$.
\end{Commentary}

\subsection{Китайская теорема об остатках и её вычисление}

Китайская теорема об остатках звучит так --- пусть даны попарно взаимно простые
модули и числа $r_1, \ldots, r_n$. Тогда существует единственное с точностью по
модулю $a_1 \ldots a_n$ решение такой системы:

\[
  \begin{cases}
    x \equiv r_1 \pmod{a_1} \\
    x \equiv r_2 \pmod{a_2} \\
    ~~~~\vdots \\
    x \equiv r_n \pmod{a_n} \\
  \end{cases}
\]

\begin{proof}
  Докажем и предъявим сразу алгоритм вычисления за $O(n \log \max(a_1, \ldots, a_n))$.

  Пусть $x = \sum\limits_{i = 1}^n r_i M_i M_i^{-1}$, где $M_i = \dfrac{a_1\ldots a_n}{a_i},
  M_i^{-1}$ это обратное к $M_i$ по модулю $a_i$ (такое всегда найдётся из
  попарной взаимной простоты). Прошу заметить, что такое
  число мы можем вычислить за $O(n \log \max(a_1, \ldots, a_n))$(см. пример в 
  расширенном алгоритме Евклида). 

  Докажем, что это число подходит по любому модулю $a_i$.
  \[
    x \equiv \sum\limits_{j = 1}^n r_jM_jM_j^{-1} \equiv r_iM_iM_i^{-1} \equiv r_i \pmod{a_i}
  \]
  Второе равенство следует из того, что $a_i \ | \ M_j$ при $j \neq i$ (из построения).

  Докажем единственность решения по модулю. Пусть $x, x'$ --- различные решения
  данной системы, тогда $0<|x - x'| < a_1 \ldots a_n$ и $|x  - x'|$ делится
  на $a_1 \ldots a_n$, что невозможно, так как ни одно положительное число до
  $a_1 \ldots a_n$ не делится на $a_1 \ldots a_n$.
\end{proof}

\subsection{Решето Эратосфена}

Решето Эратосфена --- это один~из первых алгоритмов в истории человечества. Он
позволяет найти все простые числа на отрезке от $[1; n]$ за $O(n\log\log n)$, а
разложить все числа на простые множители за $O(n\log n)$

В первом случае у нас задача состоит в том, чтобы вернуть 1, если число простое 
и 0, если непростое.

Предъявим псевдокод такого алгоритма:

\begin{algorithm}
  \caption{Решето Эратосфена.}
  \begin{algorithmic}[1]
    \Function{Sieve\_of\_Eratosthenes}{int $n$} \Comment найти --- массив $prime_i$,
    означающий характеристическую функцию простых чисел от $1$ до $n$.
    \For{$i \gets 1 \textbf{ to } n$}
      \Let{$prime_i$}{$true$}
    \EndFor
    \Let{$prime_1$}{$false$}
    \For{$i \gets 2 \textbf{ to } n$}
      \If{$prime_i = true$}
        \Let{$j$}{$2i$}
        \While{$j \leqslant n$}
          \Let{$prime_j$}{$false$}
          \Let{$j$}{$j + i$}
        \EndWhile
      \EndIf
    \EndFor
    \EndFunction
  \end{algorithmic}
\end{algorithm}

\begin{Lemma}
 Алгоритм Sieve\_of\_Eratosthenes корректно оставит все простые числа.
\end{Lemma}

\begin{proof}
  Докажем по индукции по $n$. База $n = 2$ очевидна.

  Переход $n \to n + 1$. Заметим, что наш алгоритм и корректно завершит
  для $n$ чисел, потому что мы только расширяем область рассматриваемых чисел.

  Если $n + 1$ составное, тогда $n + 1 = p \cdot m $ для какого-то простого $p < n + 1$.
  По предположению индукции мы рассмотрим простое число $p$
  правильно, то есть удалим из массива все числа, которые
  кратны $p$, а значит и $n + 1$ мы правильно уберём.

  Если $n + 1$ простое, то если мы его убрали на каком-то шаге, то оно делилось
  на то простое, которые мы рассматривали до этого, но это противоречит определению
  простых чисел.
\end{proof}

Заметим, что алгоритм будет выполняться за время

\[
  \sum_{\substack{p \leqslant n,\\ p \text{ --- простое}}} \frac{n}{p}
\]

Потому что для каждого простого числа мы рассматриваем в таблице все числа,
кратные $p$.

Можно оценить очень грубо и получим, что

\[
  \sum_{\substack{p \leqslant n,\\ p \text{ --- простое}}} \frac{n}{p} \leqslant \sum_{i = 1}^{n} \frac{n}{i} \approx n\ln n + o(n) = O(n \log n)
\]

Но используя свойства ряда $\sum\limits_{\substack{p \leqslant n,\\ p \text{ --- простое}}} \frac{n}{p}
\approx n\ln\ln n + o(n)$, следует, что алгоритм работает за $O(n \log\log n)$, 
но факт про асимпототику этого ряда мы оставим без доказательства.

Если теперь первый раз, приходя в составное число в алгоритме, хранить его
наименьший простой делитель, то рекурсивно мы можем разложить
число на простые множители. Всего количество простых делителей у числа не может превышать $O(\log n)$ 
(так как самый наименьший простой делитель это двойка), поэтому разложение
на множители будет выполняться за $O(n \log n)$.

\subsection{Решето Эйлера}

Составим двусвязный список из чисел от $2$ до $n$, а также ещё массив длиной
$n$ с указателями на каждый элемент.

Будем идти итеративно: первый непросмотренный номер в списке берётся как простое
число, и определяются все произведения с последующими элементами в списке (само на себя тоже умножим), пока
не выйдем в произведении за пределы $n$. После этого удаляются
все числа, которые мы вычислили (смотрим в массив укзателей и удаляем по указателю
за $O(1)$) и повторяем процедуру.

\begin{Lemma}
После $k$ шагов алгоритма останется первых $k$ простых чисел в начале и в списке
будут только числа взаимно простые с первыми $k$.
\end{Lemma}

\begin{proof}

  База при $k = 1$ очевидна. Просто убираем все четные числа.

  Переход $k \to k + 1$.

  Докажем, что следующим нерассмотренным элементом списка мы возьмём $p_{k + 1}$. Действительно, простые
  числа мы не выкидываем, а значит следующим шагом после $p_k$
  мы возьмём число, не большее $p_{k + 1}$, но по предположению
  индукции все числа от $(p_k, p_{k + 1})$ были убраны, так как
  они составные и содержат в разложении только простые, меньшие $p_{k + 1}$.

  Предположим, что после ещё одного шага алгоритма у нас осталось число, кратное
  $p_{k + 1}$ (и большее $p_{k + 1}$)
  (все числа, делящиеся на предыдущие простые до этого были убраны).

  Тогда пусть это будет $m = p_{k + 1} \cdot a, a > 1$. Если $a$ содержит в разложении
  на простые хотя бы одно число, меньшее $p_{k + 1}$, то получим противоречие, так
  как все числа не взаимно простые с $p_1, \ldots, p_k$ по предположению индукции были убраны.

  Значит $a$ содержит в разложении на простые числа, не меньшие $p_{k + 1}$, а значит
  $a \geqslant p_{k + 1}$ и это число ещё было в списке, значит мы это число уберем, противоречие.
\end{proof}

Если мы вдруг на шаге алгоритма получили в умножении число, которое мы уже убрали, то
значит у этого числа есть меньший простой делитель, чем $p_{k}$, но по доказанной лемме у нас все такие числа к $k$-ому шагу
были убраны. Значит каждое составное число мы рассмотрим ровно 1 раз и ровно
1 раз уберем за $O(1)$.

Также по лемме получаем, что в начале списка останутся только простые числа.

Простые числа мы тоже рассматриваем по 1 разу в нашем алгоритме, значит
общая сложность решета Эйлера будет $O(n)$.

\subsection{Наивная факторизация числа за $O(\sqrt{n}\,)$}

На данный момент не существует алгоритма факторизации числа за
полином от размера числа, а не от значения. Здесь мы рассмотрим наивный алгоритм
факторизации числа. На следующей лекции рассмотрим $\rho$-метод Полларда, который работает за $O(\sqrt[4]{n}\,)$.

Пусть $n' = n$.
Будем перебирать от $2$ до $\left\lceil \sqrt{n'} \right\rceil$ числа и пока
текущее $n$ делится на данное число, делим $n$ на это число.

Легко показать, что делим мы только на простые числа (иначе мы поделили бы на
меньшее простое несколькими шагами раньше).

В конце $n$ будет либо 1 (тогда факторизация удалась), либо простым. Составным
оно не может быть, иначе $n = ab, a, b > 1$ и $a, b > \left\lceil \sqrt{n'}\right\rceil$, 
так как на все числа, меньшие корня, мы поделили.

Осталось оценить, сколько операций раз мы обращаемся к циклу $while$. В нём мы делаем
суммарно не более, чем $O(\log n + \sqrt{n}\,)$ действий, так как сумма
степеней при разложении числа не более, чем $O(\log n)$ (см. выше). Ну а также
обращаемся по 1 разу каждый шаг внешнего цикла.

\end{document}
