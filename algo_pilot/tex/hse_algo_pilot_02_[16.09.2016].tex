\documentclass[a4paper, 12pt]{article}
\usepackage{header}

\begin{document}
\pagestyle{fancy}

\section{Лекция 3 от 16.09.2016. Алгоритм Карацубы, алгоритм Штрассена}

\subsection{Перемножение 2 длинных чисел с помощью FFT}

Пусть $x = \overline{x_1 x_2 \ldots x_n}$ и $y = \overline{y_1 y_2 \ldots y_n}$. Распишем их умножение в столбик:
\begin{center}
  \[
  \renewcommand{\arraystretch}{0.8}
  \arraycolsep=1pt
  \begin{array}{r}
  \times\begin{array}{rrrr}
  x_1 & x_2 & \ldots & x_n \\
  y_1 & y_2 & \ldots & y_n \\
  \hline
  \end{array}
  \\
  +\begin{array}{rrrrrrr}
  & & & z_{11} & z_{12} & \ldots & z_{1n} \\
  & & z_{21} & z_{22} & \ldots & z_{2n} & \\
  & \hdotsfor{4} & & \\
  z_{n1} & z_{n2} & \ldots & z_{nn} & & & \\
  \hline
  \end{array}
  \\
  \begin{array}{rrrrrrrr}
  z_{1} & z_{2} & \dots & \dots & \dots & \dots & \dots & z_{2n}\\
  \end{array}
  \end{array}\]
\end{center}

Понятно, что наивное умножение 2 длинных чисел будет иметь сложность $O(n^2)$.

Давайте научимся перемножать 2 числа быстрым преобразованием Фурье за $O(n\log n)$.

Пусть $a = \overline{a_{n - 1}\ldots a_0}, b = \overline{b_{n - 1}\ldots b_0}$.

Тогда введём многочлены $f(x) = \sum\limits_{i = 0}^{n - 1} a_ix^i,
g(x) = \sum\limits_{i = 0}^{n - 1} b_ix^i$.

За $O(n\log n)$ мы можем найти $h(x) = (f(x) \cdot g(x)) = \sum\limits_{i = 0}^{2n - 2} c_i x^i$.

После этого надо аккуратно провести переносы разрядов таким образом:

\begin{algorithm}
  \caption{Умножение 2 длинных чисел.}
  \begin{algorithmic}[1]
    \Function{Умножение 2 длинных чисел}{$h(x)$} \Comment{$h(x)$ --- 
    перемножение 2 многочленов $f(x)$ и $g(x)$.}
      \Let{$carry$}{0}
      \For{$i = 0 \text{ to } 2n - 2$}
        \Let{$h_i$}{$h_i + carry$}
        \Let{$carry$}{$\left\lfloor\frac{h_i}{10}\right\rfloor$}
        \Let{$h_i$}{$h_i \mod 10$}
      \EndFor
    \EndFunction
  \end{algorithmic}
\end{algorithm}

Но этот метод плохо применим на практике из-за того, что быстрое преобразование
Фурье имеет очень большую константу.

\subsection{Алгоритм Карацубы}

Какое-то время человечество не знало алгоритмов перемножения быстрее, чем за
$O(n^2)$. А.Н. Колмогоров считал, что это вообще невозможно. В один момент
собрались математики на мехмате МГУ и решили доказать, что это невозможно. Но
один из аспирантов (Анатолий Алексеевич Карацуба) Колмогорова пришёл и сказал, что у него получилось сделать
это быстрее. Давайте посмотрим, как:

Будем считать, что $n = 2^k$ (если это не так, дополним нулями, сложность вырастет лишь в константу раз).

Для начала просто попробуем воспользоваться стратегией <<Разделяй и властвуй>>. Разобьём числа в 
разрядной записи пополам. Тогда
\[\begin{array}{c}
\times \begin{cases}
x = 10^{n/2}a + b\\
y = 10^{n/2}c + d\\
\end{cases} \\
\Downarrow\\
xy = 10^{n}ac + 10^{n/2}(ad+bc)+bd
\end{array}\]

Как видно, получается 4 умножения чисел размера $\frac{n}{2}$. Так как сложение имеет сложность $\Theta(n)$, то

\[T(n) = 4T\left( \frac{n}{2} \right) + \Theta(n)\]

Чему равно $T(n)$? Если посмотреть на дерево исходов или воспользоваться 
индукцией, то получим, что $T(n) = O(n^2)$, что, конечно,
неэффективно.

Анатолий Алексеевич проявил недюжие способности и предложил
следующее:

Разложим $(a + b)(c + d)$:

\[(a+b)(c+d) = ac+(ad+bc) + bd \implies ad + bc = (a + b)(c + d) - ac - bd\]

Подставим это в начальное выражение для $xy$:

\[xy = 10^{n}ac + 10^{n/2}((a + b)(c + d) - ac - bd)+bd\]

Отсюда видно, что достаточно посчитать три числа размера $\frac{n}{2}$: $(a + b)(c + d), ac$ и $bd$. Тогда:

\[T(n) = 3T\left( \frac{n}{2} \right) + \Theta(n)\]

Докажем, что $T(n) = O(n^{\log_2 3})$.

Рассмотрим дерево исходов: в каждой вершине дерева мы выполняем не более
$Cm$ действий, где $C$ ---какая-то фиксированная константа, а $m$ --- размер числа на данном шаге, поэтому 

\[
  T(n) \leqslant Cn\left(1 + \frac{3}{2} + \ldots + \frac{3^{\log_2 n}}{2^{
  \log_2 n}}\right), \text{ так как
  на каждом шаге мы запускаемся 3 раза от задачи в 2 раза меньше.}
\]

Откуда $T(n) \leqslant Cn \cdot \frac{\frac{3}{2}^{\log_2 n} - 1}{1/2} = 
2Cn^{\log_2 3} = O(n^{\log_2 3}) \approx O(n^{1.5849})$

Полученный алгоритм называется \emph{алгоритмом Карацубы}.

\subsection{Перемножение матриц. Алгоритм Штрассена}

После идеи А.А. Карацубы, появились многие алгоритмы, использующие ту же идею.
Одним из этих алгоритмов является алгоритм Штрассена. Будем считать, что $n = 
2^k$ снова (оставляем читалелю самим подумать, как дополнить матрицы $m \times t,
t \times u$, чтобы потом лего восстановить ответ)

Пусть у нас есть квадратные матрицы
\[A = \begin{pmatrix}
a_{11} & a_{12} & \ldots & a_{1n} \\
a_{21} & a_{22} & \ldots & a_{2n} \\
\vdots & \vdots & \ddots & \vdots \\
a_{n1} & a_{n2} & \ldots & a_{nn} \\
\end{pmatrix}
\text{и } 
B = \begin{pmatrix}
b_{11} & b_{12} & \ldots & b_{1n} \\
b_{21} & b_{22} & \ldots & b_{2n} \\
\vdots & \vdots & \ddots & \vdots \\
b_{n1} & b_{n2} & \ldots & b_{nn} \\
\end{pmatrix}\]

Сколько операций нужно для умножения матриц? Умножим их по определению. Матрицу
$C = AB$ заполним следующим образом:
\[c_{ij} = \sum\limits_{k = 1}^{n} a_{ik}b_{kj}\]

Всего в матрице $n^2$ элементов. На получение каждого элемента уходит $O(n)$ операций (умножение за константное время и сложение $n$ элементов). Тогда умножение требует $n^2O(n) = O(n^3)$
операций.

Попробуем применить аналогичную стратегию «Разделяй и властвуй». Представим
матрицы $A$ и $B$ в виде:

\[A = \begin{pmatrix}
A_{11} & A_{12}\\
A_{21} & A_{22}
\end{pmatrix}
\text{и } 
B = \begin{pmatrix}
B_{11} & B_{12}\\
B_{21} & B_{22}
\end{pmatrix}\]
где каждая матрица имеет размер $\frac{n}{2}$. Тогда матрица $C$ будет иметь вид:
\[C = \begin{pmatrix}
A_{11}B_{11}+A_{12}B_{21} & A_{11}B_{12}+A_{12}B_{22}\\
A_{21}B_{11}+A_{22}B_{21} & A_{21}B_{12}+A_{22}B_{22}
\end{pmatrix}\]
 Как видно, получаем 8 перемножений матриц порядка $\frac{n}{2}$. Тогда

\[T(n) = 8T\left( \frac{n}{2} \right) + O(n^2)\]

По индукции получаем, что $T(n) = O\left(n^{\log_{2} 8}\right) = O(n^{3})$.

Можно ли уменьшить число умножений до 7? \emph{Алгоритм Штрассена} утверждает,
что можно. Он предлагает ввести следующие матрицы (даже не спрашивайте, как до них дошли):

\[\begin{cases}
    M_1 = (A_{11} + A_{22})(B_{11} + B_{22}); \\
    M_2 = (A_{21} + A_{22})B_{11}; \\
    M_3 = A_{11}(B_{12} - B_{22}); \\
    M_4 = A_{22}(B_{21} + B_{11}); \\
    M_5 = (A_{11} + A_{12})B_{22}; \\
    M_6 = (A_{21} - A_{11})(B_{11} + B_{12}); \\
    M_7 = (A_{12} - A_{22})(B_{21} + B_{22}); \\
\end{cases}\]
Тогда
\[\begin{cases}
    C_1 &= M_1+M_4-M_5+M_7; \\
    C_2 &= M_3+M_5; \\
    C_3 &= M_2+M_4; \\
    C_4 &= M_1-M_2+M_5+M_6; \\
\end{cases}\]

Можно проверить что всё верно (оставим это как \sout{наказание} упражнение читателю). Сложность алгоритма:

\[T(n) = 7T\left( \frac{n}{2} \right) + O(n^2) \implies T(n) = O\left(n^{\log_{2} 7} \right) \approx O(n^{2.8073})\]

Доказательство времени работы такое же, как и в алгоритме Карацубы.

Также существует модификация алгоритма Штрассена, где используется лишь 15
сложений матриц на каждом шаге, вместо 18 предъявленных выше.

\subsection{Эквивалентность асимптотик некоторых алгоритмов}

\emph{Этот раздел не войдёт в экзамен.}

Здесь мы поговорим об обращении и перемножении 2 матриц. Докажем, что асимптотики
этих алгоритмов эквивалентны.

\begin{Theorem}[Умножение не сложнее обращения]
  Если можно обратить матрицу размеров $n \times n$ за время $T(n)$,
  где $T(n) = \Omega(n^2)$, и $T(3n) = O(T(n))$ (условие регулярности), то две
  матрицы размером $n \times n$ можно перемножить за время $O(T(n))$
\end{Theorem}

\begin{proof}
  Пусть $A$ и $B$ матрицы одного порядка размера $n \times n$. Пусть

  \[D = 
  \begin{pmatrix}
      I_n & A & 0\\
      0 & I_n & B\\
      0 & 0 & I_n
  \end{pmatrix}\]

  Тогда легко понять, что
  \[
    D^{-1} =
    \begin{pmatrix}
      I_n & -A & AB\\
      0 & I_n & -B \\
      0 & 0 & I_n
    \end{pmatrix}
  \]

  Матрицу $D$ мы можем построить за $\Theta(n^2)$, которое является $O(T(n))$,
  поэтому с условием регулярности получаем, что $M(n) = O(T(n))$, где $M(n)$ ---
  асимптотика перемножения 2 матриц.
\end{proof}

С обратной теоремой предлагаем ознакомиться в книге Кормена или Ахо, Хопкрофта и Ульмана.

% Давайте поговорим о том, что обращение матрицы не сложнее умножения.

% Далее будем считать, что матрица $A$ имеет размеры  $2^k \times 2^k = n \times n$.

% \begin{Lemma}
%   Пусть матрица $A$ разбита на 4 одинаковых матрицы так:
%   \[
%     A =
%     \begin{pmatrix}
%       A_{11} & A_{12}\\
%       A_{21} & A_{22}
%     \end{pmatrix}
%   \]

%   Предположим, что существует $A_{11}^{-1}$, пусть $\Delta = A_{22} - 
%   A_{21}A_{11}^{-1}A_{12}$. Предположим, что $\Delta^{-1}$ существует. Тогда

%   \[
%     A^{-1} = 
%     \begin{pmatrix}
%       A_{11}^{-1} + A_{11}^{-1}A_{12}\Delta^{-1}A_{21}A_{11}^{-1} & -A_{11}^{-1}A_{12}\Delta^{-1}\\
%       -\Delta^{-1}A_{21}A_{11}^{-1} & \Delta^{-1}
%     \end{pmatrix}
%   \]
% \end{Lemma}

% \begin{proof} 
%   Можно убедиться в том, что такое равенство выполнено:
%   \[
%     A = 
%     \begin{pmatrix}
%       A_{11} & A_{12}\\
%       A_{21} & A_{22}
%     \end{pmatrix} =
%     \begin{pmatrix}
%       I_{n/2} & 0\\
%       A_{21}A_{11}^{-1} & I_{n/2}
%     \end{pmatrix}
%     \begin{pmatrix}
%       A_{11}& 0\\
%       0 & \Delta
%     \end{pmatrix}
%     \begin{pmatrix}
%       I_{n/2}& A_{11}^{-1}A_{12}\\
%       0 & I_{n/2}
%     \end{pmatrix}
%   \]

%   Воспользуемся фактом из линейной алгебры, что $(ABC)^{-1} = C^{-1}B^{-1}A^{-1}$, откуда

%   \[
%     A^{-1} = 
%     \begin{pmatrix}
%       I_{n/2}& -A_{11}^{-1}A_{12}\\
%       0 & I_{n/2}
%     \end{pmatrix}
%     \begin{pmatrix}
%       A_{11}^{-1}& 0\\
%       0 & \Delta^{-1}
%     \end{pmatrix}
%     \begin{pmatrix}
%       I_{n/2} & 0\\
%       -A_{21}A_{11}^{-1} & I_{n/2}
%     \end{pmatrix} = 
%     \begin{pmatrix}
%       A_{11}^{-1} + A_{11}^{-1}A_{12}\Delta^{-1}A_{21}A_{11}^{-1} & -A_{11}^{-1}A_{12}\Delta^{-1}\\
%       -\Delta^{-1}A_{21}A_{11}^{-1} & \Delta^{-1}
%     \end{pmatrix}
%   \]
% \end{proof}

% \begin{Lemma}
%   Если $A$ --- невырожденная верхняя треугольная матрица, то матрицы $A_{11}$ и
%   $\Delta$ из Леммы 1 обратимы и являются верхними треугольными.
% \end{Lemma}

% \begin{proof}
%   Ясно, что $A_{11}$ невырожденная верхняя треугольная матрица и, значит, что
%   $A_{11}^{-1}$ существует. Далее заметим, что 
% \end{proof}

\end{document}
