\section{Немного теории вероятностей}

\textbf{Disclaimer 1.} Это первая из двух глав теории вероятностей. В них мы будем говорить исключительно о дискретной (с конечными вероятностными пространствами) вероятности, так как в алгоритмах ничего другого особо не понадобится. Кроме того, эти две главы не заслуживают даже названия «Начала теории вероятностей», так как являются совсем частным случаем общей теории вероятностей. Несмотря на это, изложенный в них кусочек начала теорвера достаточен для того, чтобы понимать, что происходит. \par

\textbf{Disclaimer 2.} Иногда (примерно всегда) мы будем ставить знак «$=$» между числами тогда, когда его ставить в строгом смысле не очень честно. Однако, поскольку нас всюду интересует ассимптотическое поведение величин, то этот знак будет стоять вполне себе честно. \par

\begin{definition}
    \textit{Вероятностное пространство} $\left( \Omega, 2^{\Omega}, \P \right)$ -- структура, состоящая из:
    \begin{itemize}
        \item $\Omega$ -- множество элементарных исходов (конечное);
        \item $2^{\Omega}$ -- множество всевозможных событий (наборов исходов);
        \item $\P$ -- функция вероятности $\P: 2^{\Omega} \to [0, 1]$ такая, что $\P(\Omega) = 1$.
    \end{itemize}
\end{definition}

Теперь договоримся о некоторых обозначениях и лексике. Вместо $\P(\lbrace\omega\rbrace)$ мы всегда будем писать $\P(\omega)$ и называть это \textit{вероятностью исхода $\omega$}. Если вероятность какого-то события $B \in 2^{\Omega}$ равна нулю, то есть $\P(B) = 0$, то событие $B$ называется \textit{невозможным}. Если же $\P(B) = 1$, то $B$ называется \textit{достоверным} событием. \par

Через $\P(A \vert B)$ будем обозначать вероятность события при условии наступления события $B$. Такая вероятность считается по формуле
\[
    \P(A \vert B) = \frac{\P(A \cap B)}{\P(B)}
\]

\begin{definition}
    \textit{Независимыми} называются события $A$ и $B$, если $\P(A \vert B) = \P(A)$ или $\P(A) \cdot \P(B) = \P(A \cap B)$. \par
    \textit{Несовместными} называются события, для которых $\P(A \vert B) = 0$.
\end{definition}

\begin{definition}
    $A_1, \ldots, A_n$ - множство попарно несовместных событий, то есть $\forall i, j \ \P(A_i \vert A_j) = 0$. Если $\P(\bigcup \limits_{i = 1}^{n} A_i) = 1$, тогда набор $\set{A_i}_{i = 1}^{n}$ называется \textit{полной группой событий}.
\end{definition}

Тогда если есть событие $B$ и полная группа событий $\set{A_i}_{i = 1}^{n}$, тогда $B = \bigcup \limits_{i = 1}^{n} B \cap A_i$ и
\[
    \P(B) = \sum \limits_{i = 1}^{n} \P(B \cap A_i) = \sum \limits_{i = 1}^{n} \P(B \vert A_i) \cdot \P(A_i) 
\]