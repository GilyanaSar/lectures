\section{Семинар от 09.09.2016}
Перед тем, как начать решать задачи, кратко опишем вероятностное пространство для броска \(n\)-гранного кубика: \(\Omega = \{\omega_1, \omega_2, \ldots, \omega_n\}, \omega_i = \{\text{выпало число }i\}\), \(P(\omega_i) = \frac{1}{n}\) для всех \(i\).

\begin{problem}
    Пусть бросаются \(n\)-гранный и \(m\)-гранный кубики. Какова вероятность \(\Pr\) того, что выпадет одно чётное и одно нечётное число?
\end{problem}
\begin{proof}[Решение]
    В данной задаче есть два случая:
    \begin{enumerate}
        \item На первом выпало чётное число очков, а на втором~--- нечётное. Количество чётных чисел от 1 до \(n\) равно \(\left\lfloor\frac{n}{2}\right\rfloor\), а нечётных чисел от 1 до \(m\)~--- \(\left\lceil\frac{m}{2}\right\rceil\). Тогда есть \(\left\lceil\frac{m}{2}\right\rceil\left\lfloor\frac{n}{2}\right\rfloor\) успешных исходов.
        \item На первом выпало нечётное число очков, на втором~--- чётное. Аналогичными рассуждениями получаем \(\left\lfloor\frac{m}{2}\right\rfloor\left\lceil\frac{n}{2}\right\rceil\) успешных исходов.
    \end{enumerate}
    Всего же исходов \(mn\). Следовательно, \[\Pr =\frac{\left\lceil\frac{m}{2}\right\rceil\left\lfloor\frac{n}{2}\right\rfloor + \left\lfloor\frac{m}{2}\right\rfloor\left\lceil\frac{n}{2}\right\rceil}{mn}.\qedhere\]
\end{proof}

\begin{problem}
    Пусть бросаются два \(n\)-гранных кубика. Какова вероятность \(\Pr(i)\) того, что суммарно выпадет \(2 \leq i \leq 2n\) очков?
\end{problem}
\begin{proof}[Решение]
    В данной задаче есть два случая:
    \begin{enumerate}
        \item \(i \leq n + 1\). Представим \(i\) в следующем виде: \(i = k + (i - k)\), где \(1 \leq k \leq i - 1\). Такое ограничение сверху на \(k\) объясняется тем, что иначе \(i - k\) будет меньше 1, а при броске кубика не может выпасть меньше 1 очка. Ограничение снизу объясняется аналогично. Тогда есть \(i - 1\) подходящий случай.
        \item \(n + 2 \leq i \leq 2n\). Опять же, представим \(i\) в виде \(i = k + (i - k)\). Теперь определим границы для \(k\). Очевидно, что \(k \leq n\). Так как \(i - k \leq n\), то \(k \geq i - n\). Тогда получаем \(i - n \leq k \leq n\). Тогда есть \(n - (i - n) + 1 = 2n - i + 1\) подходящий случай.
    \end{enumerate}
    Так как всего есть \(n^2\) разных вариантов того, сколько очков выпадет на кубиках, то получаем, что
    \[\Pr(i) = \begin{cases}
    \frac{i - 1}{n^2} & i \leq n + 1, \\
    \frac{2n - i + 1}{n^2} & n + 2 \leq i \leq 2n.
    \end{cases}\qedhere\]
\end{proof}
\begin{remark}
    Если нарисовать график функции \(\Pr(i)\), то он будет выглядеть, как треугольник с вершиной в точке \(\left(n + 1, \frac{1}{n}\right)\). Такой график называют \emph{треугольным распределением}.
\end{remark}

Перейдём от кубиков к монеткам.

\begin{problem}
    Пусть последовательно бросают \(n\) монет (полагается, что \(\Omega = \{\text{О}, \text{Р}\}\)). Какова вероятность \(\Pr\) того, что не выпадет последовательно
    \begin{enumerate}
        \item орёл и решка?
        \item два орла?
    \end{enumerate}
\end{problem}
\begin{proof}[Решение]
    Как рассказывалось ранее, в такой модели есть \(2^n\) элементарных исходов. Посчитаем количество успешных исходов в каждом случае:
    \begin{enumerate}
        \item В таком случае легко понять, что будут допустимы только последовательности вида \(\underbrace{\text{P}\text{P}\ldots\text{P}}_{k\text{ раз}}\underbrace{\text{О}\text{О}\ldots\text{О}}_{n - k\text{ раз}}\), где \(0 \leq k \leq n\). Тогда есть \(n + 1\) подходящий исход.
        
        \item Пусть \(f_n\)~--- количество последовательностей длины \(n\), в которых нет двух орлов подряд. Как посчитать \(f_n\)? Попробуем выразить рекурсивно. Если при последнем броске выпал орёл, то при предпоследнем обязательно выпала решка. То, что идёт до решки, явно угадать невозможно. Но нам известно, что это последовательность размера \(n - 2\) и в ней нет двух орлов подряд. Тогда их \(f_{n - 2}\) вариантов. Если же выпала решка, то есть \(f_{n - 1}\) вариант. Отсюда получаем, что \(f_{n} = f_{n - 1} + f_{n - 2}\). Так как \(f_1 = 2\), а \(f_2 = 3\) (допускаются ОР, РО, РР), то \(f_n = F_{n + 2}\), где \(F_n\)~--- \(n\)-е число Фибоначчи.
    \end{enumerate}
    \textbf{Ответ:} 1) \(\Pr = \dfrac{n + 1}{2^n}\), 2) \(\Pr = \dfrac{F_{n + 2}}{2^n}\).
\end{proof}

Перед тем, как идти дальше, сделаем небольшое отступление. Во втором пункте последней задачи нам повезло, что последовательность совпала с последовательностью чисел Фибоначчи. А что делать, если не удаётся угадать последовательность? В таком случае можно воспользоваться общим методом решения. Рассмотрим его на примере из последней задачи:
\[\begin{array}{l}
f_{n} = f_{n - 1} + f_{n - 2} \\
f_{1} = 2 \\
f_{2} = 3
\end{array}\]
Выпишем \emph{характеристическое уравнение} для данного рекуррентного соотношения: \(\lambda^2 - \lambda - 1 = 0\). Находим его корни~--- в данном случае они равны \(\frac{1 \pm \sqrt{5}}{2}\). Тогда для каких-то констант \(a_1\) и \(a_2\) верно, что \[f_{n} = a_1\left(\frac{1 + \sqrt{5}}{2}\right)^n + a_2\left(\frac{1 - \sqrt{5}}{2}\right)^n.\]
Данные константы можно определить по начальным условиям.

\begin{problem}[Парадокс дней рождения]
    В группе \(27\) студентов. Считаем их дни рождения случайными и равновероятными. Какова вероятность \(P\) того, что хотя бы у двух студентов совпадают дни рождения?
\end{problem}
\begin{proof}[Решение]
    В данной задаче гораздо проще посчитать вероятность дополнения, то есть вероятность того, что у всех 27 студентов будут разные дни рождения. Так как порядок дней рождения важен, то эта вероятность равна \(\frac{A_{365}^{27}}{365^{27}}\). В итоге получаем, что \[\Pr = 1 - \frac{A_{365}^{27}}{365^{27}}.\]
    
    Хорошо, ответ получен. Но по нему сложно сказать, много ли это или мало. Попробуем посчитать его приближенно: \(\Pr = 1 - \left(1 - \frac{1}{365}\right)\left(1 - \frac{2}{365}\right)\ldots\left(1 - \frac{27}{365}\right)\). Так как \(1 + x \approx e^x\), то \(\Pr \approx 1 - e^{-\frac{1 + 2 + \ldots + 27}{365}} \approx 1 - e^{-1.04} \approx 0,66\). Как видно, вероятность достаточно велика.
\end{proof}

Сделаем небольшое теоретическое отступление. Вспомним формулу включений-исклю\-чений: \(\Pr(A \cup B) = \Pr(A) + \Pr(B) - \Pr(A \cap B)\). Попробуем придумать аналогичную формулу для трёх событий:
\[\begin{aligned}
\Pr(A \cup B \cup C) &= \Pr(A \cup B) + \Pr(C) - \Pr((A \cup B) \cap C) \\
&= \Pr(A \cup B) + \Pr(C) - \Pr((A \cap C) \cup (B \cap C)) \\
&= \Pr(A) + \Pr(B) + \Pr(C) - \Pr(A \cap B) - \Pr(A \cap C) - \Pr(B \cap C) + \Pr(A \cap B \cap C)
\end{aligned}\]
Уже видна некоторая закономерность. Сформулируем обобщение.
\begin{theorem}[Общая формула включений-исключений]
    Пусть \(A_1, A_2, \ldots, A_n\)~--- некоторые события на \(\Omega\), а \(S_k = \sum\limits_{1 \leq i_1 < i_2 < \ldots < i_k \leq n}\Pr(A_{i_1} \cap A_{i_2} \cap \ldots \cap A_{i_k})\). Тогда \[\Pr(A_1 \cup A_2 \cup \ldots \cup A_n) = \sum_{k = 1}^{n}(-1)^{k - 1}S_k.\]
\end{theorem}
\begin{proof}
    По индукции. База (\(n = 2\)) была доказана ранее. Теперь предположим, что утверждение верно для какого-то \(n\). Докажем, что из этого следует, что утверждение верно и для \(n + 1\):
    \begin{multline*}
        \Pr(A_1 \cup A_2 \cup \ldots \cup A_n \cup A_{n + 1}) = \Pr(A_1 \cup A_2 \cup \ldots \cup A_n) + \\ + \Pr(A_{n + 1}) - \Pr((A_1 \cup A_2 \cup \ldots \cup A_n) \cap A_{n + 1})
    \end{multline*}
    Так как \((A_1 \cup A_2 \cup \ldots \cup A_n) \cap A_{n + 1} = \bigcup\limits_{i = 1}^{n}(A_i \cap A_{n + 1})\), то, пользуясь предположением индукции и рассуждениями, аналогичными доказательству для трёх множеств, получаем желаемое.
\end{proof}
\begin{remark}
    Важное следствие из этой формулы: если \(A_1, A_2, \ldots, A_n\)~--- некоторые события на \(\Omega\), то по закону де Моргана получаем, что \(\Pr\left(\bigcap_{i = 1}^{n}\overline{A_i}\right) = \Pr\left(\overline{\bigcup\limits_{i = 1}^{n} A_i}\right) = 1 - \Pr\left(\bigcup\limits_{i = 1}^{n} A_i\right) = 1 - \sum\limits_{k = 1}^{n}(-1)^{k - 1}S_k\). Если положить \(S_0 = 1\), то эту формулу можно записать в виде \[\Pr\left(\bigcap\limits_{i = 1}^{n}\overline{A_i}\right) = \sum_{k = 0}^{n}(-1)^{k}S_k.\]
\end{remark}

\begin{problem}
    Пусть мы раскидали \(n\) шаров по \(m\) ящикам. Какова вероятность \(\Pr\) того, что ни один ящик не пуст? Рассмотрите случаи, когда шары различимы и неразличимы.
\end{problem}
\begin{proof}[Решение]
    Начнём со случая различимых шаров. В данном случае элементарным исходом будет \(\omega = (\omega_1, \omega_2, \ldots, \omega_m)\), где \(\omega_i\)~--- количество шаров в \(i\)-м ящике. В таком случае \(|\Omega| = \binom{n + m - 1}{m - 1}\) (cхема выбора неупорядоченных наборов с возвратом). Теперь посчитаем количество подходящих исходов. Так как ни один ящик не пуст, то в каждом из них есть хотя бы по одному шару. Тогда нужно посчитать количество способов раскидать \(n - m\) шаров по \(m\) ящикам. Это можно сделать \(\binom{n - 1}{m - 1}\) способом. Отсюда получаем, что вероятность равна \[\Pr = \dfrac{\binom{n - 1}{m - 1}}{\binom{n + m - 1}{m - 1}}.\]
    
    Теперь предположим, что шары неразличимы. Рассмотрим событие \(A_i = \{i\)-й ящик пуст\(\}\). Чему равна вероятность такого события? Для каждого из \(n\) шаров есть \(m - 1\) подходящий ящик. Тогда \(\Pr(A_i) = \frac{(m - 1)^n}{m^n} = \left(1 - \frac{1}{m}\right)^n\). Пересечение \(A_{i_1} \cap A_{i_2} \cap \ldots \cap A_{i_k}\) означает, что \(k\) ящиков с номерами \(i_1, i_2,\ldots, i_k\) пусты. Тогда \(\Pr(A_{i_1} \cap A_{i_2} \cap \ldots \cap A_{i_k}) = \left(1 - \frac{k}{m}\right)^n\). Заметим, что событие ``ни один ящик не пуст'' равно \(\overline{A_{1}} \cap \overline{A_{i_2}} \cap \ldots \cap \overline{A_{m}}\). Пользуясь формулой включений-исключений, получаем, что вероятность равна
    \[\Pr = \sum\limits_{k = 0}^{m}(-1)^{k}\binom{m}{k}\left(1 - \frac{k}{m}\right)^{n}.\qedhere\]
\end{proof}

\begin{problem}
    Алиса и Боб случайно подбрасывают \(n\) монет. Какова вероятность \(P\) того, что число орлов у Алисы будет строго больше, чем у Боба? Каков будет ответ на этот вопрос, если Алиса подбросила \(n + 1\) монету?
\end{problem}
\begin{proof}[Решение]
    Для начала посмотрим, чему равна вероятность того, что число орлов у Алисы равно числу орлов у Боба.
    Если у Алисы выпало \(k\) орлов, что достигается в \(\binom{n}{k}\) случаев, то у Боба тоже должно выпасть \(k\) орлов. Тогда достаточно логично, что число успешных исходов равно \(\sum\limits_{k = 0}^{n} \binom{n}{k}^2\). Как это упростить? Воспользуемся тем, что \(\binom{n}{k} = \binom{n}{n - k}\). Теперь представим себе следующую ситуацию: пусть есть строка, содержащая \(2n\) символов. Из первых \(n\) нужно выбрать \(k\) символов, из вторых \(n\) нужно выбрать \((n - k)\). Это можно сделать \(\binom{n}{k}\binom{n}{n - k}\) способами. Если просуммировать эти числа по \(k\) от 0 до \(n\), то легко заметить, что это то же самое, что и посчитать количество способов выбрать \(n\) символов из \(2n\). Тогда \[\sum\limits_{k = 0}^{n} \binom{n}{k}^2 = \binom{2n}{n}.\]
    Всего исходов \(4^n\) (по \(2^n\) на Алису и на Боба). Тогда вероятность равна \(\dfrac{\binom{2n}{n}}{4^n}\).
    
    Теперь рассмотрим вероятность из условия. Из-за симметричности она равна вероятности того, что у Алисы будет строго меньше орлов, чем у Боба. Тогда получаем, что \(2\Pr + \dfrac{\binom{2n}{n}}{4^n} = 1\). Отсюда \(\Pr = \dfrac{1}{2} - \dfrac{\binom{2n}{n}}{2^{2n + 1}}\).
    
    Теперь перейдём ко второму пункту. Его мы решим двумя способами~--- стандартным и ``олимпиадным''. Начнём со стандартного. Если у Алисы уже было больше орлов, чем у Боба, то что бы у неё не выпало, то ситуация не изменится. Если же было так, что у нё столько же орлов, сколько у Боба, то ей необходимо, чтобы выпал орёл. Тогда искомая вероятность равна \[\Pr' = \Pr + \dfrac{1}{2}\dfrac{\binom{2n}{n}}{4^n} = \dfrac{1}{2}.\]
    
    Теперь рассмотрим ``олимпиадный'' способ решения. Заметим, что вероятность того, что у Алисы будет больше орлов, чем у Боба, равна вероятности того, что у неё будет больше решек (из-за симметрии). При этом вероятность того, что у неё будет больше орлов, равна вероятности того, что у неё будет не больше решек, чем у Боба (пусть у неё на одного орла больше, тогда число решек у них совпадает). Отсюда сразу получаем, что \(\Pr' = \frac{1}{2}.\)
\end{proof}

А сейчас мы посмотрим, почему стоит быть осторожным с азартными играми.
\begin{problem}
    Пусть есть 52 карты, и игроку выдают 5 случайных карт. Найдите вероятности получения различных наборов из покера.
\end{problem}
\begin{proof}[Решение]
    Начнём с того, что заметим, что выбрать 5 карт из 52 мы можем \(\binom{52}{5}\) способами. Теперь достаточно найти количество подходящих исходов.
    \begin{enumerate}
        \item Royal Flush~--- туз, король, дама, валет и десятка одной масти. Есть лишь 4 подходящих исхода.
        \item Straight Flush~--- пять последовательных по достоинству карт одной масти (начиная не с туза). Так как первую карту можно выбрать 8 способами (от пятёрки до короля), то есть \(9 \cdot 4 = 36\) успешных исходов.
        \item Four Of A Kind~--- четыре карты одного достоинства. Выберем достоинство (это можно сделать 13) способами и последнюю карту (это можно сделать 48 способами). Тогда есть \(13 \cdot 48\) подходящих комбинаций.
        \item Full House~--- три карты одного достоинства и две карты другого достоинства. Выберем первое достоинство (13 вариантов) и выберем 3 карты из 4 подходящих (\(\binom{4}{3}\) способов). Теперь выберем второе достоинство (12 вариантов) и 2 карты из 4 (\(\binom{4}{2}\) вариантов). Тогда всего есть \(13 \cdot \binom{4}{3} \cdot 12 \cdot \binom{4}{2}\) вариантов.
        \item Flush~--- пять карт одной масти. Всего выбрать пять карт одной масти можно \(\binom{13}{5}\) способами. Но в таком случае мы ещё учитываем Straight Flush и Royal Flush. Тогда есть \(4(\binom{13}{5} - 10)\) вариантов.
        \item Straight~--- пять последовательных карт (не одной масти). Всего пять последовательных карт можно выбрать \(10 \cdot 4^5\) способами (сначала выбираем старшую карту, после чего масть для каждой). Но в таком случае учитывается Straight Flush и Royal Flush. Тогда есть \(10(4^5 - 4)\) подходящих наборов.
        \item Three Of A Kind~--- три карты одного достоинства. Сначала выберем достоинство (13 вариантов), после чего выберем 3 карты из 4 (\(\binom{4}{3}\) вариантов). После чего выберем два разных достоинства (иначе будет Full House), что можно сделать \(\binom{12}{2}\) способами, и масти для двух карт (\(4^2\) способа). Итого~--- \(13 \cdot \binom{4}{3} \cdot \binom{12}{2} \cdot 4^2\) варианта.
        \item Two Pair~--- две пары карт одного достоинства. Выберем достоинства и масти для двух пар (\(\binom{13}{2}\) варианта для достоинств, по \(\binom{4}{2}\) для выбора 2-х карт каждого достоинства). Осталось выбрать последнюю карту~--- это можно сделать \(52 - 8 = 44\) способами. Итого \(\binom{13}{2} \binom{4}{2}^2 \cdot 44\) исхода.
        \item One Pair~--- одна пара карт одного достоинства. Выберем достоинство и 2 карты из 4 (\(\binom{13}{1} \cdot \binom{4}{2}\) вариантов). Теперь выберем три разных достоинства и масти для карт (\(\binom{12}{3} \cdot 4^3\) варианта). Итого \(\binom{13}{1} \cdot \binom{4}{2} \cdot \binom{12}{3} \cdot 4^3\) исходов.
        \item High Card~--- ничего из вышеперечисленного. Выберем пять разных достоинств, не идущих подряд (\(\binom{13}{5} - 10\) вариантов) и выберем масти для каждой карты так, чтобы они не совпадали (\(4^5 - 4\) варианта). Итого \((\binom{13}{5} - 10)(4^5 - 4)\) вариантов.
    \end{enumerate}
    Теперь приближенно посчитаем вероятность каждого из наборов:
    
    \begin{center}
        \begin{tabular}{|c|c|}
            \hline
            Тип & Вероятность \\
            \hline
            Royal Flush & \(4/2598960 \approx 0,00015\%\) \\
            \hline
            Straight Flush & \(36/2598960 \approx 0,0014\%\) \\
            \hline
            Four Of A Kind & \(624/2598960 \approx 0,024\%\) \\
            \hline
            Full House & \(3744/2598960 \approx 0,15\%\) \\
            \hline
            Flush & \(5108/2598960 \approx 0,2\%\) \\
            \hline
            Straight & \(10200/2598960 \approx 0,39\%\) \\
            \hline
            Three Of A Kind & \(54912/2598960 \approx 2,11\%\) \\
            \hline
            Two Pair & \(123552/2598960 \approx 4,75\%\) \\
            \hline
            One Pair & \(1098240/2598960 \approx 42,26\%\) \\
            \hline
            High Card & \(1302540/2598960 \approx 50,12\%\) \\
            \hline
        \end{tabular}
    \end{center}
    Как видно из таблицы, получить что-то лучше, чем одну пару, уже не так просто.
\end{proof}