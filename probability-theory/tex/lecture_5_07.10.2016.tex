\section{Лекция от 07.10.2016}

\subsection{Скорость сходимости закона больших чисел}
Ранее мы формулировали два вопроса, связанные со скоростью сходимости закона больших чисел. Ответим на них.

\begin{enumerate}
    \item В первом пункте от нас требуется найти скорость сходимости \(\frac{S_n}{n} - p\) к нулю, где \(S_n \sim \mathrm{Bin}(n, p)\). Для этого вспомним теорему Муавра-Лапласа:
    \[\Pr{a < \frac{S_n - \E{S_n}}{\sqrt{\D{S_n}}} \leq b} \to \int_{a}^{b} \frac{1}{\sqrt{2\pi}}e^{-\frac{x^2}{2}}\,\mathrm{d}x = c, c \in (0, 1).\]
    
    Преобразуем условие вероятности, пользуясь тем, что \(\E{S_n} = np\), а \(\D{S_n} = np(1 - p)\):
    \[a < \frac{S_n - \E[S_n]}{\sqrt{\D[S_n]}} \leq b \implies \frac{a\sqrt{p(1 - p)}}{\sqrt{n}} < \frac{S_n}{n} - p \leq \frac{b\sqrt{p(1 - p)}}{\sqrt{n}}.\]
    
    Возьмём \(\delta \in (0, 1)\) такое, что \(1 - \delta < c\). Тогда при достаточно больших \(n\) выполнено (по определению предела):
    \[\Pr{\frac{c_1}{\sqrt{n}} < \frac{S_n}{n} - p \leq \frac{c_2}{\sqrt{n}}} > 1 - \delta.\]
    
    Увеличивая расстояние между \(a\) и \(b\) и увеличивая \(n\), получаем следующий результат:
    \[\lim\limits_{n, |a - b| \to \infty} \Pr{\frac{S_n}{n} - p = \mathcal{O}\left(\frac{1}{\sqrt{n}}\right)} = 1.\]
    
    Отсюда получаем, что \(\frac{S_n}{n} = p + \mathcal{O}\left(\frac{1}{\sqrt{n}}\right)\).
    
    %\item Для нахождения ответа на первый вопрос воспользуемся теоремой Муавра-Лапласа:
    %\[\forall \varepsilon > 0 \exists \delta = \delta(\varepsilon) > 0 : \Pr{-\varepsilon \leq \frac{S_n - \E{S_n}}{\sqrt{\D{S_n}}} \leq \varepsilon} \geq 1 - \delta.\]
    
    %Как было доказано ранее, \(\E{S_n} = np\), а \(\D{S_n} = np(1 - p)\). Тогда преобразуем это выражение:
    %\[\forall \varepsilon > 0 \exists \delta = \delta(\varepsilon) > 0 : \Pr{-\epsilon\sqrt{\frac{p(1 - p)}{n}} \leq \frac{S_n}{n} - p \leq \epsilon\sqrt{\frac{p(1 - p)}{n}}} \geq 1 - \delta.\]
    
    %Тогда получаем, что для любого \(\delta > 0\) существует достаточно большое \(n > n(\delta)\) такое, что выполнено
    %\[\Pr{\frac{S_n}{n} - p = \mathcal{O}\left(\frac{1}{\sqrt{n}}\right)} \geq 1 - \delta.\]
    
    %Из этого следует, что скорость сходимости \(\frac{S_n}{n} - p\) к 0 есть \(\mathcal{O}\left(\frac{1}{\sqrt{n}}\right)\).
    
    \item Теперь приступим ко второму вопросу. Для него достаточно применить неравенство Чебышёва:
    \[\Pr{\left|\frac{S_n}{n}-p\right| \geq \epsilon} \leq \frac{\D{\frac{S_n}{n}}}{\epsilon^2} =
    \frac{np(1 - p)}{n^2\epsilon^2} = \mathcal{O}\left(\frac{1}{n}\right).\]
\end{enumerate}

\subsection{Неравенство Чернова}
Следующая теорема показывает то, насколько вероятно отклонение случайной величины от её математического ожидания.
\begin{theorem}[Неравенство Чернова]
    Пусть \(S_n \sim \mathrm{Bin}(n, p)\), а \(\lambda = \E{S_n} = np\). Тогда для любого \(t > 0\) выполнено следующее:
    \[\Pr{S_n \geq \lambda + t} \leq \exp\left( -\frac{t^2}{2(\lambda + \frac{t}{3})} \right), \qquad\Pr{S_n \leq \lambda - t} \leq \exp\left( -\frac{t^2}{2\lambda} \right).\]
\end{theorem}
\begin{proof}\footnote{Краткая идея доказательства: достаём бубен и преобразуем. (А.Х.)}
    Пусть $t < n - \lambda$, тогда $\forall u > 0$:
    
    \[
    \Pr{S_n > \lambda + t} =
    \Pr{e^{uS_n} > e^{u(\lambda + t)}} \leq
    \left\{ \text{по неравенству Маркова} \right\} \leq
    \frac{\E{e^{uS_n}}}{e^{u(\lambda + t)}}
    \]
    \[
    \E{e^{uS_n}} = \sum\limits_{k = 1}^n e^{uk} C_n^k p^k (1-p)^{n-k} = (1-p+pe^u)^n
    \]
    
    Примем $x = e^u$ и будем минимизировать $\frac{\E{e^{uS_n}}}{e^{u(\lambda + t)}}$ по $x$; для этого, для
    начала, найдём производную:
    
    \begin{multline*}
        (x^{-(\lambda + t)}(1-p+px)^n)' = -(\lambda + t)x^{-(\lambda + t + 1)}(1-p+px)^n + np
        x^{-(\lambda+t)}(1-p+px)^{n-1} = 0
    \end{multline*}
    
    Что означает, что \( -(\lambda + t)(1-p+px) + npx = 0 \) и $x = \frac{(\lambda+t)(1-p)}{p(n-\lambda-t)}$.
    
    Теперь можно и к доказательству самого утверждения перейти:
    \newcommand*\circled[1]{\tikz[baseline=(char.base)]{\node[shape=circle,draw,inner sep=1pt] (char) {#1};}}
    \begin{itemize}
        \item \circled{$\leq$}
        \begin{multline*}
            \left(\frac{p(n-\lambda-t)}{(\lambda+t)(1-p)}\right)^{\lambda + t}\cdot
            \left(1-p+\frac{(\lambda+t)(1-p)}{n-\lambda-t}\right)^n =\\=
            \left(\frac{p(n-\lambda-t)}{(\lambda+t)(1-p)}\right)^{\lambda + t}\cdot
            (1-p)^n\left(\frac{n}{n-\lambda-t}\right)^n =\\=
            \frac{p^{\lambda+t}n^{\lambda+t}}{(\lambda+t)^{\lambda+t}}\cdot
            \left(\frac{n(1-p)}{n-\lambda-t}\right)^{n-\lambda-t} =\\=
            \left(\frac{\lambda}{\lambda+t}\right)^{\lambda+t}\cdot
            \left(\frac{n-\lambda}{n-\lambda-t}\right)^{n-\lambda-t} =\\=
            \exp\left( -(\lambda+t)\ln\frac{\lambda+t}{\lambda}
            -(n-\lambda-t)\ln\frac{n-\lambda-t}{n-\lambda}\right) =\\=
            \exp\left( -\lambda\phi\left( \frac{t}{\lambda} \right) - (n-\lambda)\phi\left(
            \frac{t}{n-\lambda}
            \right) \right),\ \text{где $\phi(x)=(x+1)\ln(x+1)-x$}
        \end{multline*}
        
        Заметим, что $\forall x > -1,\ \phi(x)>0$. Значит,
        \[
        \Pr{S_n \geq \lambda + t} \leq \exp\left(-\lambda\phi\left( \frac{t}{\lambda} \right) \right)
        \]
        и, аналогично,
        \[
        \Pr{S_n \leq \lambda - t} \leq \exp\left(-\lambda\phi\left( \frac{-t}{\lambda} \right)
        \right).
        \]
        
        Далее заметим, что $\phi(0) = 0$ и $\phi'(x) = 1 - 1 + \ln(x+1) = \ln(x+1) \leq x$; из этого
        следует, что для любого $y < 0$ выполняется
        
        \[
        -\phi(y) = \int\limits_y^0 \phi'(x)dx \leq \int\limits_y^0xdx = -\frac{y^2}{2}
        \]
        
        Значит, $\phi(y) \geq \frac{y^2}{2}$ и $\Pr{S_n \leq \lambda - t} \leq e^{-\frac{t^2}{2}}$.
        
        \item \circled{$\geq$}
        
        Дальше идёт полное шаманство [sic].
        
        При $x>0$,
        \[
        \phi''(x) = \frac{1}{1+x} \geq \frac{1}{\left( 1+\frac{x}{3} \right)^3} = \left(
        \frac{x^2}{2\left( 1+\frac{x}{3} \right)} \right)'' \implies
        \phi(x) \geq \frac{x^2}{2\left( 1+\frac{x}{3} \right)}
        \]
        
        Значит, $\Pr{S_n \geq \lambda + t} \leq \exp\left( -\frac{t^2}{2\left( \lambda+\frac{t}{3} \right)} \right)$
    \end{itemize}
\end{proof}


К счастью, на экзамене будет достаточно понимать идею, и не требуется точное воспроизведение всех 
вычислений.

\subsection{Алгебры событий}
\begin{definition}
    Пусть $\mathcal{A}$ --- система событий на $(\Omega, \Pr)$. Она называется \emph{алгеброй}, если
    
    \begin{enumerate}
        \item $\Omega \in \mathcal{A}$;
        \item $\forall A \in \mathcal{A},\ \overline{A} \in \mathcal{A}$;
        \item $\forall A, B \in \mathcal{A},\ A\cap B \in \mathcal{A}$;
    \end{enumerate}
\end{definition}

\noindent \textbf{Утверждение:} алгебра замкнута по основным операциям: $\cup, \setminus, \triangle$.

\noindent\textbf{Примеры:}

\begin{itemize}
    \item $\left\{ \emptyset, \Omega \right\}$;
    \item $2^{\Omega}$;
    \item $\left\{ \emptyset, \Omega, A, \overline{A}\right\}$ --- алгебра, порождённая $A$;
    \item $\alpha\left( A_1, \ldots, A_n \right) $ --- минимальная алгебра, содержащая $A_1, \ldots, A_n$.
\end{itemize}